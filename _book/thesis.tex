% ------------------------------------------------------------------------
% ------------------------------------------------------------------------
% Modelo UFSC para Trabalhos Academicos (tese de doutorado, dissertação de
% mestrado) utilizando a classe abntex2
%
% Autor: Alisson Lopes Furlani
% 	Modificações:
%	- 27/08/2019: Alisson L. Furlani, add pacote 'glossaries' para listas
% - 30/10/2019: Alisson L. Furlani, adjusted some spacing errors and changed math fonts
% - 17/01/2019: Alisson L. Furlani, updated certification page
% - 03/03/2020: Luiz F. P. Droubi, change file to be used as a template with R.
% ------------------------------------------------------------------------
% ------------------------------------------------------------------------

\documentclass[
	% -- opções da classe memoir --
	12pt,				% tamanho da fonte
	%openright,			% capítulos começam em pág ímpar (insere página vazia caso preciso)
	oneside,			% para impressão no anverso. Oposto a twoside
	a4paper,			% tamanho do papel.
	sumario=tradicional,
	% -- opções da classe abntex2 --
	%chapter=TITLE,		% títulos de capítulos convertidos em letras maiúsculas
	%section=TITLE,		% títulos de seções convertidos em letras maiúsculas
	%subsection=TITLE,	% títulos de subseções convertidos em letras maiúsculas
	%subsubsection=TITLE,% títulos de subsubseções convertidos em letras maiúsculas
	% -- opções do pacote babel --
	english,			% idioma adicional para hifenização
	%french,				% idioma adicional para hifenização
	%spanish,			% idioma adicional para hifenização
	brazil				% o último idioma é o principal do documento
	]{abntex2}

\usepackage{setup/ufscthesisA4-alf}

\DisemulatePackage{setspace}
\usepackage{setspace}

\usepackage{quoting}

\usepackage[bottom]{footmisc}

\addbibresource{bib/dis.bib}
\addbibresource{bib/not.bib}
\addbibresource{bib/pkgs.bib}

\usepackage[table]{xcolor}
\let\newfloat\undefined
\usepackage{floatrow}
\floatsetup[table]{capposition=top}
\floatsetup[figure]{capposition=top}

\newcommand{\pkg}[1]{{\normalfont\fontseries{b}\selectfont #1}}
\let\proglang=\textsf
\let\code=\texttt

\newlength{\cslhangindent}
\setlength{\cslhangindent}{1.5em}
\newenvironment{CSLReferences}%
  {}%
  {\par}
 

\usepackage{color}
\usepackage{fancyvrb}
\newcommand{\VerbBar}{|}
\newcommand{\VERB}{\Verb[commandchars=\\\{\}]}
\DefineVerbatimEnvironment{Highlighting}{Verbatim}{commandchars=\\\{\}}
% Add ',fontsize=\small' for more characters per line
\usepackage{framed}
\definecolor{shadecolor}{RGB}{248,248,248}
\newenvironment{Shaded}{\begin{snugshade}}{\end{snugshade}}
\newcommand{\AlertTok}[1]{\textcolor[rgb]{0.94,0.16,0.16}{#1}}
\newcommand{\AnnotationTok}[1]{\textcolor[rgb]{0.56,0.35,0.01}{\textbf{\textit{#1}}}}
\newcommand{\AttributeTok}[1]{\textcolor[rgb]{0.77,0.63,0.00}{#1}}
\newcommand{\BaseNTok}[1]{\textcolor[rgb]{0.00,0.00,0.81}{#1}}
\newcommand{\BuiltInTok}[1]{#1}
\newcommand{\CharTok}[1]{\textcolor[rgb]{0.31,0.60,0.02}{#1}}
\newcommand{\CommentTok}[1]{\textcolor[rgb]{0.56,0.35,0.01}{\textit{#1}}}
\newcommand{\CommentVarTok}[1]{\textcolor[rgb]{0.56,0.35,0.01}{\textbf{\textit{#1}}}}
\newcommand{\ConstantTok}[1]{\textcolor[rgb]{0.00,0.00,0.00}{#1}}
\newcommand{\ControlFlowTok}[1]{\textcolor[rgb]{0.13,0.29,0.53}{\textbf{#1}}}
\newcommand{\DataTypeTok}[1]{\textcolor[rgb]{0.13,0.29,0.53}{#1}}
\newcommand{\DecValTok}[1]{\textcolor[rgb]{0.00,0.00,0.81}{#1}}
\newcommand{\DocumentationTok}[1]{\textcolor[rgb]{0.56,0.35,0.01}{\textbf{\textit{#1}}}}
\newcommand{\ErrorTok}[1]{\textcolor[rgb]{0.64,0.00,0.00}{\textbf{#1}}}
\newcommand{\ExtensionTok}[1]{#1}
\newcommand{\FloatTok}[1]{\textcolor[rgb]{0.00,0.00,0.81}{#1}}
\newcommand{\FunctionTok}[1]{\textcolor[rgb]{0.00,0.00,0.00}{#1}}
\newcommand{\ImportTok}[1]{#1}
\newcommand{\InformationTok}[1]{\textcolor[rgb]{0.56,0.35,0.01}{\textbf{\textit{#1}}}}
\newcommand{\KeywordTok}[1]{\textcolor[rgb]{0.13,0.29,0.53}{\textbf{#1}}}
\newcommand{\NormalTok}[1]{#1}
\newcommand{\OperatorTok}[1]{\textcolor[rgb]{0.81,0.36,0.00}{\textbf{#1}}}
\newcommand{\OtherTok}[1]{\textcolor[rgb]{0.56,0.35,0.01}{#1}}
\newcommand{\PreprocessorTok}[1]{\textcolor[rgb]{0.56,0.35,0.01}{\textit{#1}}}
\newcommand{\RegionMarkerTok}[1]{#1}
\newcommand{\SpecialCharTok}[1]{\textcolor[rgb]{0.00,0.00,0.00}{#1}}
\newcommand{\SpecialStringTok}[1]{\textcolor[rgb]{0.31,0.60,0.02}{#1}}
\newcommand{\StringTok}[1]{\textcolor[rgb]{0.31,0.60,0.02}{#1}}
\newcommand{\VariableTok}[1]{\textcolor[rgb]{0.00,0.00,0.00}{#1}}
\newcommand{\VerbatimStringTok}[1]{\textcolor[rgb]{0.31,0.60,0.02}{#1}}
\newcommand{\WarningTok}[1]{\textcolor[rgb]{0.56,0.35,0.01}{\textbf{\textit{#1}}}}

\newcommand{\bcenter}{\begin{center}}
\newcommand{\ecenter}{\end{center}}

\newcommand{\bapendices}{\begin{apendicesenv}}
\newcommand{\eapendices}{\end{apendicesenv}}

\newcommand{\banexos}{\begin{anexosenv}}
\newcommand{\eanexos}{\end{anexosenv}}

% ---
% Filtering and Mapping Bibliographies
% ---
\DeclareSourcemap{
	\maps[datatype=bibtex]{
		% remove fields that are always useless
		\map{
			\step[fieldset=abstract, null]
			\step[fieldset=pagetotal, null]
			\step[fieldset= doi, null]
		}
%		 remove URLs for types that are primarily printed
		\map{
			\pernottype{software}
			\pernottype{online}
			\pernottype{report}
			\pernottype{techreport}
			\pernottype{standard}
			\pernottype{manual}
			\pernottype{misc}
			\step[fieldset=url, null]
			\step[fieldset=urldate, null]
		}
		\map{
			\pertype{inproceedings}
			% remove mostly redundant conference information
			\step[fieldset=venue, null]
			\step[fieldset=eventdate, null]
			\step[fieldset=eventtitle, null]
			% do not show ISBN for proceedings
			\step[fieldset=isbn, null]
			% Citavi bug
			\step[fieldset=volume, null]
		}
	}
}
% ---

% ---
% Informações de dados para CAPA e FOLHA DE ROSTO
% ---
% FIXME Substituir 'Nome completo do autor' pelo seu nome.
\autor{Fernando Anderson Pereira de Souza}
% FIXME Substituir 'Título do trabalho' pelo título da trabalho.
\titulo{Os efeitos da Conversão Religiosa entre Egressos de Comunidades Terapêuticas: Uma análise comparativa}
% FIXME Substituir 'Subtítulo (se houver)' pelo subtítulo da trabalho.
% Caso não tenha substítulo, comente a linha a seguir.
% FIXME Substituir 'XXXXXX' pelo nome do seu
% orientador.
\orientador{Prof.~Dr.~Cláudio Santiago Dias Júnior}
% FIXME Se for orientado por uma mulher, comente a linha acima e descomente a linha a seguir.
% \orientador[Orientadora]{Nome da orientadora, Dra.}
% FIXME Substituir 'XXXXXX' pelo nome do seu
% coorientador. Caso não tenha coorientador, comente a linha a seguir.
\coorientador{Prof.~Dr.~Glauber Loures de Assis}
% FIXME Se for coorientado por uma mulher, comente a linha acima e descomente a linha a seguir.
% \coorientador[Coorientadora]{XXXXXX, Dra.}
% FIXME Substituir '[ano]' pelo ano (ano) em que seu trabalho foi defendido.
\ano{2021}
% FIXME Substituir '[dia] de [mês] de [ano]' pela data em que ocorreu sua defesa.
\data{08 de Agosto de 2021}
% FIXME Substituir 'Local' pela cidade em que ocorreu sua defesa.
\local{Belo Horizonte}
\instituicaosigla{UFMG}
\instituicao{Universidade Federal de Minas Gerais}
% FIXME Substituir 'Dissertação/Tese' pelo tipo de trabalho (Tese, Dissertação).
\tipotrabalho{Dissertação}
% FIXME Substituir '[mestre/doutor] em XXXXXX' pela grau adequado.
\formacao{Mestre em Sociologia}
% FIXME Substituir '[mestrado/doutorado]' pelo nivel adequado.
\nivel{mestrado}
% FIXME Substituir 'Programa de Pós-Graduação em XXXXXX' pela curso adequado.
\programa{Programa de Pós-Graduação em Sociologia}
% FIXME Substituir 'Campus XXXXXX ou Centro de XXXXXX' pelo campus ou centro adequado.
\centro{FAFICH - FACULDADE DE CIÊNCIAS HUMANAS}
\preambulo
{%
\imprimirtipotrabalho~submetida~ao~\imprimirprograma~da~\imprimirinstituicao~para~a~obtenção~do~título~de~\imprimirformacao.
}
% ---

% ---
% Configurações de aparência do PDF final
% ---
% alterando o aspecto da cor azul
\definecolor{blue}{RGB}{41,5,195}
% informações do PDF
\makeatletter
\hypersetup{
     	%pagebackref=true,
		pdftitle={\@title},
		pdfauthor={\@author},
    	pdfsubject={\imprimirpreambulo},
	    pdfcreator={LaTeX with abnTeX2},
		pdfkeywords={ufsc, latex, abntex2},
		colorlinks=true,       		% false: boxed links; true: colored links
    	linkcolor=black,%blue,          	% color of internal links
    	citecolor=black,%blue,        		% color of links to bibliography
    	filecolor=black,%magenta,      		% color of file links
		urlcolor=black,%blue,
		bookmarksdepth=4
}
\makeatother
% ---

% ---
% compila a lista de abreviaturas e siglas e a lista de símbolos
% ---

% Declaração das siglas
%lista de siglas
\siglalista{CT}{Comunidades Terap{\^e}uticas}
\siglalista{CNAE}{Classifica{\c c}{\~a}o Nacional de Atividades Eco{\^o}micas}
\siglalista{CFESS}{Conselho Federal de Serviço Social}
\siglalista{CFP}{Conselho Federal de Psicologia}
\siglalista{RAPS}{Rede de Assist{\^e}ncia Psicossocial}
\siglalista{CAPS}{Centro de Assist{\^e}ncia Psicossocial}
\siglalista{CAPS AD}{Centro de Assist{\^e}ncia Psicossocial {\'A}lcool e Drogas}
\siglalista{SENAD}{Secretaria Nacional de Pol{\'i}tica sobre Drogas}
\siglalista{PRD}{Programa de Reduç{\~a}o de Danos}
\siglalista{SUPERA}{Sistema para Detec{\c c}{\~a}o do Uso Abusivo e Dependência de Subst{\^a}ncias Psicoativas: Encaminhamento, interven{\c c}{\~a}o Breve, Reinser{\c c}{\~a}o Social e Acompanhamento}
\siglalista{PEAD}{Plano Emergencial de Amplia{\c c}{\~a}o do Acesso ao Tratamento e {\`a} Preven{\c c}{\~a}o em {\'A}lcool e outras Drogas}
\siglalista{FEBRACT}{Federa{\c c}{\~a}o Brasileira de Comunidades Terap{\^e}uticas}
\siglalista{HIV}{Human Immunodeficiency Virus}
\siglalista{CONFENACT}{Confedera{\c c}{\~a}o Nacional de Comunidades Terap{\^e}uticas}
\siglalista{FETEB}{Federa{\c c}{\~a}o de Comunidades Terap{\^e}uticas Evang{\'e}licas do Brasil}
\siglalista{ANVISA}{Ag{\^e}ncia Nacional de Vigil{\^a}ncia Sanit{\'a}ria}
\siglalista{QCA}{Qualitative Comparative Analysis}
\siglalista{SUS}{Sistema {\'U}nico de Sa{\'u}de}
\siglalista{INUS}{Insufficiente, but Necessary}
\siglalista{SUIN}{Sufficiente, but Unnecessary}
\siglalista{MJ}{Minist{\'e}rio da Justi{\c c}a}
\siglalista{MDS}{Minist{\'e}rio do Desenvolvimento Social}
\siglalista{MNPCT}{Mecanismo Nacional de Preven{\c c}{\~a}o e Combate {\`a} Tortura}
\siglalista{PFDC}{Procuradoria Federal dos Direitos do Cidad{\~a}o}
\siglalista{PNAD}{Pol{\'i}ticas Nacionais de Drogas}
\siglalista{SENAPRED}{Secretaria Nacional de Cuidados e Preven{\c c}{\~a}o {\`a}s Drogas}
\siglalista{ECRIECT}{Efeitos da Convers{\~a}o Religiosa entre Egressos de CT}
\siglalista{A.A}{Alco{\'o}licos An{\^o}nimos}
\siglalista{CNPJ}{Cadastro Nacional de Pessoa Jur{\'i}dica}
\siglalista{IPEA}{Instituto de Pesquisa Econ{\^o}mica Aplicada}
\siglalista{RDC}{Regime Diferenciado de Contrata{\c c}{\~a}o}
\siglalista{OBID}{Observat{\'o}rio Brasileiro de Informa{\c c}{\~o}es sobre Droga}
% Declaração dos simbolos
%lista de simbolos
\simbololista{Reais}{$ \Re $}{N{\'u}meros reais}
\simbololista{Fuzzy}{$ \Sigma $}{Somat{\'o}rio}
\simbololista{efeito}{$ \mu $}{Mi}
\simbololista{total}{$ \subset $}{Subconjunto}
\simbololista{Semi}{$ \cap $}{Uni{\~a}o}
\simbololista{ou}{$ \cup $}{Intercess{\~a}o}
\simbololista{pertencimento}{$ \not\subset $}{N{\~a}o Pertencimento}
\simbololista{carga}{$ \alpha $}{Alpha de Crombach}

% compila a lista de abreviaturas e siglas e a lista de símbolos
\makenoidxglossaries

% ---

% ---
% compila o indice
% ---
\makeindex
% ---

% ----
% Início do documento
% ----
\begin{document}

% Seleciona o idioma do documento (conforme pacotes do babel)
%\selectlanguage{english}
\selectlanguage{brazil}

% Retira espaço extra obsoleto entre as frases.
\frenchspacing

% Espaçamento 1.5 entre linhas
\OnehalfSpacing

% Corrige justificação
%\sloppy

% ----------------------------------------------------------
% ELEMENTOS PRÉ-TEXTUAIS
% ----------------------------------------------------------
% \pretextual %a macro \pretextual é acionado automaticamente no início de \begin{document}
% ---
% Capa, folha de rosto, ficha bibliografica, errata, folha de apróvação
% Dedicatória, agradecimentos, epígrafe, resumos, listas
% ---
% ---
% Capa
% ---
\imprimircapa
% ---

% ---
% Folha de rosto
% (o * indica que haverá a ficha bibliográfica)
% ---
\imprimirfolhaderosto*
% ---

% ---
% Inserir a ficha bibliografica
% ---
% http://ficha.bu.ufsc.br/
\begin{fichacatalografica}
	\includepdf{Ficha_Catalografica.pdf}
\end{fichacatalografica}
% ---

% ---
% Inserir folha de aprovação
% ---
\begin{folhadeaprovacao}
	\OnehalfSpacing
	\centering
	\imprimirautor\\%
	\vspace*{10pt}		
	\textbf{\imprimirtitulo}%
	\ifnotempty{\imprimirsubtitulo}{:~\imprimirsubtitulo}\\%
	%		\vspace*{31.5pt}%3\baselineskip
	\vspace*{\baselineskip}
	%\begin{minipage}{\textwidth}
	O presente trabalho em nível de \imprimirnivel~foi avaliado e aprovado por banca examinadora composta pelos seguintes membros:\\
	%\end{minipage}%
	\vspace*{\baselineskip}
    Prof\textordfeminine. Cristina Maria de Castro, Dr\textordfeminine.\\
  Universidade Federal de Minas Gerais - UFMG\\
  \vspace*{\baselineskip}
    Prof\textordfeminine. Taniele Cristina Rui, Dr\textordfeminine.\\
  Universidade Estadual de Campinas - UNICAMP\\
  \vspace*{\baselineskip}
    Prof. Manoel Leonardo Wanderley Duarte Santos, Dr.\\
  Universidade Federal de Minas Gerais - UFMG\\
  \vspace*{\baselineskip}
    
	\vspace*{2\baselineskip}
	\begin{minipage}{\textwidth}
		Certificamos que esta é a \textbf{versão original e final} do trabalho de conclusão que foi julgado adequado para obtenção do título de \imprimirformacao.\\
	\end{minipage}
	%    \vspace{-0.7cm}
	\vspace*{\fill}
	\assinatura{\OnehalfSpacing Prof\textordfeminine. Dr\textordfeminine. Ana Marcela Ardila \\ Coordenação do Programa de Pós-Graduação}
	\vspace*{\fill}
	\assinatura{\OnehalfSpacing\imprimirorientador \\ \imprimirorientadorRotulo}
	%	\ifnotempty{\imprimircoorientador}{
	%	\assinatura{\imprimircoorientador \\ \imprimircoorientadorRotulo \\
	%		\imprimirinstituicao~--~\imprimirinstituicaosigla}
	%	}
	% \newpage
	\vspace*{\fill}
	\imprimirlocal, \imprimirano.
\end{folhadeaprovacao}
% ---

% ---
% Dedicatória
% ---
\begin{dedicatoria}
	\vspace*{\fill}
	\noindent
	\begin{adjustwidth*}{}{5.5cm} 
		\raggedleft       
		À Maria Aline, minha mãe.
	\end{adjustwidth*}
\end{dedicatoria}
% ---

% ---
% Agradecimentos
% ---
\begin{agradecimentos}
	Essa dissertação representa, mais do que um trabalho de conclusão, o resultado de um sonho. Em ordem de conseguir realizar o que almejei tive que mudar de estado, morar só pela primeira vez na minha vida e passar por bons bocados. Durante esses dois anos vivi mais experiências do que imaginei e cresci bastante, tanto como pessoa quanto profissionalmente, de forma que não sou mais o mesmo do que quanto entrei. Fico feliz pelo que vivi, pelo que não vivi e pelo futuro que se desvela e sou extremamente grato a todas as pessoas que me ajudaram até aqui.

 Agradeço imensamente aos professores que conheci na UFC, em especial Domingos e Jakson que acreditaram em mim e me proporcionaram toda ajuda possível para que eu pudesse me manter durante a graduação e dar meus primeiros passos no mestrado. Vocês foram os pais que eu nunca tive e minha gratidão com vocês será eterna. Obrigado por tudo. Agradeço também à Kleyton Rattes que corrigiu o projeto que enviei ao Programa no período de seleção. Suas pontuações precisas me ajudaram muito e tornaram meu texto muito mais robusto.

 Agradeço também aos professores que conheci durante o Ensino Médio, em especial Edna e Antoine que despertaram em mim o pensamento crítico e me fizeram tomar gosto pelas humanidades. Obrigado por terem me acompanhado até aqui, pela amizade e pela disposição.

 No mesmo ensejo agradeço à minha família por, mesmo não entendendo direito o que eu faço, terem me dado o suporte que podiam. Obrigado pelas orações, pelas ligações, pelas mensagens de boa noite e por sempre estarem de portas abertas para me receber.

 Também não poderia deixar de fora dos agradecimentos meus amigos que fizeram das tripas coração para me ajudar nesse processo. Agradeço a Mariana, Márcia, Isabel, Janaína, Eduardo, Trajano, Mara Mônica, Bia, Ângela e todos os outros por sempre estarem comigo e criarem uma verdadeira campanha massiva de compartilhamento da minha pesquisa. Vocês conseguiram alegrar até mesmo os dias mais ruins e sou extremamente grato por isso.

 Agradeço a Eric, meu melhor amigo e meu amor, por ter me acompanhado desde o início (literalmente). Obrigado por ter me consolado quando precisei, por ter se animado comigo, por ter lidado como um santo com a distância que esse processo impôs em nosso relacionamento e por sempre estar ao meu lado, independente do que acontecesse. Essa não é a primeira conquista que comemoro ao seu lado, mas é uma das mais especiais. Te amo.

 Agradeço também aos professores que conheci na UFMG, em especial Cláudio e Glauber, que me acolheram como orientando, foram extremamente pacientes comigo e me deram insights valiosos para a construção dessa pesquisa, e a Marden e Jorge que viram potencial em mim e me ajudaram a aumentar ainda mais minha bagagem de conhecimentos quantitativos ao me incluírem em seus projetos. Foi um prazer trabalhar com todos vocês e espero que possamos manter contato no futuro.

 Também sou grato a todas as pessoas que se disponibilizaram a responder a minha pesquisa. Não sei se vocês chegaram a ler este estudo algum dia, mas saibam que sem vocês essa pesquisa não seria possível.

 Agradeço, finalmente, a CAPES pelo financiamento.
\end{agradecimentos}
% ---

% ---
% Epígrafe
% ---
\begin{epigrafe}
	\vspace*{\fill}
	\begin{flushright}
		\textit{Progress\\
Pushing through the mould\\
Tracing with my fingers, waking up\\
Wanting growth\\
(SOPHIE, Whole New World)}
	\end{flushright}
\end{epigrafe}
% ---

% ---
% RESUMOS
% ---

% resumo em português
\setlength{\absparsep}{18pt} % ajusta o espaçamento dos parágrafos do resumo
\begin{resumo}
	\SingleSpacing
  As Comunidades Terapêuticas, definidas enquanto centros de terapia focados na recuperação de indivíduos que escolhem por livre e espontânea vontade receberem tratamento para males de ordem mental baseado em vivência comunitária e na transformação de hábitos e comportamentos, ganham cada vez mais espaço no campo das políticas públicas nacionais de cuidado a pessoas que fazem uso abusivo de drogas. Tais instituições costumam empregar no tratamento, entre outras coisas, elementos religiosos e/ou espirituais que quase sempre são atrelados a incentivos diretos e indiretos à Conversão Religiosa dos internos. Pouco se sabe, no entanto, sobre o efeito destas práticas na recuperação e manutenção da abstêmia de egressos desse sistema. Este estudo tentou mensurar tais efeitos e entender qual papel a Conversão empregou nos casos analisados. Não apenas isso mas também traçar uma gênese dessas instituições no Brasil, tendo como ponto de partida seus encontros e desencontros com instâncias público governamentais e privadas. Parte-se do pressuposto de que, baseado nos princípios da multicausalidade, sua presença não seria necessária ou suficiente para produzir os resultados por ela almejados e que outras condições, como Redes Sociais/de Apoio, seriam mais efetivas. Os dados para testar as hipóteses aqui formuladas foram obtidos através de survey on-line aplicado a nível nacional com egressos de Comunidades Terapêuticas e analisados no software R. A técnica empregada para análise foi a Qualitative Comparative Analysis (QCA). Fórmulas causais obtidas revelaram que, entre os pesquisados, a Recuperação era causada pela Ausência de Conversão e que a Abstêmia acontecia apenas na Presença de Redes de Apoio e Ausência de Conversão. 
  
  \textbf{Palavras-chave}: 
    Comunidades Terapêuticas.
    Conversão Religiosa.
    Políticas Públicas.
  \end{resumo}
% resumo em inglês
\begin{resumo}[Abstract]
	\SingleSpacing
	\begin{otherlanguage*}{english}
		The Therapeutic Communities are gaining more and more space in the field of national public policies for the care of people who abuse drugs. Such institutions usually employ in the treatment, among other things, religious/spiritual elements that are almost always linked to direct and indirect incentives to the Religious Conversion of the inmates. Little is known, however, about the effect of these practices on the recovery and maintenance of abstemiousness among egresses. This study tried to measure such effects and understand what role Conversion played in the analyzed cases. It was assumed that it would not produce effects and that its presence would not be necessary or sufficient to produce the desired results and that other conditions, such as sociability, would be more effective. The data to test the hypotheses formulated here were obtained through an online survey applied at national level with egresses from Therapeutic Communities and analyzed using the R software. The technique used for analysis was the Qualitative Comparative Analysis (QCA). Causal formulas obtained in the analyzes revealed that the Recovery was caused by the Absence of Conversion and that Abstemia only happened in the Presence of Support Networks and the Absence of Conversion. In addition, an attempt was made to trace the genesis of these institutions in Brazil, having as a starting point their encounters and disagreements with instances of the public, governmental and private spheres.
		
		\textbf{Keywords}:
	      Therapeutic Communities.
        Religious Conversion.
        Public Policy.
    	\end{otherlanguage*}
\end{resumo}
%% resumo em francês 
%\begin{resumo}[Résumé]
% \begin{otherlanguage*}{french}
%    Il s'agit d'un résumé en français.
% 
%   \textbf{Mots-clés}: latex. abntex. publication de textes.
% \end{otherlanguage*}
%\end{resumo}
%
%% resumo em espanhol
%\begin{resumo}[Resumen]
% \begin{otherlanguage*}{spanish}
%   Este es el resumen en español.
%  
%   \textbf{Palabras clave}: latex. abntex. publicación de textos.
% \end{otherlanguage*}
%\end{resumo}
%% ---

{%hidelinks
	\hypersetup{hidelinks}
	% ---
	% inserir lista de ilustrações
	% ---
	\pdfbookmark[0]{\listfigurename}{lof}
	\listoffigures
	\cleardoublepage
	% ---
	
	
	% ---
	% inserir lista de tabelas
	% ---
	\pdfbookmark[0]{\listtablename}{lot}
	\listoftables
	\cleardoublepage
	% ---
	
	% ---
	% inserir lista de abreviaturas e siglas (devem ser declarados no preambulo)
	% ---
	\imprimirlistadesiglas
	% ---
	
	% ---
	% inserir lista de símbolos (devem ser declarados no preambulo)
	% ---
	\imprimirlistadesimbolos
	% ---
	
	% ---
	% inserir o sumario
	% ---
	\pdfbookmark[0]{\contentsname}{toc}
	\tableofcontents*
	\cleardoublepage
	
}%hidelinks
% ---

% ---

% ----------------------------------------------------------
% ELEMENTOS TEXTUAIS
% ----------------------------------------------------------
\textual

\hypertarget{intro}{%
\chapter{Introdução}\label{intro}}

Entre maio e julho de 2020 um dos assuntos mais comentados do Twitter\footnote{Disponível em: \url{https://bityli.com/rm13V}} foi uma coleção de revistinhas infantis evangélicas chamada ``Dudão''. Lançada pela Editora Vida entre os anos de 1991 e 2000, e com um corpo editorial feito majoritariamente por protestantes, a HQ segue a vida de Dudão, um menino cristão, e seus amigos Binho, Paçoca, Rebeca, Zuca e Pita.

Boa parte das histórias se resumem à Dudão ajudando seus amigos, seja com conselhos religiosos ou ações diretas, a saírem das enrascadas que eles se meteram. Dentre as várias histórias disponíveis uma me chamou a atenção. Em \emph{``Elas estão por aí''} Dudão encontra Binho, seu amigo, chorando desolado em um campo após ter tido suas chinelas roubadas por um garoto de rua. Dudão e Binho passam boa parte da história tentando localizar o dito cujo até que o encontram embaixo de uma ponte. Quando eles tentaram pegar os calçados de volta a criança pediu que não batessem nele e justificou as suas ações falando que ``nunca teve uma chinela''. Sensibilizados, Binho e Dudão acabam indo conversar com o pastor da igreja deles para saber se o garoto podia ficar no orfanato regido pela instituição. A imagem abaixo é extraída diretamente da revistinha e descreve o evento em questão:
\begin{figure}[H]

{\centering \includegraphics[width=0.6\linewidth]{images/dudão35} 

}

\caption{ (Em ordem) Binho, Menino de Rua e Dudão conversando com o Pastor}\label{fig:imagem00}
\end{figure}
\bcenter

Fonte: Dudão, 1995
\ecenter

O pastor pergunta a criança se ``ela quer ser de Jesus'' e fala que ``com Jesus nunca lhe faltará nada'', o que leva a criança a pensar em um belo frango assado e reiterar a sua escolha. Apesar de caricata a história é exemplo de uma prática muito comum no meio cristão brasileiro: embrulhar teologia nas boas ações.

É frequente, principalmente no meio protestante, que as ações de cunho social empregadas sejam também formas de evangelização e anúncio da boa nova\footnote{ Mais informações sobre essa discussão podem ser encontradas em Rezende \& Oliveira\autocite*{rezende_as_2014}}. O \emph{modus operandi} dessas instituições se baseia em, na maioria das vezes, fornecer subsídios morais e espirituais junto com a ajuda material, sendo estes quase sempre ancorados no processo de ``aceitar Jesus'', em outras palavras, Conversão Religiosa. Dentre as que se utilizam dessas técnicas uma vem ganhando cada vez mais destaque, chegando até mesmo a ser pauta nas eleições de 2020: As \emph{Comunidades Terapêuticas.}

Essas instituições costumam ofertar acolhimento e cuidado para pessoas que mostram sinais de uso abusivo de drogas\footnote{Neste trabalho opto pelo termo ``uso abusivo de drogas'' ao invés de ``dependência química''. Isso acontece, principalmente, pela estigmatização que este segundo termo carrega consigo \autocite{rosa2010uso}. Não é a pretensão deste estudo debater uso de drogas por uma ótica moralista ou patológica, muito menos condenar ninguem. Aqui pretende-se instaurar um diálogo aberto, horizontal e longe de esteriótipos.} por longos períodos por meio de abstêmia e dentro delas é comum que, junto com o cuidado físico e mental, também se ofereça um espiritual que acontece na forma de missas, cultos, leituras da bíblia e incentivo direto/indireto a Conversão.

Tais práticas imbuíram essas instituições de diversas controvérsias, de forma que vários Conselhos e instituições se manifestaram contra elas, sendo essas críticas adereçadas com mais força quando as CT's começaram a receber financimaneto público. Apesar de parecer uma questão relativamente simples, ela está envolta em uma trama mais densa e se demonstra bastante peculiar dado os atores que se movem em sua defesa.

Apesar de tudo isto, algumas perguntas devem ser feitas: Alimentar essas práticas religiosas no contexto do tratamento tem algum efeito prático no resultado final? No meio de tantas argumentações pouco ou quase nada se ouve falar sobre a eficácia dos próprios métodos empregados por estas instituições, principalmente no que corresponde as pessoas que já passaram por esse tipo de tratamento. Como essas pessoas se encontram hoje? Elas continuam abstêmicas? Se sim, a que elas atribuem esse processo? A Conversão que foi oferecida no contexto da Comunidade Terapêutica teve alguma influência no resultado final? E o mais importante: Vale a pena investir nesse tipo de método?

Estas e outras indagações moveram a construção desta pesquisa. Nas próximas páginas será possível encontrar debates mais detalhados sobre o que são essas instituições, o que é Conversão e se, de fato, isso teve ou não efeito entre os que passaram por este tipo de tratamento.

\hypertarget{objetivos}{%
\chapter{Objetivos}\label{objetivos}}

\hypertarget{objetivo-geral}{%
\section{Objetivo Geral}\label{objetivo-geral}}

Partindo de tudo o que foi discutido acima, é possível desenhar o que viria a ser o objetivo desta pesquisa. Qual seja, \emph{entender se o processo de Conversão Religiosa e/ou contato com a Religiosidade/Espiritualidade durante o tratamento influenciou na Recuperação e Manutenção da abstêmia de indivíduos que já deram entrada em Comunidades Terapêuticas Religiosas}. A pesquisa será quantitativa e a nível nacional, podendo participar homens e mulheres que tenham dado entrada ou concluído o tratamento para uso abusivo de drogas em uma Comunidade Terapêutica.

\hypertarget{objetivos-especuxedficos}{%
\section{Objetivos Específicos}\label{objetivos-especuxedficos}}
\begin{enumerate}
\def\labelenumi{\arabic{enumi}.}
\tightlist
\item
  Entender como os egressos de Comunidades Terapêuticas avaliam a importância da Conversão Religiosa dentro do contexto do Programa Terapêutico e fora dele.
\item
  Avaliar se, nos dados coletados, existe alguma relação sólida entre a manutenção da abstêmia e ser convertido ao credo da CT.
\item
  Avaliar se, nos dados coletados, existe alguma relação sólida entre a recuperação do uso abusivo de drogas e ser convertido ao credo da CT.
\item
  Observar, através de levantamentos bibliográficos e documentais, as controvérsias percebidas nos discursos sobre Comunidades Terapêuticas emitidas por atores nos níveis micro e macrossocial.
\item
  Observar através de levantamentos bibliográficos e documentais quais são atores mobilizados nas discussões.
\item
  Entender, no contexto dos dados coletados, qual seria a implicância da introdução de espiritualidade e religiosidade no itinerário do tratamento.
\end{enumerate}
\newpage

\hypertarget{referencial-teuxf3rico}{%
\chapter{Referencial Teórico}\label{referencial-teuxf3rico}}

\hypertarget{a-questuxe3o-da-conversuxe3o-religiosa-na-sociologia}{%
\section{A Questão da Conversão Religiosa na Sociologia}\label{a-questuxe3o-da-conversuxe3o-religiosa-na-sociologia}}

Seja por não haver uma definição única, seja pela falta de consenso na área sobre o tema, a Conversão Religiosa se tornou com o passar dos anos uma das temáticas mais controversas que a Sociologia da Religião se debruça e tenta rastrear. Foi durante a década de 60 que ela se tornou um objeto de estudo propriamente sociológico, graças a alta demanda de pesquisas sobre o crescimento quase exponencial de novos movimentos religiosos na América do Norte \autocite[167]{snow_sociology_1984}. As primeiras discussões no campo sociológico, assim como uma primeira tentativa de criar um modelo explicativo para a conversão, remetem aos estudos feitos em 1965 por Lofland e Stark \autocite{stark_acts_2000-1}. Seus resultados se encontram no artigo \emph{``Becoming a world-saver: a theory of conversion to a deviant perspective''}.

Desse período em diante vários estudos foram conduzidos, seja para confirmar os achados de Lofland e Stark, seja para apontar erros ou sugerir novas rotas de análise. A partir da observação do conjunto de trabalhos e proposições produzidas, alguns autores defendem a existência de dois grandes paradigmas que conduziam as pesquisas até então feitas: um que focava em \emph{elementos estruturais e exteriores} ao indivíduo e um segundo que abordava a conversão a partir do \emph{próprio sujeito e sua trajetória}. Em outras palavras, o nível de agência é o que caracterizaria e daria o tom a cada uma das formas pelas quais a Conversão Religiosa foi estudada dentro da Sociologia até o presente momento. Apesar desta discussão extensa sobre as formas pelas quais um sujeito chega a se converter ser extremamente importante uma outra pergunta deve ainda ser feita para que um total entendimento sobre a discussão seja, de fato, atingido: \emph{O que é uma conversão?}

Apesar de simples esta segunda pergunta gera muitos conflitos na área e até o presente momento não possui uma resposta definitiva: Mudança radical, Trajetória, Transformação do Discurso ou simplesmente Mudança de Religião são algumas das várias opções possíveis oferecidas pela literatura para definir este fenômeno. Isso acontece pois são diversas as variáveis e as searas que se entrecruzam no processo de explicar a conversão. Cada um dos paradigmas anteriormente citados definia este fenômeno à sua maneira, cabendo ao pesquisador optar pela definição que melhor se encaixasse em seu \emph{framework}.

Nos próximos capítulos cada uma das problemáticas acima citadas serão adereçadas. Primeiramente serão apresentadas cada uma das abordagens explicativas da Conversão Religiosa, suas principais características, suas críticas e como a temática vem sendo trabalhada no Brasil. Logo em seguida se discutirá sobre o que, de fato, esse fenômeno é. Acredito que possa parecer estranho ao leitor inverter a ordem da discussão, mas pode ser mais fácil enxergar o que um elemento é a partir de seus modos de ser. Ao final será feita uma discussão sobre estratégias de mensuração deste conceito, avaliando as possibilidades e limitações.

\hypertarget{o-paradigma-tradicional}{%
\subsection{O Paradigma Tradicional}\label{o-paradigma-tradicional}}

Como colocado anteriormente podem-se distinguir dois grandes paradigmas para explicar Conversões Religiosas. O primeiro a ser criado é o chamado tradicional. Ele é focado principalmente em elementos exteriores ao indivíduo e na sua falta de agência sobre o processo e é melhor representado na Conversão do apóstolo Paulo narrada no livro bíblico de Atos dos Apóstolos\footnote{Cf. At 9:01-22.}. Nele as Conversões se configuravam como momentos de extrema comoção e epifania catalisadas pela suposta sensação do agir de um poder ou divindade maior e alheio ao indivíduo. Esse fenômeno também possui a capacidade de transformar radicalmente a vida do sujeito, de forma a determinar divisores de água em sua biografia \autocite[165]{richardson_active_1985}. Esse tipo de Conversão é a que mais se aproxima da noção do senso comum sobre o assunto, dado que promove o renascimento pessoal e a ``morte'' de um suposto ``antigo eu'' \autocite[1]{kilbourne_paradigm_1989}.

Ao ser tratado pela Sociologia, no entanto, esse modelo passa a ganhar novos contornos e significações. Inicialmente ocorre um distanciamento do caráter místico/teológico do fenômeno e são adicionados dois elementos que tinham um enorme papel no entendimento da Conversão Religiosa enquanto um fenômeno social: A \emph{Socialização} e a \emph{Coerção}.

Autores como Toch \autocite*{toch_social_2013} que tentam explicar o processo de Conversão através da Socialização tendem a assumir que é ela quem delimita o quão predisposto um indivíduo está ou não a se converter a um determinado credo. Suas experiências prévias, especialmente as vivenciadas na infância, são o que conduzem todo o processo e o guiam a religiões e cultos com características similares as da primeira socialização \autocite[08]{kilbourne_paradigm_1989}. A Coerção, por outro lado, se baseia no uso de técnicas de restrição e isolamento do sujeito em ordem de torná-lo um convertido, esse tipo costuma ser mais comum entre seitas. A soma de elementos psicológicos como a atração de indivíduos com maior grau de vulnerabilidade emocional e da indução de \emph{breakdowns} sociais e emocionais, combinada muitas vezes com o afastamento do indivíduo de suas redes sociais, lentamente o induzem ao total estado de submissão e controle do culto, não é à toa que esse modelo também é chamado de \emph{Lavagem Cerebral} (brainwashing). É exatamente esta operação que a Sociologia Norte Americana acreditava que acontecia em cultos como o Hare Khrishna, Synanon \footnote{Synanon, que será tratada no segundo capítulo, é considerada a precursora do modelo de Comunidades Terapêuticas que será aqui pesquisado.} e algumas Igrejas Pentecostais, que na época eram consideradas seitas \autocite[283]{robbins_deprogramming_1982}.

Esse modelo, apesar de ter sido uma das primeiras tentativas de se entender a Conversão Religiosa, é bastante criticado e pouco usado \autocite[346]{gooren_reassessing_2007-1}. Uma das primeiras críticas feitas é a exclusão total do indivíduo no processo de Conversão, cabendo a ele apenas repetir padrões ensinados na infância ou ser manipulado. Não levar em consideração no modelo o próprio sujeito e a sua vontade, que ganha cada vez mais força para fazer suas opções e traçar suas rotas a partir do advento da modernidade, é não estar em consonância com a realidade, sem falar que não há provas empíricas de que esses mecanismos realmente ocorram nos processos de Conversão \autocite[346]{gooren_reassessing_2007-1}.

\hypertarget{o-novo-paradigma-modelos-de-conversuxe3o-ativa}{%
\subsection{O Novo Paradigma: Modelos de Conversão Ativa}\label{o-novo-paradigma-modelos-de-conversuxe3o-ativa}}

Apesar do paradigma tradicional ter sido um dos primeiros a ser pensado a sua repercussão foi baixa. Um segundo, pensado a partir das limitações deste, foi quem conseguiu mais espaço. Intitulado por alguns autores como ``Novo Paradigma'', ele tinha como principal característica o foco na agência do indivíduo no processo de Conversão. As escolhas e preferências pessoais, significados, sentidos e negociações são temas chaves \autocite[02]{kilbourne_paradigm_1989}. Um outro diferencial desse modelo é seu distanciamento total de elementos teológicos, o que facilitou sua acolhida dentro da Sociologia. Podem-se destacar nessa classe os modelos \emph{intelectual}, de \emph{deriva social} e \emph{experimental}.

No modelo intelectual, como o próprio nome sugere, a conversão se dá através do contato do próprio indivíduo com conteúdo religioso/espiritual, sem a necessidade da comunidade, interação social ou de qualquer outro \emph{input} \autocite[376]{lofland_conversion_1981}. Livros, vídeos, televisão etc. oferecem subsídios para que, sozinho, o indivíduo possa moldar suas próprias experiências \autocite[377]{lofland_conversion_1981}. Apesar desse modelo ser bastante incomum na época em que foi criado é possível pensar o quanto ele se faz presente na atualidade com os avanços e facilidades que a era digital trouxe. Não é à toa que alguns pesquisadores chamam a internet de \emph{``dataline do paraíso''} \autocite[86]{george_religion_2006}. Por meio dela é possível ter acesso a cultos, celebrações, eventos e doutrinas para que, por conta própria, o indivíduo possa, no seu ritmo, digerir a doutrina e a ela se ajustar.

O modelo experimental\footnote{Do original, \emph{seeker}} é o mais popular dentre os modelos ativos. Nele o sujeito está aberto a experimentação e faz seu próprio caminho rumo a Conversão. Diferente do modelo passivo aonde o contato com o divino era crucial, definitivo e transformador aqui ele é apenas o ponto de partida. As mudanças, quando ocorrem, são graduais e no ritmo do indivíduo, que é livre para dosar o quanto aceita de forma plena e o quanto adapta a sua própria realidade \autocites[06]{kilbourne_paradigm_1989}[346]{gooren_reassessing_2007-1}. Admite-se também neste modelo a existências de múltiplas conversões, dada a variedade de experiências que se permite viver e da construção de Carreiras de Conversões \autocite[172]{richardson_active_1985}.

Fora da Sociologia Norte Americana também é possível encontrar modelos bastante similares. Um dos exemplos, mesmo que não seja relacionado diretamente a uma categoria de Conversão, é a do Peregrino de Hervieu-Leger \autocite*{hervieu-leger_peregrino_2015}. Para ela a modernidade atuaria sobre a religião de duas formas. A primeira seria na dissolução da chamada ``civilização paroquial'' e a segunda na flexibilização do ser religioso, instaurando assim uma condição de \emph{``peregrino''}.

Herdeira direta do catolicismo, tal civilização paroquial se manifesta na figura do praticante ou convertido, que é entendido enquanto indivíduo que atende fielmente as obrigações impostas pela igreja e articula sua vida em torno da comunidade e das expressões diárias de fé \autocite[84]{hervieu-leger_peregrino_2015}. A diminuição da influência da igreja sobre a sociedade e o advento da modernidade, no entanto, fazem com que essa civilização perca força e uma nova figura se estabeleça: o \emph{peregrino}. Diferente do praticante que se define pela sua vivência religiosa estática, o peregrino age a partir da bricolagem, ou ajustes, entre crença e experiência, cabendo a ele interpretar suas experiências de forma a moldar uma trajetória \autocite[89]{hervieu-leger_peregrino_2015}. Outro diferencial do peregrino é sua relação com a comunidade, que atua como uma espécie de ``apoio'' ao invés de determinador das suas ações \autocite[86]{hervieu-leger_peregrino_2015}.

Por fim, o modelo de Deriva Social\footnote{Do original, \emph{social drift}.}, também conhecido como modelo de Long e Hadden \autocite*{long_religious_1983}, é bastante similar ao modelo experimental, a principal diferença entre eles é a inserção de uma nova variável no processo de construção identitária: os \emph{fatores situacionais}. Stress, situações inesperadas tendem a colocar o indivíduo em estado de tristeza/dor e outros fatores se tornam gatilhos para que se busque ajuda na religião e a partir desse processo novas identidades e visões de mundo são construídas, guiando assim o processo de Conversão \autocite[07]{kilbourne_paradigm_1989}.

Apesar de ser um modelo que inclui o indivíduo ele falha ao desconsiderar a estrutura ou os chamados ``elementos passivos'' que coexistem com os ativos no processo de conversão. Não integrar agência individual e estrutura/fatores institucionais torna grande parte das premissas elaboradas incompletas. A linha que divide a Conversão enquanto ``processo'' ou ``evento'' não é rígida, e uma abordagem que leve ambos os paradigmas em consideração é a melhor para lidar com este fenômeno (\textcite{flinn_conversion_1999},55). Em outras palavras, a aceitação de uma individualização na vivência da Conversão não deve excluir completamente processos nos quais o indivíduo se transforma radicalmente e se entrega de forma total a instituição e seus ideais, mesmo se isso não ocorrer aos moldes dos antigos modelos de adesão \autocite[05]{rosas_conversao_2015}.

\hypertarget{modelos-mistos-integrando-estrutura-e-aguxeancia}{%
\subsection{Modelos Mistos: Integrando Estrutura e Agência}\label{modelos-mistos-integrando-estrutura-e-aguxeancia}}

Feitas as devidas explanações e críticas aos modelos exclusivistas (tradicional e ativo) resta abordar modelos que levem em consideração elementos de ambos os paradigmas e tentam superar tanto o determinismo quanto o voluntarismo presente nas versões anteriores. Um dos primeiros a ser destacado é o da Escolha Racional, no qual a escolha por se converter/filiar ou não a alguma religião/culto seria guiada pela racionalidade e teria como principal fator a preservação de um capital religioso acumulado com o passar dos anos \autocite[119]{stark_acts_2000-1}. Partindo desta noção, Stark e Finke criam um modelo para a Conversão baseado no nível de preservação do Capital Religioso e Cultural de um indivíduo.

Processos de Conversão seriam caracterizados por um trânsito \textbf{entre} tradições religiosas, ou seja, não existiria preservação de capital dado que ele seria totalmente remodelado seguindo os moldes da nova tradição ao qual o sujeito se converteu \autocite[114]{stark_acts_2000-1}. Um segundo processo, no entanto, também é perceptível e até mais comum do que este primeiro: o de \emph{Refiliação}. Tal processo se configura como trânsito \textbf{dentro de} uma mesma tradição religiosa \autocite[114]{stark_acts_2000-1}. Ou seja, existiria alguma preservação de capital dado que não se opera uma mudança cultural brusca. Nesses casos ocorre apenas a mudança de denominação dentro do escopo de uma mesma tradição. Processos de refiliação acabam por ser os mais frequentes e, ao executar estes processos, os indivíduos tendem a procurar opções que melhor se encaixem em seu atual arcabouço religioso, em ordem de otimizar sua manutenção \autocite[123]{stark_acts_2000-1}.

Ao avaliar ambas as noções apresentadas por este modelo, se torna perceptível sua ligação com modelos de socialização, mas diferente deste existe a possibilidade de romper com o que já se tem construído a partir de um processo (literal) de Conversão. O grande diferencial, no entanto, está no conceito de \emph{Capital Religioso}. Criado com base na noção de ``Capital Social'' de Bourdieu, ele é definido enquanto ``grau de domínio e apego a uma cultura religiosa particular'' \autocite[120]{stark_acts_2000-1}. Em outras palavras, é o nível de investimento emocional e cultural que o molda. Os indivíduos, em condições normais, tendem a preservá-lo ao optar por não fazer nenhuma das duas transições possíveis \autocite[121]{stark_acts_2000-1}. É importante destacar a ênfase na expressão ``condições normais'' pois, como debatido por outros modelos, situações extraordinárias como morte, separação, migração, casamentos, angústias e tristezas podem conduzir o indivíduo ao questionamento deste arcabouço que já possui e tentar encontrar um novo através do processo de busca e experimentação.

É possível, ainda, estabelecer algumas conexões entre as noções de peregrino e processos de refiliação. Todos compartilham de um certo grau de autonomia do indivíduo em ser o próprio ator de sua trajetória religiosa e por optar como pretende viver sua fé. Cada vez menos se dá foco em identidades mais institucionalizadas e começa a se admitir, por exemplo, a existência de ``crentes sem igreja'', pessoas que vivem a fé do seu jeito e até mesmo um crer sem pertencer \autocites[135]{davie_believing_1990,willaime_sociologia_2012-1}.

Um segundo modelo é o criado por Lofland e Stark, citado no começo deste capítulo. Ao estudar as motivações para conversão dentro de um culto chamado \emph{``Lord of the Second Advent''} os autores chegam ao que seriam 7 passos necessários para uma pessoa se tornar convertida, a se saber:
\begin{quote}
\begin{enumerate}
\def\labelenumi{\arabic{enumi}.}
\tightlist
\item
  Experience enduring, acutely felt tensions
\item
  Within a religious problem-solving perspective,
\item
  Which leads him to define himself as a religious seeker;
\item
  Encountering the Divine Precepts\footnote{Chamado ao longo do texto de D.P, é o conjunto de preceitos do culto estudado pelos autores \autocite[863]{lofland_becoming_1965-1}.}at a turning point in his life,
\item
  Wherein an affective bond is formed (or pre-exists) with one or more converts;
\item
  Where extra-cult attachments are absent or neutralized
\item
  And, where, if he is to become a deployable agent, he is exposed to intensive interaction.
  \autocite[874]{lofland_becoming_1965-1}
\end{enumerate}
\end{quote}
Percebe-se neste modelo a presença de elementos ``ativos'' nos primeiros pontos, especialmente no ponto 3. Do 4 ao 6 temos condições situacionais e estruturais, elementos passivos. A ênfase em unidades estruturais e na recriação dos laços afetivos com outros membros do culto é combinada com a vontade pessoal do convertido, gerando assim uma espécie de sinergia entre os componentes. No entanto, como este foi um dos primeiros modelos criados e ainda foi elaborado com base em um culto não convencional ele apresentava erros e foi bastante criticado, em especial pela falta de fundamentação empírica e especificidade do caso estudado, o que limitava generalizações e replicações \autocite[338]{gooren_reassessing_2007-1}.

Um último modelo misto a ser destacado é o de Carreira de Conversão elaborado por Richardson\autocite*{richardson_conversion_1977} em seu livro \emph{Conversion careers: In and out of the new religions} e refinado por Gooren em seus trabalhos posteriores \autocite*{gooren_reassessing_2007-1,gooren_religious_2010-1}. Pensar este fenômeno enquanto carreira permite analisá-lo por várias frente e ainda captura com clareza as várias experimentações realizadas ao longo da vida religiosa de um sujeito. Para além disso ele é sensível a idade, elemento que não era levado em consideração até então \autocite[49]{richardson_active_1985}. Gooren, no entanto, incrementa esse conceito ao redefini-lo enquanto ``passagem do membro, dentro de seu contexto social e cultural, por níveis, tipos e fases de participação na igreja'' \autocite[349]{gooren_reassessing_2007-1}. O foco deixa de ser apenas a carreira num sentido macro, interinstitucional, mas também micro, dado que avalia os níveis de participação \autocite{girardo_rodriguez_religious_2021}. O modelo, enfim, se enriquece ao incluir tanto os pré estágios da conversão quanto os finais que configuram no afastamento da instituição, chamada de desfiliação. A imagem abaixo representa o que viria a ser o ciclo de vida do indivíduo numa instituição religiosa e os níveis de cometimento:
\begin{figure}[H]

{\centering \includegraphics[width=0.8\linewidth]{images/imagem01} 

}

\caption{Fases da Conversão Religiosa.}\label{fig:imagem}
\end{figure}
\bcenter

Fonte: Gooren\autocite*{gooren_reassessing_2007-1}
\ecenter

A pré afiliação consiste nas visitas esporádicas ou de reconhecimento feitas a uma dada religião/culto, elas configuram uma aproximação sem compromisso. A filiação, por sua vez, representa um aumento no grau de cometimento com a instituição, as visitas ocorrem com mais frequência e o indivíduo se torna um membro formal. A conversão representa a alteração da identidade e das visões de mundo, percebida pelo próprio sujeitos e confirmada pelos que estão ao redor. A confissão representa o grau de cometimento completo com a visão da organização e o desenvolvimento de um ``espírito missionário'' em relação aos não membros. A desfiliação, por fim, representa o afastamento total \autocite[350]{gooren_reassessing_2007-1}.

Dois elementos deste modelo podem ser destacados. O primeiro é a relação entre agência e estrutura representada pelas setas, percebe-se que a qualquer momento é permitido ao sujeito a possibilidade de ``pular'' ou ``regredir'' estágios. É totalmente possível, por exemplo, sair da Conversão para a Desfiliação. A vontade pessoal do indivíduo e seu poder de controle do processo são levados em conta ao mesmo tempo que não se omite o peso que a estrutura tem ao lhe oferecer uma ``rota institucional''. O segundo ponto a ser destacado é que também são levados em consideração nesse modelo fatores etários, sociais, culturais, contingenciais, institucionais e de personalidade. De modo a criar um panorama completo da trajetória pessoal \autocite[351]{gooren_reassessing_2007-1}.

\hypertarget{o-que-muda-e-o-que-resiste-o-estado-da-conversuxe3o-nos-estudos-acaduxeamicos-nacionais}{%
\subsection{O que Muda e o que Resiste: O Estado da Conversão nos estudos acadêmicos nacionais}\label{o-que-muda-e-o-que-resiste-o-estado-da-conversuxe3o-nos-estudos-acaduxeamicos-nacionais}}

Até agora foram debatidos alguns dos modelos de Conversão mais utilizados pela Sociologia Norte Americana. Estes modelos, no entanto, são pensados segundo a realidade de onde foram criados e não costumam se encaixar perfeitamente nas vivências de outras localidades, especialmente em países como o Brasil aonde a cultura e as vivências religiosas possuem padrões bastante específicos. Nos parágrafos abaixo irei discutir rapidamente sobre as especificidades do caso brasileiro, em especial sobre como este fenômeno vem sendo adereçado, tipologias empregadas e discussões atuais.

\hypertarget{o-caso-brasileiro}{%
\subsubsection{O Caso Brasileiro}\label{o-caso-brasileiro}}

De modo geral o panorama brasileiro se caracteriza pela maioria da população sendo criada dentro do Catolicismo, religião majoritária no país. É nesse meio aonde a vida religiosa de muitos é iniciada e de onde advêm boa parte dos elementos presentes no dia a dia como, por exemplo, nomes de ruas e estados (São Paulo, Santa Catarina, Espírito Santo etc.) \autocite{prandi_converting_2008}. Por consequência prática não seriam identificados muitos casos de Conversão caso a análise fizesse uso, por exemplo, das ferramentas da Escolha Racional. As mudanças para o Protestantismo seriam interpretadas como simples refiliação dado que ambas as religiões compartilham do mesmo arcabouço cristão.

Para além disso outras especificidades tendem a ir contra uma parte dos modelos existentes. A IURD\footnote{Igreja Universal do Reino de Deus}, por exemplo, que se encaixa no rol das chamadas \emph{Religiões de Salvação}, ou seja, possuem foco em Conversão e dão a garantia da Salvação Eterna \autocite[121]{pierucci_religiao_2006}, não tem forte apelo a ruptura e nem exige, necessariamente, mudanças bruscas nos padrões de comportamento. Todas essas particularidades aqui citadas vão totalmente contra o que os modelos Norte Americanos de Conversão pressupunham \autocite{mafra_na_2018}.

Foi com essas questões em mente que Sociólogos e Antropólogos, mais os segundos do que os primeiros, se debruçaram sobre a temática e geraram modelos e discussões, ora profícuas, ora problemáticas, que pudessem fornecer pistas para entender esse fenômeno e suas especificidades no caso brasileiro. Os estudo sobre Conversão começam no país na década de 80 e tendem a ser relacionados, no início, ao Pentecostalismo que começava a ganhar força no país \autocites[156]{prandi_converting_2008}[10]{lamb_religious_1999} e aos encontros e desencontros da Umbanda com o Catolicismo \autocite{birman_cultos_1996} e depois com o Protestantismo \autocite{mafra_o_2002}. Outras pesquisas também se debruçavam sobre a relação entre os processos de Conversão outros elementos da vida social como Gênero \autocite{machado_conversao_1998}, Pobreza \autocite{mariz_religiao_1991}, Crime \autocite{cunha_traficantes_2008} e até mesmo Comunidades Terapêuticas \autocite{targino_experiencias_2020}.

Dentro do escopo destas pesquisas as definições de Conversão tendem a levar em consideração três elementos: \emph{Ruptura}, \emph{Negociação} e \emph{Gramática} \autocite{teixeira_processos_2021}. Esses conceitos costumam vir interligados nas definições de tal forma que falar deste tema no contexto nacional é falar diretamente sobre eles. A ruptura diz respeito ao processo de geração de novos sujeitos a partir de rompimentos com versões de si, tratando então o fenômeno enquanto processo de \emph{descontinuidade existencial}. Apesar de se aceitar essa descontinuidade também se aceita que ela não é necessariamente total ou radical. Seja por motivos culturais ou por consequências da modernidade, é possível constatar a existência de áreas cinzas dentro do cometimento que permitem que a mudança seja mediada, ou negociada, pelo próprio sujeito como foi destacado, por exemplo, no caso da IURD. A esse processo dá-se o nome de Negociação. A gramática, por fim, diz respeito ao conjunto de ferramentas oferecidas pela doutrina que são assimiladas pelo indivíduo e que auxiliam na tradução do mundo ao redor. É também através da obtenção e uso dela que as mudanças de \emph{self} advindas dos processos de ruptura costumam se manifestar no mundo exterior.

\hypertarget{minimalismo-e-maximalismo}{%
\subsubsection{Minimalismo e Maximalismo}\label{minimalismo-e-maximalismo}}

Um dos modelos desenvolvidos com base nas noções discutidas no parágrafo anterior, e que se destaca enquanto exemplo de como esses conceitos são trabalhados, é o de Clara Mafra. Em seu livro \emph{Na posse da palavra: Religião, Conversão e Liberdade Pessoal}, a autora constata, a partir de trabalho de campo, a existência de dois tipos de conversão: A \emph{Maximalista} e a \emph{Minimalista}. No modelo maximalista ela acontece a partir da coesão e do controle do grupo, que o encaminha para um ``novo mundo de crenças e disposições'' \autocite{mafra_na_2018}, ou seja, aqui tem-se o modelo tradicional, aonde a estrutura se sobrepõe à vontade individual.

O grande diferencial desse modelo se encontra no que ela chama de \emph{Conversão Minimalista}. Nesse tipo ocorre uma maior plasticidade entre a interação agência/estrutura, de forma que não se torna necessário uma mudança abrupta de vida ou uma transformação radical de hábitos para que o fenômeno aconteça. O que conta é articulação feita pelo próprio sujeito do capital adquirido via Conversão \autocite{mafra_na_2018}. Nos casos por ela analisados os sujeitos pesquisados sempre moldavam de forma \emph{sui generes} como este evento atingia suas vidas. Não havia mudanças de sociabilidade, nem necessidade de ressignificação total do passado, tal qual preconizava os modelos antigos.

Apesar dessa autonomia não se deve confundir esse processo com o da individualização advinda da modernidade que aparece nas discussões dos modelos ativos de Conversão. No modelo minimalista os processos são orgânicos e não existe uma individualização \emph{ipsi literi}, mas sim a construção de um \emph{``projeto de identidade''} que funciona ``no tempo de Deus'', ou seja, é aberto a intempéries e obras do acaso. Essa forma de vivenciar a Conversão é bastante próxima dos processos de bricolagem citados por Hervieu-Leger e dos modelos de trajetória vistos no capítulo anterior.

Da mesma forma que outros modelos aqui citados destaca-se como ponto a ser criticado a desatenção a elementos estruturais que também possuem impacto sobre como o indivíduo dispõe e manuseia do que lhe é oferecido. Apesar disso, admite-se que essa omissão é, em parte, compensada pela atenção aos contextos no qual as Conversões ocorrem.

\hypertarget{panorama-atual}{%
\subsubsection{Panorama Atual}\label{panorama-atual}}

A crítica a falta de atenção a estrutura, apesar de compensada em alguns casos, é apenas um dos problemas no que diz respeito a como a Conversão vem sendo tratada a algum tempo na literatura nacional. Entende-se a vivência da religiosidade nas pesquisas mais atuais enquanto um processo extremamente fluido, no qual se imperam \emph{Continuidades} \autocite{almeida_transito_2001}. Assume-se, na maioria das vezes, que os processos são negociáveis e que em alguns casos são até inexistentes. A consequência prática disso é um certo ``sumiço'' do conceito de filiação. A vivência da religiosidade, mais do que nunca, acontece dentro dos limites impostos pelo próprio sujeito o que cria híbridos e formas únicas de se viver, ou não viver, a fé. Não é baixo o número de produções, principalmente na Antropologia, que são criadas nesses moldes e dão força a essa concepção \autocite{oro_o_2006,reinhardt_espelho_2006,aureliano_eu_2021}.

Tal perspectiva advém, em parte, do peso excessivo que se dá as noções de negociação e de sincretismo nas análises. É comum no ambiente acadêmico pensar o brasileiro enquanto sujeito de pouco cometimento e propenso ao sincretismo \autocite[232]{frigerio_identidades_2005}, essa forma de enxergar, ou \emph{estrabismo} como pontuam Campos e Reesink \autocite*[55]{reesink_conversao_2014}, penetra as pesquisas sobre religião na forma de elevada atenção a aspectos que favorecem essa porosidade, como multifiliação e fluidez das trajetórias religiosas ou até mesmo no hábito de produzir pesquisas que favoreçam visões desse tipo como as citadas a pouco. Ser brasileiro é ser sincrético e visões contrárias a essa lógica vem sendo pouco debatidas pela produção nacional \autocite{reesink_conversao_2014}.

Isso, no entanto, tem como principal consequência a ``morte'' do conceito de Conversão no panorama intelectual nacional \autocite{reesink_conversao_2014} dado que esse fenômeno repousa, como visto anteriormente, nas noções de rompimento e transformação. Se tais elementos são suprimidos da equação e se admite, por outro lado, o oposto disso, a ideia perde o sentido e sua aplicação em produções científicas se torna menor. Em outras palavras, para a Academia, brasileiros não parecem mais se converter à moda antiga.

Apesar de ser importante reconhecer o que muda nesses processos e quais novidades surgem é tão importante quanto entender o que permanece. Aceitar que existem continuidades e negociações não deve ser feito a base de suprimir o oposto. A flexibilização e a filiação total coexistem na realidade e devem ser considerados nas produções, em ordem de possibilitar um entendimento honesto da vivência religiosa. É no \emph{processo}, no fim das contas, e não nos elementos específicos que dele despontam que se deve focar. Não considerar este fenômeno enquanto processual apenas aumenta as chances de produzir visões distorcidas e enviesadas do campo religioso e de suas vivências \autocite{engelke_discontinuity_2004,teixeira_processos_2021,reesink_conversao_2014}.

\hypertarget{definindo-conversuxe3o}{%
\subsection{Definindo Conversão}\label{definindo-conversuxe3o}}

Toda esta discussão sobre como a Conversão acontece oferece, no fim das contas, várias pistas de como defini-la. Uma das primeiras é que uma variedade de abordagens gera uma variedade de definições, logo não existe uma resposta concreta sobre ``o que é Conversão Religiosa''. Tudo depende, no fundo, do ponto de vista que se opta por assumir. A primeira definição, e também a que impera no senso comum, é a de que Conversão é a simples mudança de igrejas/culto. Isso, no entanto, é oriundo do senso comum e pouco agrega a uma visão científica deste fenômeno.

Em ordem de superar esta limitação foram criadas algumas definições das quais três se destacam pelo uso na construção de vários modelos sociológicos. A primeira é a da Conversão enquanto \emph{``mudança pessoal radical''}. Muito comum nos modelos passivos, em especial no Paulino, um sujeito só pode se considerar um convertido quando acontecer uma ruptura total em sua personalidade e formas de ser/estar no mundo \autocites[169]{snow_sociology_1984}[224]{snook_issues_2019}. Essa definição, no entanto, recaí sobre a visão teológica e não há especificações sobre como e em qual grau essa mudança deve ocorrer para que seja considerada uma Conversão \autocite[169]{snow_sociology_1984}, o que colabora com a diminuição do uso dessa definição específica nas pesquisas hodiernas sobre o assunto.

Uma segunda definição é a fornecida por Stark e Finke \autocite*{stark_acts_2000-1} ao proporem o modelo da Escolha Racional. Para os autores ela é caracterizada única e exclusivamente pela ``mudança de tradição religiosa'' \autocite[114]{stark_acts_2000-1}. A transição entre religiões dentro de um mesmo espectro, como por exemplo, Catolicismo e Protestantismo, não seria categorizada como Conversão, e sim Refiliação. Casos de Conversão não são comuns dado que geram altos gastos de energia e Capital Religioso \autocite[119]{stark_acts_2000-1}. Apesar de ser um modelo bastante pragmático se torna difícil a aplicação em locais com hegemonia religiosa e poucas opções de mercado.

Uma última definição, essa mais utilizada nas pesquisas contemporâneas e que será aqui adotada, é a da Conversão enquanto \emph{``Mudança no universo de discurso''}. Para além de produzir mudanças na identidade, valores e crenças do indivíduo, ela opera uma mudança no seu universo discursivo, na realidade central e é essa mudança específica que caracterizaria a Conversão Religiosa \autocite[170]{snow_sociology_1984}. Quando se opera uma mudança no universo discursivo se altera as formas até então utilizadas para entender e traduzir o mundo ao redor, se um elemento \emph{X} era traduzido como \emph{A} antes da Conversão ele passa a ganhar um novo valor, \emph{B}, fundamentado na teologia transmitida. O infográfico abaixo ilustra esse processo.
\begin{figure}[H]

{\centering \includegraphics[width=0.8\linewidth]{images/imagem02} 

}

\caption{Modelo de Mudança de Universo de Discurso.}\label{fig:imagem3}
\end{figure}
\bcenter

Fonte: Fonte: \autocite{snow_sociology_1984}
\ecenter

Um exemplo prático disso seria quando, no meio pentecostal, o alcoolismo é reinterpretado como ação demoníaca através da gramática pentecostal recém adquirida, ao invés de patologia \autocite{mariz_libertacao_1994}.

\hypertarget{estratuxe9gias-de-identificauxe7uxe3o-e-mensurauxe7uxe3o-da-conversuxe3o-religiosa}{%
\subsection{Estratégias de Identificação e Mensuração da Conversão Religiosa}\label{estratuxe9gias-de-identificauxe7uxe3o-e-mensurauxe7uxe3o-da-conversuxe3o-religiosa}}

A discussão até então feita mostra como as pesquisas lidam com a Conversão enquanto elemento subjetivo ou como fenômeno aonde a subjetividade desempenha um grande papel. Foi destacada também a sinergia entre agência e estrutura e como cada um desses age no processo aqui discutido. O conjunto dessas implicações levam grande parte das pesquisas sobre a temática a serem de ordem qualitativa, dado que se torna mais fácil capturar todas as nuances da questão com essa metodologia. O uso frequente dessa técnica, no entanto, gerou uma grande lacuna sobre como discutir isso do ponto de vista quantitativa, ou como gerar dados mensuráveis sobre este fenômeno. Irei gastar algumas linhas neste final de seção para endereçar essa problemática, como ela vem sendo trabalhada e como ela será trabalhada nesta pesquisa em especial.

Pode-se destacar três tipos de indicadores empíricos, logo passíveis de mensuração, de conversão religiosa utilizados pela pesquisa sociológica: filiação, demonstração e retórica \autocite[171]{snow_sociology_1984}. A primeira tática, e talvez a mais simples, se baseia na consideração da conversão enquanto mudança de religião/culto, logo a simples refiliação seria o suficiente para categorizar um indivíduo como convertido. Essa técnica era bastante comum nos estudos iniciais \autocite{newport_religious_1979,roof_denominational_1979}. Apesar de prático vários outros aspectos da conversão que foram debatidos até então ficavam de fora ao considerar a simples filiação como indicador. Esse tipo de classificação logo caiu em desuso.

O segundo tipo, demonstrações, é um dos mais empíricos e se baseia na consideração de manifestações públicas de fé, por exemplo, testemunho, repouso no espírito, glossolalia, batismo etc. como \emph{proxies} de conversão. Essas demonstrações, no entanto, não são necessariamente indicativos. Na maioria das vezes elas dizem mais respeito a atos de reavivamento/reafirmação do que Conversão em si \autocite[172]{snow_sociology_1984}. Em suma por mais que essas atitudes possam ser feitas por pessoas que sejam convertidas, não quer dizer que quem as faça seja, necessariamente, convertido.

O último indicador, o retórico, é o que mais ressoa com as pesquisas mais recentes, dado que levam em consideração a agência do indivíduo e se baseia em elementos discursivos presentes nas falas de pessoas convertidas. Ele é fundamentado em quatro pontos:
\begin{quote}
\begin{enumerate}
\def\labelenumi{\arabic{enumi}.}
\tightlist
\item
  reconstrução biográfica;
\item
  vocabulário de motivos;
\item
  suspensão do pensamento analógico;
\item
  adoção do papel de convertido. \autocite[174-175]{snow_sociology_1984}
\end{enumerate}
\end{quote}
Inicialmente se tenta perceber se o sujeito reconstrói a sua biografia a partir do universo de discurso da nova fé (1\textordmasculine ponto) , o próximo passo é avaliar se existe a absorção deste vocabulário e o uso dele no cotidiano (2\textordmasculine ponto), seguido de um sentimento de exclusividade nessas percepções mundanas, de modo que a forma como se interpreta o espaço ao redor é incompatível com lógica não religiosa (3\textordmasculine ponto). Por fim se estabelece um modo de ser/estar no mundo guiado pelos direcionamentos do culto/religião (4\textordmasculine ponto). Apesar de robusto esse modelo foi desenhado com base em um estudo sobre um único culto, logo carece de suporte empírico sobre seu funcionamento \autocite[175]{snow_sociology_1984}.

A existência desses compilados de indicadores, no entanto, não significa que eles possam ser aplicados em todos os casos possíveis. Especificidades do caso brasileiro, por exemplo, não parecem ser plenamente contempladas em nenhum destes três citados anteriormente. O processo se torna ainda mais difícil quando não se tem pesquisas no mesmo contexto nacional que tenham enfrentado problema similar e possam apontar soluções. O desafio é encontrar ou construir uma opção que leve em consideração as especificidades/subjetividades envolvidas nessa questão ao mesmo tempo que não se ignora as questões estruturais da conversão. Para isso o primeiro trabalho foi o de identificar elementos comuns nos modelos e definições para então descobrir quais poderiam ser transformados em indicadores empíricos da Conversão Religiosa.

Segundo Snook et al. \autocite*{snook_issues_2019} existem dois pontos que estão presentes em grande parte das definições de Conversão Religiosa: \emph{mudança de percepções pessoais} e \emph{melhorias de vida}. O primeiro quesito é perceptível em grande parte das definições aqui abordadas e também é levada em consideração nos modelos. O segundo item é originado do anterior e apoiado por várias pesquisas que apontam os benefícios da Conversão \autocite{cerqueira-santos_religiao_2004-1,krause_stress_2011,verona_explanations_2011,mariz_libertacao_1994,potter_growth_2016}.

Ambos os itens, apesar de subjetivos, compartilham uma característica importante para o desiderato desse estudo: podem ser analisados de forma objetiva através do contraste entre o período anterior a Conversão e o período posterior. Somando esses itens a pergunta direta sobre Conversão é possível detalhar mais o processo vivenciado pelo sujeito que se diz convertido.

Uma segunda estratégia é seguir o modelo de Gooren \autocite*{gooren_reassessing_2007-1}, de forma que a pergunta sobre conversão sai da dualidade (sim/não) e adota mais opções que condigam com as fases da conversão (pré-afiliação, afiliação, conversão, confissão e desfiliação). Apesar de adotar mais opções formular a pergunta seguindo este modelo permitirá que se observe de forma mais efetiva em qual parte do ciclo o respondente se encontra e a partir disso.

Apesar de possíveis, ambas as opções não encaixariam muito bem na realidade nacional e seriam difíceis de captar. Em ordem de sanar essa problemática preferiu-se adotar a definição de Gooren com algumas modificações. Ao invés de assumir as fases da conversão preferiu-se criar uma opção que lida com a ``área cinza''. Admitindo que a Conversão é um processo estabelece-se um meio ponto entre ser convertido (1) e não ser (0) intitulado \emph{Estou em processo de conversão} (0.5). Longe de atribuir uma fase na qual o respondente está se prefere entender que ele admite estar em um estado de semi pertencimento, assim é possível capturar, ao mesmo tempo, a noção plena de Conversão e elementos negociáveis que a permeiam.

\pagebreak

\hypertarget{as-comunidades-terapuxeauticas}{%
\section{As Comunidades Terapêuticas}\label{as-comunidades-terapuxeauticas}}

Comunidades Terapêuticas (doravante CT's) podem ser definidas de forma genérica enquanto centros de terapia focados na recuperação \emph{(recovery oriented)} de indivíduos que escolhem por livre e espontânea vontade receberem tratamento para males de ordem mental baseado em vivência comunitária (também chamada de ``entre pares'') e na transformação de hábitos e comportamentos. Elas estão presentes atualmente em mais de 60 países \autocite[87]{avery_current_2019} e possuem público extremamente variado, sendo esta variação um dos elementos chave para a construção e reconstrução dos tipos de CT'S. Atualmente os principais modelos empregados são os \emph{Baseados em Justiça Criminal}, \emph{Baseado nos 12 passos}, \emph{Religioso}, \emph{Democrático} e \emph{Modificado}, cada um destes com suas próprias variações internas \autocite[37]{avery_modern_2019}.

Esses diversos modelos, no entanto, podem ser divididos por uma linha em dois grandes grupos: \emph{Democráticos} e \emph{Exclusivo para Dependentes Químicos} sendo o nível de liberdade e poder hierárquico, especificamente o nível de horizontalidade entre administração e internos, o que os define. Neste estudo o foco será dado para o segundo tipo de \acrshort{CT}, dado que ela é a que predomina no Brasil.

Essa variação gera uma das primeiras discussões a serem feitas neste capítulo: Como esses modelos se comportam? O que uma pessoa que opta por um deles irá encontrar? O que aproxima e distancia cada tipo de CT? Existiria algum tipo de predominância ou preferência? Se sim, como ela se manifesta? Outro ponto importante é avaliar como essas instituições se instalam em solo brasileiro: Quais adaptações elas sofrem ao chegar aqui? Qual seu estilo, seus itinerários, suas características e seu nível de aproximação com outras instituições, sejam elas públicas ou privadas/ Essas questões e outras mais serão discutidas neste subcapítulo e no próximo, aonde uma linha do tempo para o caso brasileiro será oferecida.

\hypertarget{caracteruxedsticas-gerais-das-cts}{%
\subsection{Características Gerais das CT's}\label{caracteruxedsticas-gerais-das-cts}}

O modelo ``padrão'' de CT surgiu no período pós Segunda Guerra Mundial a partir do trabalho de Maxwell Jones sobre stress pós combate entre soldados \autocite{smart_outcome_1976,jones_therapeutic_1953} e era definido como um modelo focado na recuperação de indivíduos através da vivência comunitária (também chamada de ``entre pares'') e na transformação de hábitos e comportamentos. Mesmo tendo sido criado no século XX era possível encontrar instituições com modos similares já no século XVIII. Criada pelos Tuke, uma família de comerciantes quakers, \emph{The Retreat} tinha como principal público os chamados ``insanos''. Nesse lugar era aplicado o chamado \emph{Tratamento Moral}. Baseado em princípios humanitários e religiosos o Retiro\footnote{Tradução literal de \emph{The Retreat.}} concedia ao paciente a chance de experimentar os confortos de uma ``boa vida'' e da presença de ``relacionamentos amigáveis'' \autocite[234]{whiteley_evolution_2004} ao mesmo tempo que mantinha contato com princípios da doutrina Quaker, como ``produtividade física e mental'', visando a produção de uma moral ressocializadora que era a base do ``Tratamento Moral'' empregado \autocite[05]{mcbride_advancement_2017}.

Como dito anteriormente, existem cinco modelos de CT's que possuem variações específicas. Tais variações surgem em ordem de acomodar melhor o público e pelas modificações operadas na teoria central através do contato com novas informações ou formas de cuidado, o que cria processos de hibridização \autocite[36]{avery_modern_2019}. A tabela abaixo resume cada um desses tipos, assim como suas características e subtipos.

\setcounter{table}{0}
\begin{table}[H]

\caption{\label{tab:ct}Descrição dos tipos existentes de Comunidades Terapêuticas geradas a partir de Hibridização.}
\centering
\begin{tabular}[t]{>{\raggedright\arraybackslash}p{1.5in}>{\raggedright\arraybackslash}p{2.8in}>{\raggedright\arraybackslash}p{2in}}
\toprule
\textbf{Modelo} & \textbf{Descrição} & \textbf{Subtipos/Públicos}\\
\midrule
Religiosa & CT’s que fazem uso da fé/religiosidade/espiritualidade como parte central do tratamento. & Cristã Católica, Cristã Ortodoxa, Cristã Protestante, Mulçumana, Hinduísta\\
12 Passos & CT’s que fazem uso das metodologias do AA/NA como parte central do tratamento. & AA; NA\\
Modificada & CT’s que se adaptam a novas terapias e públicos & Adolescentes, Duplo diagnóstico\\
Justiça Criminal & CT’s que fazem parte do sistema de justiça e atendem público que tem como parte da pena a participação em Comunidades & Comunidades em Prisões, Menores infratores, Comunidades para prisioneiros\\
Democrática & CT’s que fazem uso de técnicas do Serviço Social ou Psiquiatria & Teoria dos Sistemas, Psicanalítico, Comunidades em Prisões\\
\bottomrule
\end{tabular}
\end{table}
\bcenter

Fonte: \autocite{avery_modern_2019}
\ecenter

Apesar dessa divisão bastante estrita é possível ainda subdividir esses grupos em dois grandes tipos: \emph{Democrático} e \emph{Exclusivo para dependentes químicos}.

O primeiro é mais comum nos países do continente europeu e tem como maior característica, como o próprio nome indica, o seu caráter democrático no qual os chamados residentes possuem um grande papel na tomada de decisões e são estimulados a desenvolverem habilidades sociais através da promoção de valores como cidadania, empoderamento pessoal, responsabilidade coletiva e liderança carismática. A hierarquia nesse tipo de CT tende a ser mais flexível \autocite{campling_therapeutic_2001,smart_outcome_1976} e o público atingido por esse modelo tende a ser bastante amplo, sendo encontrada na literatura evidências de sua aplicação em, por exemplo, crianças e adolescentes que passaram por eventos traumáticos \autocite{diamond_mulberry_2013} e Detentos \autocite{rawlings_therapeutic_2017}. Se encaixam nesse escopo os modelos Modificado, Democrático e Baseado em Justiça Criminal.

Já o segundo tipo, formado pelos 12 Passos e Religiosos, é mais comum na América, principalmente nos Estados Unidos (aonde foi criado) e em alguns países da América Latina, incluindo o Brasil. Ele tende a ser o oposto do modelo Democrático e tem como principais características o foco na geração de mudança comportamental, programa terapêutico construído com base num esquema de recompensas e graus, forte obediência a uma hierarquia rígida e bem delineada, sistema de confronto direto e uso de modelos explicativos simples da dependência química e do tratamento \autocite[365]{campling_therapeutic_2001}. Ele surge a partir do Programa Synanon criado por Charles Dederick em 1958, e tem sua origem no chamado Grupo de Oxford, que também foi precursor dos Alcoólicos Anônimos (\acrshort{A.A}). O Grupo era uma organização religiosa fundada pelo Dr.~Frank Nathan Daniel Buchman em 1921 sendo majoritariamente constituída por alcoólatras e que tinha como principal missão provocar o ``renascimento espiritual da humanidade'' \autocite[16]{glaser_origins_1981}.

Tal modelo teve várias vezes seu caráter terapêutico questionado, dada a presença de forte hierarquia e a existência de admissão forçada de internos \autocite[145]{smart_outcome_1976} sendo a própria Synanon, por exemplo, obrigada a fechar as portas em 1991 após Charles Dederick transformar seu programa terapêutico em uma espécie de ``residência alternativa'', na qual foram relatados casos de abuso infantil e tentativa de assassinato de internos que não concordavam com a hierarquia interna \autocite[30]{mitchell_light_1980}.

É perceptível ao olhar a linha evolutiva das CT's como um todo, e especialmente nos modelos Exclusivos para Dependentes Químicos, o quanto a Religião é um elemento presente e que permeia as práticas por ela adotadas. Tanto Charles Dederick quanto o Dr.~Frank Buchman alegam terem vivenciado experiências místicas que os conduziram na criação de seus trabalhos terapêuticos \autocite[16]{glaser_origins_1981}. Elementos gerais do Cristianismo e específicos de algumas doutrinas costumam quase sempre ser infundidos no \emph{milieu} terapêutico oferecido por estas instituições. No Grupo de Oxford, por exemplo, a Conversão Religiosa era não apenas incentivada como também era um dos quatro pilares do ``tratamento''. A mudança provocada por ela livraria o indivíduo de todo pecado, e como consequência prática, do uso abusivo de drogas \autocite[18]{glaser_origins_1981}. Tal visão ainda persiste em uma parte das CT's, principalmente nas ditas ``Religiosas'', nas quais a Conversão tende a ser considerada peça chave para a Recuperação \autocites[198]{chu_religious_2012}[167]{mcbride_therapeutic_2018}.

\hypertarget{comunidades-terapuxeauticas-no-brasil}{%
\subsection{Comunidades Terapêuticas no Brasil}\label{comunidades-terapuxeauticas-no-brasil}}

No Brasil, como dito em parágrafos anteriores, é comum a existência de Comunidades voltadas exclusivamente para indivíduos que fazem uso abusivo de drogas, sendo ainda mais comuns as ditas ``Religiosas''. As primeiras CT's, \emph{Desafio Jovem} e \emph{Fazenda do Senhor Jesus}, surgem em território nacional durante as décadas de 60 e 70, sendo a primeira criada em 1968 por um movimento Norte Americano de origem protestante e a segunda em 1978 pelo padre missionário Norte Americano Haroldo Rahm em Campinas - SP \autocite[99]{fossi_o_2015}. Desse período em diante elas começam a expandir até que em 1990 é criada a Federação Brasileira de Comunidades Terapêuticas (FEBRACT). Segundo dados do Ministério da Justiça, em 2016 existiam pelo menos 1,830 CT's ativas e com \acrshort{CNPJ} cadastrado, o infográfico abaixo aponta a distribuição destas pelo país.
\begin{figure}[H]

{\centering \includegraphics[width=0.85\linewidth]{images/imagem03} 

}

\caption{Distribuição das Comunidades Terapêuticas pelo Território brasileiro em 2016.}\label{fig:imagem2}
\end{figure}
\bcenter

Fonte: Ministério da Justiça, 2016
\ecenter

A maioria das CT's em 2016 se concentravam nas regiões Sul e Sudeste do país e juntas elas representavam mais de 60\% do total existente. Para além dos dados do próprio Ministério da Justiça, uma pesquisa encomendada pelo \acrshort{IPEA} em 2017 intitulada \emph{``Perfil das Comunidades Terapêuticas Brasileiras''} achou resultado similar. 67\% das CT's por ele pesquisadas por eles provinham dessas regiões.

\hypertarget{guxeanero}{%
\subsubsection{Gênero}\label{guxeanero}}

O Banco de dados do MJ revela ainda que 80\% do público que é atendido é masculino e apenas 20\% delas aceitavam, seja de forma exclusiva ou parcial, indivíduos do sexo feminino. Esse dado se torna mais preocupante quando se percebe que cada vez mais o uso de substâncias ilícitas deixa de ser um ``privilégio'' masculino e se torna homogeneizado entre os sexos. Dados da Fiocruz demonstram não haver diferenças significativas entre o uso de substâncias ilícitas quando analisado pelo viés de sexo\autocite{bastos_iii_2017}. Em outras palavras, homens e mulheres se tornam dependentes numa frequência bastante similar \autocite[133]{bastos_iii_2017} de forma que dar atenção a um e não a outro além de incoerente pode significar a morte de várias mulheres vítimas deste mal.

Esse tipo de problemática, também presente no sistema prisional \autocite{cury_mulher_2017}, não é apenas sinal de que existe no país um viés de gênero em relação aos métodos de acolhimento e reinserção social, mas que o próprio sistema é desenvolvido com base nessa desigualdade. Ao pesquisar internas de Comunidades Terapêuticas, Targino \autocite*[14]{targino_experiencias_2020} constata em suas narrativas que elas consideram o período de internação e o contato da religiosidade como algo que as faz entrar em contato com a feminilidade e as reensina como ser uma mulher decente que cumpre com seus papeis sociais de mãe/esposa de forma plena, o que é similar ao que tentam incutir na mulher no sistema prisional que, por meio da pena, tentam recobrar seu pudor \autocite[02]{cury_mulher_2017}

\hypertarget{filiauxe7uxe3o-religiosa}{%
\subsubsection{Filiação Religiosa}\label{filiauxe7uxe3o-religiosa}}

Outra característica das CT's nacionais é a forte presença da Religião e seus elementos. Um dos primeiros demonstrativos disto é que pelo menos 82\% dessas instituições são mantidas por igrejas e entidades religiosas, sendo a maioria dela cristãs.
\begin{figure}[H]

{\centering \includegraphics[width=1\linewidth]{images/g1-1} 

}

\caption{Filiação Religiosa das Comunidades Terapêuticas Nacionais.}\label{fig:g1}
\end{figure}
\bcenter

Fonte: \autocite{ipea_perfil_2017-1}
\ecenter

Para além do fato de serem mantidas por igrejas, duas das dez atividades mais empregadas por CT's no Brasil tem relação direta com religiosidade. Em 89\% delas a Bíblia é lida diariamente e em 88\% Orações e Cultos também são feitas com a mesma frequência. Mesmo que tais ações sejam diretamente ligadas com religião (dado que essa Bíblia que é lida e essas orações/cultos que são feitos pertencem a um credo institucionalizado) o relatório os classifica simplesmente como ``Espiritualidade''. O foco da análise é direcionado para a prática de ações ``técnico-científicas'' como atendimento psicoterápico, que costuma aparecer em maior porcentagem \autocite[22]{ipea_perfil_2017-1}. O mascaramento da religiosidade sobre o termo guarda-chuva ``Espiritualidade'', somada a ênfase nessas atividades ``científicas'', são elementos comuns nas narrativas feitas tanto pelas entidades governamentais quanto pelas próprias CT's. Este assunto será novamente abordado na terceira seção deste capítulo.
\begin{figure}[H]

{\centering \includegraphics[width=1\linewidth]{images/g2-1} 

}

\caption{Dez Atividades diárias mais praticadas em CT's brasileiras.}\label{fig:g2}
\end{figure}
\bcenter

Fonte: \autocite{ipea_perfil_2017-1}
\ecenter

Ainda dentro da questão da religiosidade percebe-se que uma marca desse tipo de tratamento é o incentivo a Conversão Religiosa. Existem pesquisas que alegam a não existência dessa prática \autocite{loeck_dependencia_2014} ou que a abordem de outro ponto de vista. Dentre elas pode-se destacar o próprio relatório do IPEA que alega que a religiosidade no contexto das CT's não tem como meta principal a Conversão Religiosa do interno e sim ``(\ldots) sua conversão moral, onde a fé no divino e o apoio das escrituras sagradas do cristianismo são percebidos como aliados poderosos, seja na proteção contra as recaídas, seja na fixação de uma ética heterônoma de conduta'' \autocite[35]{ipea_perfil_2017-1}. Esse processo de conversão moral, no entanto, acarreta uma série de operações que geram mudanças de visão de mundo e de significação, não obstante, processos de conversão religiosa. Para além disso, existem pesquisas que demonstram a existência dessa tradição nas CT's nacionais \autocite{targino_comunidades_2017-1,targino_experiencias_2020,mcbride_therapeutic_2018}.

Para além de todas estas colocações é válido pontuar que não existem pesquisas ou estudos consolidados que justifiquem ou comprovem o estímulo à Conversão Religiosa enquanto fator que promove resultados positivos para a Recuperação e manutenção da abstêmia, seja em tratamento ou fora dele \autocite[356]{shields_religion_2007}. Há, de fato, estudos que ligam religiosidade a menores chances de contato com drogas ilícitas durante a adolescência \autocite{hill_religious_2009} e com baixos níveis de stress, ansiedade e depressão na vida adulta \autocite{krause_stress_2011}. Poucos, no entanto, avaliam a relevância disso em pessoas que já entraram no estado de uso abusivo ou que se submeteram a tratamentos infundidos de religião ou totalmente religiosos \autocite{davis_religious_2014-1,flynn_looking_2003,shields_religion_2007,chu_religious_2012,perrone_fatores_2019}. Se pesquisas desta alcunha são poucas, menos ainda são as que comparam ou constatam numericamente a eficácia deste modelo em revelia de outros, sendo esses um dos principais motivos que levaram esse modelo a sofrer uma severa queda no decorrer do tempo \autocites[205]{yates_drug-free_2017-1}[57]{yates_rise_2017}. O que se encontra, apesar disso, são análises que apontam o efeito do uso de Religiosidade neste contexto como algo ambíguo, que pode gerar malefícios na mesma medida, ou que até perpassem, supostos benefícios existentes \autocites[256]{davis_religious_2014-1}[08]{williams_spiritual_2016}.

\hypertarget{financiamento-puxfablico}{%
\subsubsection{Financiamento Público}\label{financiamento-puxfablico}}

Uma última característica das CT's nacionais é que desde 2011 elas fazem parte da Rede de Assistência Psicossocial (\acrshort{RAPS}) o que as habilita a receberem por meio da Secretaria Nacional de Política sobre Drogas (\acrshort{SENAD}) financiamento público. Em relação ao número de vagas que são mantidas por meio de custeio de verba do SENAD é perceptível que grande parte delas é destinada as regiões Nordeste e Sudeste do país. O infográfico abaixo demonstra o número de CT's financiadas apenas pelo Governo Federal por estado, não contabilizando verbas do governo estadual ou municipal.
\begin{figure}[H]

{\centering \includegraphics[width=0.85\linewidth]{images/imagem04} 

}

\caption{Quantitativo de Comunidades Terapêuticas financiadas pelo SENAD.}\label{fig:imagem4}
\end{figure}
\bcenter

Fonte: SENAD, 2019
\ecenter

Ao se observar, no entanto, a distribuição de vagas por gênero é perceptível que não apenas as CT's não parecem se importar com a questão como o próprio sistema de financiamento parece apoiar esse tipo de prática. Mais de 80\% do financiamento do Governo Federal é destinado única e exclusivamente a homens, sejam eles adultos ou adolescentes. O gráfico abaixo demonstra a distribuição:
\begin{figure}[H]

{\centering \includegraphics[width=1\linewidth]{images/g3-1} 

}

\caption{Porcentagem de vagas oferecidas por sexo e faixa etárea nas Comunidades Terapêuticas brasileiras financiadas pela SENAD.}\label{fig:g3}
\end{figure}
\bcenter

Fonte: SENAD, 2019
\ecenter

A decisão de financiamento foi duramente criticada por instituições como os Conselhos Federais de Psicologia (CFP) e de Serviço Social (\acrshort{CFESS}). A principal pauta levantada por eles levantada é a de que esse ação sinalizava uma concordância do governo com o proselitismo promovido pelas CT's, que se caracteriza principalmente pela mistura de elementos terapêuticos com religiosos, gerando assim uma espécie ``fé medicamentada'' e no constante apoio da ``cura'' do sujeito numa conversão do interno ao credo pregado pela instituição \autocite[79]{targino_comunidades_2017-1}. Elementos que vão contra a lógica de Redução de Danos até então empregada.

\pagebreak

\hypertarget{comunidades-terapuxeauticas-e-poluxedticas-puxfablicas}{%
\section{Comunidades Terapêuticas e Políticas Públicas}\label{comunidades-terapuxeauticas-e-poluxedticas-puxfablicas}}

Como colocado no final da seção anterior, as CT's entram no rol de políticas públicas e começa a ser financiada em 2011. Desse ano em diante elas passaram a interagir com outras entidades, seja para agir em conjunto, seja para entrar em atrito. O resultado final deste longo percurso foi a transformação delas em referências no que diz respeito ao cuidado e tratamento do uso abusivo de drogas no Brasil.

Este capítulo se dedica a contar esta trajetória através de uma linha do tempo que se inicia em 2011, com a criação do Programa \emph{``Crack é possível vencer''} e se estende até 2020, com a promulgação da lei 13.840 que tornou os modelos baseados em abstinência, em outras palavras, Comunidades Terapêuticas, como oficiais. Para isso foram catalogadas e analisadas notícias publicadas online, artigos científicos e jornalísticos, promulgações, leis e relatórios disponibilizados por entidades ligadas ao Governo Federal e as CT's.

Contudo, antes de iniciar a exposição, é importante entender o que acontecia antes de 2011 e que possibilitou o contato inicial entre ambas as instâncias. O Brasil, assim como o resto do mundo, iniciou suas políticas de combate as drogas no começo do século XX a partir das resoluções da Primeira Conferência Internacional do Ópio \autocite{fiore_o_2012} por meio do Decreto-Lei n\textordmasculine 891, de 25 de novembro de 1938. O que foi precedido, muitos anos depois, em 1980, pela instituição do Sistema Nacional de Prevenção, Fiscalização e Repressão de Entorpecentes, que surge enquanto resposta ao surto de \acrshort{HIV} que acontecia no país dado o compartilhamento de seringas \autocite{andrade_reflexoes_2011-2}.

Em 2006 essas políticas se consolidam através do SISNAD e da lei n\textordmasculine 11.343/2006 que cria a primeira regulamentação para prevenção, atenção e reinserção social de indivíduos com histórico de uso abusivo de drogas e junto com ele outros programas e instituições como o Programa de Redução de Danos (\acrshort{PRD}), o \acrshort{SUPERA} (Sistema para Detecção do Uso Abusivo e Dependência de Substâncias Psicoativas: Encaminhamento, intervenção Breve, Reinserção Social e Acompanhamento) e órgãos exclusivamente dedicados à questão como a \acrshort{SENAD} (Secretaria Nacional de Políticas sobre Drogas), a Coordenação Nacional de Saúde Mental, Álcool e Outras Drogas e os \acrshort{CAPS AD}. Todas estas políticas, inclusive a própria lei de 2006, tinham como diretriz principal a Redução de Danos \autocite{andrade_reflexoes_2011-2}.

A partir de 2009, no entanto, uma segunda droga começa a ocupar o cenário nacional e a trazer grandes problemas: o Crack. Por mais que seu uso fosse registrado desde a década de 90 é neste período, graças ao crescimento exponencial, que ela atinge seu ápice, chegando a ser considerada na época um elemento ``pandêmico'', sendo inclusive tema de debate presidencial nas eleições de 2010 e parte da agenda política dos candidatos eleitos daquela época \autocite[198]{mattos_crack_2017}.

As medidas tomadas pelo governo eleito foram a criação do \acrshort{PEAD} (Plano Emergencial de Ampliação do Acesso ao Tratamento e à Prevenção em Álcool e outras Drogas) e do chamado \emph{``Plano Crack''}, que deu origem em 2011 ao programa \emph{``Crack, é possível vencer''}. É a partir deste ano que se pode traçar um divisor de águas entre as políticas empregadas até então pois, junto com esse plano, veio a introdução das Comunidades Terapêuticas, e por sua vez dos religiosos, na política de drogadição. Essas instituições representam o total oposto do que vinha sendo empregado dado que elas funcionam, em sua maioria, em modelos de Abstinência, que representam o inverso dos modelos de Redução de Danos. Somado a isso existem ainda outras questões como o constante uso de elementos religiosos e acusações de proselitismo que feriam os princípios de laicidade que os tratamentos indicados pelo governo mantinham. Todas essas questões geraram, e ainda geram, uma série de embates sobre qual modelo de cuidado era válido e deveria ser usado como padrão, além de questionar quais são os reais limites entre religião e política(s) no Brasil.

\hypertarget{primeiros-contatos}{%
\subsection{(2011-2014) Primeiros Contatos}\label{primeiros-contatos}}

Demarca-se como ponto de partida desta discussão, como colocado anteriormente, o ano de 2011. Foi neste período que se iniciou um estreitamento nas relações entre Governo Federal e Comunidades Terapêuticas, a partir da instituição do Programa \emph{``Crack é possível vencer''}, sendo estas últimas representadas pela Confederação Nacional de Comunidades Terapêuticas (\acrshort{CONFENACT}). Os seguintes trechos da matéria\footnote{Disponível em: \url{https://tinyurl.com/yagj3nf2}} publicada pelo site oficial do Governo Federal em junho de 2011 fornecem mais detalhes sobre o primeiro encontro destas duas entidades:
\begin{quoting}[rightmargin=0cm,leftmargin=4cm]
\begin{singlespace}
{\footnotesize 
  A presidenta Dilma Rousseff determinou a constituição de um grupo de trabalho sob liderança da ministra-chefe da Casa Civil, Gleisi Hoffmann, para preparar legislação que permita a inclusão de comunidades terapêuticas no atendimento aos cidadãos dependentes de substâncias químicas. Dilma se reuniu nesta quarta-feira (22), em Brasília, com representantes das entidades. De acordo com pastor Wellington Vieira, que preside a Federação de Comunidades Terapêuticas Evangélicas do Brasil (Feteb), existem no Brasil cerca de 3 mil comunidades que cuidam de aproximadamente 60 mil dependentes químicos. 'Estas entidades atendem atualmente cerca de 80\% das pessoas que estão em tratamento', disse o pastor Vieira. A titular da Secretaria Nacional de Políticas sobre Drogas (Senad), Paulina Duarte, disse que a reunião atende pedido da presidenta Dilma para conhecer o trabalho desenvolvido pelas comunidades. A partir deste momento, será estabelecido programa que vai incluir as entidades na rede pública para tratamento de dependentes químicos. Assim, estas comunidades passam a receber recursos públicos para prestar serviço.}
\end{singlespace}  
\end{quoting}
\begin{figure}[H]

{\centering \includegraphics[width=0.85\linewidth]{images/imagem05} 

}

\caption{Presidenta Dilma Rousseff posa para foto com representantes de Comunidades Terapêuticas, no Palácio do Planalto.}\label{fig:imagem5}
\end{figure}
\bcenter

Fonte: Biblioteca da Presidência da República\footnote{Disponível em: \url{https://tinyurl.com/ya5jjlpl}. Acesso em 06 out. 2019}
\ecenter

Esta reunião resultou na promessa de reconstrução da Resolução 101/2001 da Agência Nacional de Vigilância Sanitária (\acrshort{ANVISA}) permitindo assim aumentar o escopo de CT's que poderiam receber auxílio governamental\footnote{\url{https://rb.gy/qnwos5}}. Segundo Duarte:
\begin{quoting}[rightmargin=0cm,leftmargin=4cm]
\begin{singlespace}
{\footnotesize 
Temos uma resolução da Anvisa que estará sendo revista por esse grupo criado esta tarde pela presidenta. É uma resolução que coloca normas mínimas de funcionamento para comunidades terapêuticas. Essa resolução deve ser revista para que se possa atender a nova perspectiva de acolhimento das comunidades, como rede de apoio à rede pública de tratamento \cite{ultimato2011}.}
\end{singlespace}  
\end{quoting}
Ainda neste encontro nasce a \acrshort{RDC} 029/2011, uma Resolução que ameniza as exigências anteriores. A edição de agosto de 2011\footnote{\url{https://www.senado.gov.br/noticias/Jornal/emdiscussao/dependencia-quimica/sociedade-e-as-drogas/anvisa-ameniza-regras-para-comunidades-terapeuticas.aspx}} da Revista Em Discussão, publicada pelo Senado, traz uma matéria sobre a flexibilização. É destacado que:
\begin{quoting}[rightmargin=0cm,leftmargin=4cm]
\begin{singlespace}
{\footnotesize 
Foram retirados requisitos que, segundo Wellington Dias, eram os principais responsáveis pelas dificuldades das comunidades terapêuticas em obter financiamento público. De acordo com o senador, dois deles praticamente impediam que as comunidades se adequassem às regras: a proibição de que os internos fossem obrigados a participar das atividades religiosas e a exigência de que eles fossem atendidos, pelo menos uma vez por mês, por médico psiquiatra. \cite[p. 67]{discussao2011}}
\end{singlespace}  
\end{quoting}
Por mais que a criação desta RDC represente um grande passo na entrada das CT's para financiamento público, só é a partir de 2013 que podemos ver uma efetiva inserção destas nas políticas públicas de combate a drogadição, graças a posse de Vitore Maximiano como Secretário da SENAD. Sua entrada é marcada pela renúncia da secretária anterior, Paulina do Carmo Duarte. Em entrevista à Globo o ministro da Justiça deste período, José Eduardo Cardozo, afirmou que a saída de Duarte foi em parte motivada por pressão do Palácio do Planalto para a liberação de R\$ 130 milhões para Comunidades Terapêuticas\footnote{\url{https://www.anadep.org.br/wtk/pagina/materia?id=17172}} \footnote{\url{https://tinyurl.com/y9hx5bew}}. Segundo Cardozo, Maximiano ``{[}\ldots{]} tem uma sensibilidade grande para esse assunto e concordância total com o plano e o papel das Comunidades Terapêuticas''

As matérias publicadas no período anterior a saída de Duarte ajudam a entender que, de fato, havia uma tensão entre ela e os representantes das CT's. Neste outro trecho da revista ``Em Discussão'' podemos presenciar o que parece ser uma manifestação contrária dela em relação a metodologia empregada no tratamento ofertado pelas instituições, Duarte diz:
\begin{quoting}[rightmargin=0cm,leftmargin=4cm]
\begin{singlespace}
{\footnotesize
Não posso financiar, com dinheiro público, uma instituição católica que recebe para tratar um evangélico e o obriga a assistir a uma missa. Para essas comunidades, a nossa sugestão é de que seja seguida a metodologia, mas que se dê ao interno o direito de escolha \cite[p. 66]{discussao2011}.}
\end{singlespace}  
\end{quoting}
Essa fala foi contraposta à de Adalberto Calmon Barbosa, diretor de Projetos da Fazenda da Esperança e um dos principais representantes da inserção das CT's nas políticas públicas de combate à drogadição, segundo ele:
\begin{quoting}[rightmargin=0cm,leftmargin=4cm]
\begin{singlespace}
{\footnotesize
A ideia não é converter quem quer que seja. Não ensinamos religião. Ensinamos respeito, amor, responsabilidade. Mas é preciso que o interno esteja com o grupo, que participe. Não podemos deixá-lo sozinho, ainda que ele queira”. A Fazenda da Esperança abriu mão de participar do edital por “não poder prescindir de sua filosofia e métodos \cite[p.66]{discussao2011}.}
\end{singlespace}  
\end{quoting}
A entrada de Maximiano é marcada por uma aproximação deste com representantes das CT's, assim como uma preocupação dele com auxiliar estas instituições, seja no repasse de verba\footnote{\url{https://tinyurl.com/ycpfxxt7}} ou na criação de aporte legal para estas instituições.
\begin{figure}[H]

{\centering \includegraphics[width=0.6\linewidth]{images/imagem06} 

}

\caption{Secretário Vitore Maximiano com o Dep. Federal Ronaldo Benedet, defensor do modelo das comunidades terapêuticas.}\label{fig:imagem6}
\end{figure}
\bcenter

Fonte: Blog Drogas e Direitos Humanos\footnote{Disponível em: drogasedireitoshumanos.org/2013/08/08/na-contramao-senad-mantem-acordo-com-comunidades-terapeuticas-secretario-afirma-que-esta-investindo-em-rede-alternativa-de-tratamento-fora-do-sus/. Acesso em 06 out. 2019}\\
\ecenter
\begin{figure}[H]

{\centering \includegraphics[width=0.6\linewidth]{images/imagem07} 

}

\caption{Secretário Vitore Maximiano com o Padre Haroldo Rohm, fundador da Fazenda da Esperança e membro ativo da CONFENACT.}\label{fig:imagem7}
\end{figure}
\bcenter

Fonte: Blog Drogas e Direitos Humanos\footnote{Disponível em: \url{https://tinyurl.com/ycpfxxt7}. Acesso em 06 out. 2019}
\ecenter

A CONFENACT, em postagem datada de 2014 e assinada por Egon Schluter, secretário da instituição, menciona as ações de Maximiano a favor da causa:
\begin{quoting}[rightmargin=0cm,leftmargin=4cm]
\begin{singlespace}
{\footnotesize
Célio Barbosa, presidente, fez um pequeno histórico da parceria construída junto ao Governo Federal, onde o apoio da presidente Dilma, tem sido um marco para o trabalho das CT’s. \textbf{Da mesma forma a SENAD, através do Dr. Vitore Maximiano, tem sido grande parceiro do segmento, colocando em prática o compromisso de apoio da presidência}. Adalberto Calmon, vice-presidente, destacou a necessidade da continuidade da parceria, com a construção de uma política de estado, com a integração da modalidade de CT’s de forma efetiva e concreta na política pública sobre drogas. A pauta principal da audiência foi a continuidade da parceria do Governo Federal com as Comunidades Terapêuticas, com o financiamento de vagas de acolhimento. \textbf{Segundo Ministro José Eduardo Cardoso e Dr. Vitore Maximiano, secretário nacional da SENAD, estão assegurados no orçamento de 2015 recursos para os atuais e novos contratos que serão firmados com as entidades. Foi destacado por estes, o incremento do orçamento da SENAD de 2015 para fazer frente a demanda no acolhimento em CT’s, bem como projetos na área da prevenção, rede de apoio de familiares de dependentes e reinserção social} \cite[grifo meu]{confenact2014}}
\end{singlespace}  
\end{quoting}
\begin{figure}[H]

{\centering \includegraphics[width=0.6\linewidth]{images/imagem08} 

}

\caption{Membros da CONFENACT em encontro com o Ministro da Justiça José Eduardo Cardoso e o Secretário Vitore Maximiano.}\label{fig:imagem8}
\end{figure}
\bcenter

Fonte: Sítio da CONFENACT\footnote{Disponível em: www.confenact.org.br/?p=162, acesso em 06 out. 2019}
\ecenter

Para além de Maximiano, outro ator de grande importância para a causa é o deputado Eros Biodini (PTB/MG), que foi fundador da Frente Parlamentar das Comunidades Terapêuticas e da Associação de Proteção e Assistência ao Condenado - APAC´s. Esta frente foi criada em 2011 pelo mesmo deputado e foi reinstalada em 2015, anos que marcam acontecimentos estratégicos na relação entre apoiadores das CT's e o Governo Federal. A Frente conta com 198 deputados e 23 senadores das mais diversas siglas, como mostra o gráfico abaixo.
\begin{figure}[H]

{\centering \includegraphics[width=1\linewidth]{images/g4-1} 

}

\caption{Frequência de participantes (por partido) da Frente Parlamentar das CT's e APAC’S.}\label{fig:g4}
\end{figure}
\bcenter

Fonte: Câmara dos Deputados\footnote{Disponível em: \url{https://www.camara.leg.br/internet/deputado/frenteDetalhe.asp?id=385}}
\ecenter
\begin{figure}[H]

{\centering \includegraphics[width=0.6\linewidth]{images/imagem09} 

}

\caption{Frente Parlamentar no Congresso Nacional de Apoio as Comunidades Terapêuticas e APAC´S.}\label{fig:imagem9}
\end{figure}
\bcenter

Fonte: Sítio da Cruz Azul\footnote{Disponível em: www.cruzazul.org.br/novidade/93/criada-frente-parlamentar-no-congresso-nacional-de-apoio-as-comunidades-terapeuticas-e-apac\%C2\%B4s. Acesso em 06 out. 2019}
\ecenter

Um grande feito da gestão de Maximiano foi a construção do ``Marco Regulatório das Comunidades Terapêuticas'' a partir da resolução CONAD N\textordmasculine 01/2015. A partir deste Marco as CT's foram regulamentadas e reconhecidas enquanto parte do Sistema Nacional de Políticas Públicas sobre Drogas (SISNAD)\footnote{\url{http://www.politicasobredrogas.pr.gov.br/arquivos/File/CONAD_01_2015.pdf}}. A CONFENACT celebrou a instauração dele na seguinte postagem em seu blog oficial:
\begin{quoting}[rightmargin=0cm,leftmargin=4cm]
\begin{singlespace}
{\footnotesize
O dia 06 de maio de 2015 entrará para a história do movimento das Comunidades Terapêuticas do Brasil: FOI APROVADO PELO CONAD – Conselho Nacional de Políticas sobre Drogas – o MARCO REGULATÓRIO DAS COMUNIDADES TERAPÊUTICAS DO BRASIL. Diversas reuniões e contatos foram realizadas neste ano e no ano passado em Brasília, junto a SENAD e CONAD no sentido de construir e consolidar o Marco Regulatório. Todas as ações e iniciativa em prol da construção do mesmo, bem como aprovação partiram da Confederação Nacional de Comunidades Terapêuticas – CONFENACT em parceria com a SENAD, que tem a frente o Secretário Nacional da SENAD, Dr. Vitore André Zílio Maximiano. A CONFENACT, integrada pelas Federações (FETEB, FENNOCT, CRUZ AZUL NO BRASIL, FEBRACT e FNCTC) participou ativamente de todas as ações, reuniões, contatos e discussões em Brasília em prol, da aprovação do Marco Regulatório, e continua na luta pelo reconhecimento e valorização das Comunidades Terapêuticas no Brasil \cite{confenact2015}}
\end{singlespace}
\end{quoting}
\hypertarget{impasses}{%
\subsection{(2014-2016) Impasses}\label{impasses}}

Este Marco, no entanto, foi revogado em agosto de 2016 pela Justiça Federal do Estado de SP. Tal revogação é resultado de uma série de ações provenientes de instituição que se colocaram contra a entrada das Comunidades Terapêuticas no SISNAD e em seu financiamento público. Entidades como Conselho Federal de Psicologia (CFP), Rede Nacional Internúcleos da Luta Antimanicomial (RENILA), Procuradoria Federal dos Direitos do Cidadão (PFDC) emitiram seus posicionamentos oficiais sobre esta questão. Seguem trechos destes abaixo:
\begin{quoting}[rightmargin=0cm,leftmargin=4cm]
\begin{singlespace}
{\footnotesize 
A proliferação maciça das Comunidades Terapêuticas, atualmente parece indicar insuficiente expansão, organização e capacitação das redes de saúde e assistência social para o cuidado de pessoas que usam drogas. Neste sentido, justifica-se um posicionamento contrário ao financiamento público das Comunidades Terapêuticas, em defesa veemente do necessário aumento de recursos para investimento na RAPS. \cite[p.03]{cfp2014}}
\end{singlespace}
\end{quoting}
\begin{quoting}[rightmargin=0cm,leftmargin=4cm]
\begin{singlespace}
{\footnotesize
Há muito, a RENILA , em coro com várias outras organizações de defesa dos direitos humanos, da saúde pública e com movimentos antiproibicionistas, vem denunciando a profunda indiferença do governo com as deliberações expressas dos movimentos sociais no país a não participação das Comunidades Terapêuticas no âmbito dos serviços públicos de atenção às populações usuárias de drogas. Em contrário ao preceito fundamental da participação social no âmbito das políticas de saúde e da legitimação democrática das políticas públicas promovidas pelo Estado, o governo ignora a disposição contrária ao financiamento público das Comunidades Terapêuticas e sua inclusão na rede de serviços do Sistema Único de Saúde (SUS), conforme disposto na IV Conferência Nacional de Saúde Mental e na XIV Conferência Nacional de Saúde. Mais uma vez, no decorrer da criação do Grupo de Trabalho do Conselho Nacional de Políticas sobre Drogas (CONAD), voltado à criação da Resolução de Regulamentação das Comunidades Terapêuticas, observamos a repetição de um processo apenas formalmente democrático, de construção envolvendo atores restritos, com prazos apertados e a baixa participação de segmentos interessados na participação da construção das proposições que resultaram na minuta de regulamentação agora em consulta pública: notadamente, os usuários e destinatários da política de acolhimento e internações nas Comunidades Terapêuticas foram os ilustres ausentes do processo de construção das disposições e parâmetros para definição de quais são as entidades financiadas, quais são os critérios para a prestação dos serviços, quais são os direitos a serem observados e garantidos nessas instituições \cite[p. 01]{renila2014}.}
\end{singlespace}
\end{quoting}
Dentro do escopo de críticas tecidas às atitudes da CONAD, se destacam a vinculação das CT's dentro da SISNAD enquanto órgãos apartados do Sistema Único de Saúde (\acrshort{SUS}), a questão manicomial e o proselitismo religioso.

Como foi trabalhado no tópico anterior, a RDC 29 introduziu uma série de flexibilizações no trato político das CT's, assim como deu a elas mais liberdade de ação. Um dos grandes problemas para a entrada destas no circuito do financiamento público eram as restrições que lhe assemelhavam ao SUS, isso pode ser destacado no seguinte trecho de uma das reportagens da ``Em Discussão'' sobre a RDC:
\begin{quoting}[rightmargin=0cm,leftmargin=4cm]
\begin{singlespace}
{\footnotesize
Frei Hans Heinrich Stapel, fundador da Fazenda da Esperança, rede católica de comunidades com 52 unidades no Brasil, explica: ‘Eu rejeitei. Sabem por quê? Porque não entendem a comunidade terapêutica.\textbf{Querem fazer de nós um hospital, o que não somos} \cite[p. 66, grifo meu]{discussao2011}}
\end{singlespace}
\end{quoting}
A solução encontrada pela RDC para proporcionar essa flexibilização era tornar as CT's uma espécie de \emph{apoio} ao SUS, não uma parte integrante dele, como é possível nas considerações iniciais do documento:
\begin{quoting}[rightmargin=0cm,leftmargin=4cm]
\begin{singlespace}
{\footnotesize
CONSIDERANDO que as entidades que realizam o acolhimento de pessoas com problemas associados ao uso nocivo ou dependência de substância psicoativa não são estabelecimentos de saúde, mas de interesse e **apoio** das políticas públicas de cuidados, atenção, tratamento, proteção, promoção e reinserção social; \cite{brasil2015}}
\end{singlespace}
\end{quoting}
A Procuradoria Federal dos Direitos do Cidadão (PFDC) em fevereiro de 2015 enviou um ofício\footnote{pfdc.pgr.mpf.mp.br/temas-de-atuacao/saude-mental/atuacao-do-mpf/oficio-125-2015} ao então presidente da CONAD, José Eduardo Cardozo, no qual destaca a necessidade das CT's serem avaliadas e reguladas pelo próprio SUS, dada a natureza de suas ações, como é destacado pelos pontos 6 e 7 do documento:
\begin{quoting}[rightmargin=0cm,leftmargin=4cm]
\begin{singlespace}
{\footnotesize
6.  As denominadas comunidades terapêuticas, com ou sem fim lucrativo, estão no setor privado da economia. O SUS vale-se do setor privado da economia, de forma suplementar, para cumprir seu dever constitucional e legal, mediante convênios e contratos. Assim é o SUS que deve, cumprindo seu dever de gerir a Saúde Pública, regular a forma de prestação de serviços daquelas entidades privadas (comunidades terapêuticas ou não) de acordo com a planificação da prestação das ações e serviços públicos de saúde que está a seu cargo.
7.  Nesse sentido, a Portaria n\textordmasculine 3.088, de 23 de dezembro de 2011, do Ministério da Saúde, (Rede de Atenção Psicossocial-RAPS, artigo 5\textordmasculine, IV) define “Serviços de Atenção em Regime Residencial, entre os quais Comunidades Terapêuticas”. \textbf{Entende-se assim que as Comunidades Terapêuticas deveriam ser tratadas, quando for o caso, como equipamentos do SUS, e não como “apoio”, como está sendo disposto na minuta} \cite[p. 02, grifo meu]{pfdc2015}}
\end{singlespace}
\end{quoting}
O Conselho Federal de Serviço Social (CEFSS) também se manifestou sobre a situação, se opondo ao Marco e ao financiamento. Em seu posicionamento oficial\footnote{Disponível em www.cfess.org.br/arquivos/comunidade-terapeutica-2014timbradocfess.pdf} eles apontam várias incoerências presentes no documento:
\begin{quoting}[rightmargin=0cm,leftmargin=4cm]
\begin{singlespace}
{\footnotesize
A minuta está permeada de contradições, pois, se no ‘considerando’ acima citado, há uma clara e inequívoca ‘tentativa’ de caracterizar as “Comunidades Terapêuticas” como local de mero acolhimento, já em seu artigo 4\textordmasculine é determinada uma exigência própria e específica de serviços típicos de saúde. De acordo com a proposta: 
Art. 4\textordmasculine A instalação e o funcionamento de entidades que promovem o acolhimento de pessoas com problemas decorrentes do abuso ou dependência de substância psicoativa, denominadas ou não de comunidades terapêuticas, ficam condicionados à concessão de alvará sanitário ou outro instrumento congênere de acordo com a legislação sanitária específica aplicável a essas entidades. 
Todos os serviços que prestam atendimento à saúde da população requerem alvará sanitário, como objetivo de assegurar que disponha de condições mínimas para atendimento e permanência daqueles que recorrem aos serviços. Apesar de a própria resolução não caracterizar as “comunidades terapêuticas” como serviços de saúde, estas se propõem a realizar institucionalização/acolhimento e, portanto, não caberia a previsão de uma legislação sanitária específica para estas entidades \cite[p.03]{cfess2014}.}
\end{singlespace}
\end{quoting}
Por fim o Núcleo Especializado de Cidadania e Direitos Humanos (NECDH) do estado de São Paulo também se posiciona contra o Marco, alegando que a categoria \emph{apoio} vai contra uma série de premissas da saúde básica, para ele:
\begin{quoting}[rightmargin=0cm,leftmargin=4cm]
\begin{singlespace}
{\footnotesize
(...) o que não podemos aceitar é que a caracterização como equipamentos de ‘apoio aos sistemas de saúde e assistência social’ retire dessas instituições a responsabilidade de atuar segundo os parâmetros normativos dessas políticas públicas, e até mesmo de forma contrária às suas diretrizes, o que implica em não prestar contas perante os órgãos de fiscalização competentes na exata medida da complexidade do atendimento a que se propõem: um atendimento de internação na qual se retira o usuário de drogas de seu convívio, restringindo-se um dos direitos mais caros à humanidade, o direito à liberdade, com o fim de ‘tratamento’. Somente a título de comparação, equipamentos do Sistema Único de Assistência Social (SUAS) que acolhem pessoas nos moldes descritos acima, tais como serviços de acolhimento institucional (tanto para crianças e adolescentes quanto para adultos) são considerados pela política pública como sendo de ‘alta complexidade’ \cite[p.03]{necdh2014}.}
\end{singlespace}
\end{quoting}
A questão manicomial é outro ponto defendido pelos que se posicionam contra as CT's. A Associação Brasileira de Saúde Mental (ABRASME), em sua nota sobre o Marco, deixa clara a associação entre as instituições e o modelo manicomial, segundo a ABRASME:
\begin{quoting}[rightmargin=0cm,leftmargin=4cm]
\begin{singlespace}
{\footnotesize
As comunidades terapêuticas vão de encontro com a Lei 10.216/2001 por ter como dispositivo central o isolamento social e a internação, além de ser um equipamento privado, que tem em sua maioria uma fundamentação de cunho religioso. No entanto, sejam de fundamentação religiosa ou médica, as CT´s têm sido inspiradas num modelo de internação compulsória e violação dos direitos das pessoas em tratamento. Essa situação de financiar com recursos públicos o aumento e a sustentabilidade econômica das CT´s, não só é uma afronta a Lei 10.216/2001 e os anos de construção da Reforma Psiquiátrica brasileira, como também, ao caráter laico do Estado brasileiro. \cite{abrasme2014}.}
\end{singlespace}
\end{quoting}
O CFP, por sua vez, focou uma parte do seu posicionamento na questão religiosa das CT's. Para o Conselho, o proselitismo presente nestas instituições fere os direitos de liberdade individual, como é destacado no último dos 12 pontos abordados em seu posicionamento:
\begin{quoting}[rightmargin=0cm,leftmargin=4cm]
\begin{singlespace}
{\footnotesize
O Conselho Federal de Psicologia, em acordo com as deliberações do CNP, reafirma a defesa da laicidade do Estado e das políticas públicas, bem como, no âmbito da profissão e da promoção dos Direitos Humanos, posicionando-se criticamente em relação ao fundamentalismo religioso ou moral e garantindo o exercício da Psicologia calcado em seus princípios éticos, técnicos e científicos \cite[p. 04]{cfp2014}.}
\end{singlespace}
\end{quoting}
Uma última alegação do posicionamento da CFP que deve ser destacada é a de seu compromisso com os meios de tratamento baseados na ciência e nas pesquisas científicas, assim como uma defesa da Redução de Danos, à revelia de modelo baseado em abstinência como o das CT's:
\begin{quoting}[rightmargin=0cm,leftmargin=4cm]
\begin{singlespace}
{\footnotesize
Os tratamentos para o uso abusivo ou dependente de substâncias psicoativas devem ser \textbf{qualificados, sistemáticos e baseados em evidências científicas}, a exemplo daqueles desenvolvidos para outros problemas crônicos de saúde.
A referida implementação de serviços de tratamento de dependências baseados em evidências científicas, deve pautar-se, sobretudo, na compreensão de que o desenvolvimento do transtorno é o resultado de uma complexa interação multifatorial entre a exposição repetida a drogas e fatores biológicos e ambientais. \textbf{Posto isto, considera-se que iniciativas para tentar tratar e ou prevenir o uso de drogas por meio de sanções penais são ineficazes, por não levarem em conta as mudanças neurológicas provocadas em regiões do cérebro envolvidas no processo de motivação}. É necessário que uma política efetiva e eficaz considere a Redução de Danos como diretriz no cuidado às pessoas que usam drogas, pautadas na autonomia, no protagonismo \cite[pp. 03-04, grifo meu]{cfp2014}.}
\end{singlespace}
\end{quoting}
Atuações de ordem mais local também podem ser destacadas, como por exemplo a publicação, em 2016, do dossiê intitulado \emph{``Relatório de inspeção de comunidades terapêuticas para usuárias(os) de drogas no estado de São Paulo: Mapeamento das violações de direitos humanos''} no qual o Conselho Regional de Psicologia de São Paulo expõe uma série de relatos e dados sobre maus tratos e violência física/mental praticadas por algumas CT's no Estado, destes podemos destacar:
\begin{quoting}[rightmargin=0cm,leftmargin=4cm]
\begin{singlespace}
{\footnotesize

* Restrição ao uso de telefone e monitoramento de ligações telefônicas

* Monitoramento de correspondências

* Monitoramento das saídas

* Monitoramento das visitas

*   Restrições/rompimento de vínculos familiares e sociais

*   Restrição da liberdade dos usuários e características asilares

*   Obrigatoriedade em participar de atividades na instituição

*   Laborterapia obrigatória

*   Desrespeito à escolha ou ausência de credo;

*   Obrigatoriedade em participar de atividades de espiritualidade e/ou de atividades voltadas à crença religiosa determinada;

*   Adultos e adolescentes residindo no mesmo espaço;

*   Sem acesso à educação (para adultos);

*   Adolescentes sem acesso à educação;

*   Quando havia intercorrência médica ou odontológica simples, a saída não era autorizada e eles não recebiam atendimento adequado;

*   Isolamento, segregação e confinamento em quarto;

*   Cobrança diferenciada para usuários com comorbidades;

*   Confinamento em quarto/sala de contenção;

*   Contenção física e medicamentosa;

*   Monitoramento constante, ameaças e chantagens;

*   Violência física, agressões verbais, maus-tratos, humilhações, constrangimento;

*   Penalidades/punições (chamadas, por exemplo, de “educativas”, “disciplina”, “medidas reeducativas”, “processo disciplinar”, “advertências”);

*   Resgate forçado;

*   Vistoria/ revista vexatória (“baculejo”);

*   (Relato de) Violência sexual;

*   Interrupção de telefonemas se/quando a pessoa mencionasse desejo de sair do local (interromper a internação) \cite[pp. 25-26]{crp2016}.}

\end{singlespace}
\end{quoting}
Por mais que denúncias sobre maus tratos e tortura nestes ambientes já fossem feitas antes de 2015\footnote{Disponível em: \url{https://tinyurl.com/ycwzl62t}}, é neste ano que elas vão insurgir na mídia em maior grau. Um exemplo disto é uma matéria escrita pela Carta Capital intitulada ``Comunidades Terapêuticas, política e religiosos = bons negócios''\footnote{Disponível em: www.cartacapital.com.br/sociedade/comunidades-terapeuticas-politica-e-religiosos-bons-negocios-9323.html}. Nela é possível ler, em tom dramático e jocoso, duras críticas ao modelo da CT, seguem trechos abaixo:
\begin{quoting}[rightmargin=0cm,leftmargin=4cm]
\begin{singlespace}
{\footnotesize
'Temos mais de 50 pessoas em cárcere privado onde a família não quer em casa e os colocaram confinados como coelhos e estão todos amontoados. Ainda como o pior de tudo somos chamados de lixo, [sic] esquesitos, [sic] mocorongos, bicha, vagabundos, lesados e outros [sic] diseres pelo dono, que acaba de vez com nós. Sei que a nossa família paga um bom preço para nós estarmos aqui internado, necessitamos de ajuda, mas do jeito que vai não dá.’
O e-mail, com o título “Ajuda Pelo amor de Deus”, triste e ironicamente, chegou em minha caixa postal na mesma semana em que as CT’s foram regulamentadas no Conselho Nacional de Políticas sobre Drogas do Ministério da Justiça. Venceu o lobby das CT’s e da Frente Parlamentar das CT’s, contra os que defendem uma moderna abordagem de saúde pública.
O que nasceu como tentativa genuína de “solução” para o “vício nas drogas”, quando foram criadas há algumas décadas, tornou-se a versão século 21 dos manicômios. Virou também um negócio lucrativo que envolve religiosos e políticos - alguns desavisados e bem intencionados, grande parte deles picaretas.\cite{carta2015}.}
\end{singlespace}
\end{quoting}
\hypertarget{o-retorno}{%
\subsection{(2016-2018) O Retorno}\label{o-retorno}}

Ao mesmo tempo que essas instituições e atores defendem seus posicionamentos e ideais, a CONFENACT também se defendia das acusações, frisando em sua argumentação a forte atuação das CT's no combate a drogadição e a pureza das suas intenções. O argumento central da instituição é a de que instituições irregulares, como as encontradas pela CFP, eram uma pequena parte do todo, chamadas por eles de ``mau exemplo'', e que não serviriam para categorizar e rotular todas as instituições deste cunho. Isso pode ser visto, por exemplo, neste trecho retirado de seu pronunciamento oficial intitulado \emph{Comunidade Terapêutica regulamentada é uma segurança para a família do acolhido: vamos esclarecer à população o que é o certo e o que é o errado}:
\begin{quoting}[rightmargin=0cm,leftmargin=4cm]
\begin{singlespace}
{\footnotesize      
Segundo o último senso realizado em agosto/2012, existem no Brasil 1.847 Comunidades Terapêuticas – CT’s – que querem trabalhar de forma correta e que precisam do MARCO REGULATÓRIO, para continuar qualificando e ampliando os serviços prestados às pessoas afetadas pela dependência química. Sabemos que há um número maior de entidades no Brasil, tendo em vista que muitas atuam na informalidade, sem os devidos registros legais, estrutura física e equipe de trabalho. Dentre este grande universo de entidades, há uma minoria, assim como acontece em outros serviços, que não atuam com uma metodologia adequada. Por desconhecimento, muitas famílias acabam caindo em mãos erradas e seus filhos continuam sofrendo. Infelizmente, esta minoria de entidades acaba recebendo a atenção da mídia e compromete o trabalho sério das entidades que atuam há muitos anos. \textbf{Os movimentos contrários ao trabalho das CT’s (segmentos do governo, alguns conselhos profissionais e outros movimentos sociais) não querem esta continuidade e usam os maus exemplos destas entidades para denegrir a imagem de todas as entidades sérias do Brasil.} O que nós queremos é mostrar o bom trabalho de quem atua de forma correta, com base na técnica e no amor, procurando deixar clara a diferença de quem não faz este trabalho e apenas usa o nome “Comunidade Terapêutica” como fachada para “caça níqueis” às famílias desesperadas \cite[p. 01, grifo meu]{confenact2016}.}
\end{singlespace}
\end{quoting}
Esta ``técnica'' e ``amor'' citados a pouco parecem ser elementos centrais da filosofia da CONFENACT. Isso se prova verdadeiro quando eles destacam estes elementos em seu próprio lema: \emph{Amor e Ciência a serviço da qualidade de vida}. Existem também em seus pronunciamentos e ações uma urgência em se afirmar enquanto algo científico, sério e de confiança.
\begin{figure}[H]

{\centering \includegraphics[width=0.6\linewidth]{images/imagem10} 

}

\caption{Logo da CONFENACT.}\label{fig:imagem10}
\end{figure}
\bcenter

Fonte: Sítio da CONFENACT\footnote{Disponível em: \url{http://www.confenact.org.br/}}
\ecenter

As descrições e posicionamentos da CONFENACT e de seus associados sempre destcam a conexão entre ciência e religião existente em seus tratamentos. No site da Cruz Azul, uma das afiliadas da instituição, eles destacam como método {[}\ldots{]} a fé no Deus Triúno, conforme o testemunho de toda Sagrada Escritura (Bíblia Judaico-Cristã), aliado ao conhecimento técnico, científico e empírico\footnote{\url{http://www.cruzazul.org.br/sobre}}. Não há uma preocupação em negar o caráter religioso da obra, mas sim em enfatizar que ele não é a única coisa presente.

Em março de 2018 o Marco Regulatório volta a vigorar. Em Audiência Pública a desembargadora Consuelo Yoshida decide por suspender a liminar de suspensão do Marco. Segundo Yoshida as CT's se configurariam enquanto equipamentos de acolhimento, e não de saúde. ``As CTs não se enquadram adequadamente em estabelecimentos de saúde voltados para a internação de pacientes portadores de transtornos da saúde física e mental, na medida em que elas buscam promover a saúde psicossocial das pessoas acolhidas''. A questão da categoria ``apoio'' é resolvida pela introdução do termo ``acolhida''. Tal vitória é grandemente celebrada pela CONFENACT. A postagem no site oficial datada do mesmo mês fornece mais detalhes sobre o acontecimento:
\begin{quoting}[rightmargin=0cm,leftmargin=4cm]
\begin{singlespace}
{\footnotesize
Após um longo período de suspensão (desde agosto de 2016) pela Justiça Federal do estado de São Paulo, a Resolução 01/2015 do CONAD que regulamenta as Comunidades Terapêuticas do Brasil (Marco Regulatório – MR), ESTÁ NOVAMENTE VIGENTE. A revogação da suspensão aconteceu na tarde desta terça-feira, em São Paulo, capital, após a AUDIÊNCIA PÚBLICA no Tribunal Regional Federal de SP (TRF3-SP), localizado na Avenida Paulista, 1842, SP/SP. A CONFENACT participou diretamente da defesa do Marco Regulatório, desde a sua edição em 2015, e depois de agosto de 2016 quando injustamente a Justiça Federal decidiu pela suspensão, a partir do pedido do MPF. Atuaram nos debates e discussões os advogados Egon Schlüter, presidente da CONFENACT, Adalberto Calmon Calmon, assessor jurídico da confederação, acompanhados por Pablo Kurlander, secretário da CONFENACT e advogada Eunice Melhado de Lima, da Fazenda da Esperança. E também pela defesa do MR, o MS, MJ, MDSA, que colocaram por parte do Governo Federal a necessidade da regulamentação das CT’s. Estiveram presentes também o CFP, CFSS e CNDH, que através do MPF entraram com a Ação Civil Pública pedindo a revogação do MR e o não financiamento de vagas pelo Governo Federal, que culminou na suspensão do MR em 2016. Após um debate de mais de 05 horas, o TRF3, e em especial, partir de um pedido da AGU/MJ, decidiu por “derrubar” a liminar que suspendia o MR. Ficou também decidido que daqui há três meses deverá ter uma nova audiência no Tribunal Regional, com a presença do Conselho Nacional de Saúde e Conselho Nacional de Direitos Humanos, para discutir alguns pontos do MR, em especial, mecanismos de fiscalização do financiamento público de vagas. Assim, nesses próximos três meses a CONFENACT buscará avançar no diálogo com o CNS e CNDH para discutir alguns pontos do MR, caso necessário eventual mudança \cite{confenact2018}.}

\end{singlespace}
\end{quoting}
Na notícia acima, a CONFENACT destaca a presença de outros atores que foram anteriormente citados e afirma que a suspensão do Marco é resultado de suas ações e posicionamentos, como foi anteriormente debatido.

Ainda em 2018, um mês após a reativação do Marco, acontece um aumento massivo do investimento federal em entidades que tratam da dependência química. 87,3 milhões foram destinados para tais instituições. Corroboraram para esse edital o ministro Gilberto Occhi (Saúde), o secretário executivo Gilson Libório (\acrshort{MJ}), o ministro Alberto Beltrame (\acrshort{MDS}), e o secretário nacional de Políticas sobre Drogas, Humberto Vianna. A matéria publicada pela Agência Brasil sobre destaca que:
\begin{quoting}[rightmargin=0cm,leftmargin=4cm]
\begin{singlespace}
{\footnotesize
O que nós temos é uma grande procura por esse método de acolhimento. A taxa de ocupação é de quase 100\%, o que fez então nós percebermos a necessidade de ampliar a rede para esse tipo de atenção”, justifica o médico Quirino Júnior, coordenador-geral de Saúde Mental, Álcool e Outras Drogas do Ministério da Saúde. Ele explica que o encaminhamento para este tipo de tratamento, além de voluntário por parte do paciente, vai requerer, obrigatoriamente, autorização médica. Segundo o ministro do Desenvolvimento Social, Alberto Beltrame, o tempo máximo de permanência do paciente não poderá ultrapassar 12 meses. “Em média, o paciente fica internado por quatro meses, mas o limite é até um ano. Os 87 milhões de reais terão capacidade de contratar em torno de 7 mil leitos. Mantendo uma lógica de rotatividade de três usuários por leito, em torno de 20 mil vagas serão oferecidas ao longo de um ano. Nós estamos quase dobrando a capacidade de contratação e a quantidade de pessoas atendidas [em relação a editais anteriores]”, aponta. O edital prevê que as comunidades terapêuticas possam apresentar diferentes metodologias de tratamento, incluindo laborterapia (que é trabalhar na manutenção do local), psicoterapia em grupo ou individual, programa dos 12 passos (criado inicialmente para tratamento do alcoolismo), atividades espirituais, ações pedagógicas e grupo operativo. \cite{agencia2018}.}
\end{singlespace}
\end{quoting}
Da quantia citada no edital, 40 milhões são do orçamento do Ministério da Saúde; 37 milhões, do Ministério da Justiça; e R\$ 10 milhões, do MDS. A CT contemplada receberá 1.172,28 por adulto acolhido; 1.596,44, por adolescente; e 1.528,02, para mães viciadas acompanhadas de bebês em fase lactante.

A CFP, por outro lado, continua fazendo denúncias sobre irregularidades encontradas em CT's. Em junho de 2018 foi publicado pelo Conselho, em parceria com o Mecanismo Nacional de Prevenção e Combate à Tortura (\acrshort{MNPCT}) e a Procuradoria Federal dos Direitos do Cidadão do Ministério Público Federal (\acrshort{PFDC}PFDC/MPF), o Relatório da Inspeção Nacional em Comunidades Terapêuticas. No documento é possível perceber que muitas das Comunidades pesquisadas restringem a liberdade religiosa dos internos \autocite[15]{cfp2018}, assim como há práticas que configuram como indícios de tortura \autocite[14]{cfp2018}. Das comunidades pesquisadas 64\% recebia financiamento público \autocite[18]{cfp2018}.

Em relação a este Relatório, a CONFENACT se pronunciou formalmente em um manifesto lançado no mesmo mês. Neste manifesto a instituição argumenta que o número de CT's estudadas não era o suficiente para fazer uma generalização. Mais uma vez o argumento da seletividade na escolha das Comunidades avaliadas é utilizado, como é possível ver no trecho abaixo:
\begin{quoting}[rightmargin=0cm,leftmargin=4cm]
\begin{singlespace}
{\footnotesize
Infelizmente o Relatório de Inspeção, dentro de um contexto de mais de 2.000 CTs no Brasil, de forma dirigida, com o objetivo de prejudicar todo um segmento, selecionou 28 entidades em sua maioria involuntárias, com problemas de violação de direitos humanos, entidades com fins lucrativos, para dar característica de uma pesquisa nacional, de uma avaliação do todo do segmento. Evidencia-se esta grande contradição, pelo próprio nome dado ao Relatório de Inspeção (Relatório da Inspeção Nacional em Comunidades Terapêuticas), com uma vistoria pontual em somente 0,08\% das entidades/unidades (CTs), e em somente 12 dos 27 estados do Brasil. Que ainda é reiterada, pelo registro no próprio texto, quando é citado no “Resumo Executivo”: “A sistematização das informações coletadas nos 28 estabelecimentos vistoriados busca, portanto, trazer um retrato do modo de atuação dessas instituições, permitindo um olhar geral, sem que se perca de vista as especificidades de cada local.” (Página 11). Ou seja, lança-se um olhar geral, generalizando-se 28 casos específicos para todo um segmento, sem mencionar informações específicas, números, mas com o uso da expressão vaga “grande parte” das entidades \cite[p. 02]{confenact2018b}.}
\end{singlespace}
\end{quoting}
Outra alegação da CONFENACT é a de que a CFP e outros movimentos contrários as CT's não ouvem os próprios psicólogos que atuam nessas instituições, muitas vezes por livre e espontânea vontade, e que esses ataques são de ordem ideológica.
\begin{quoting}[rightmargin=0cm,leftmargin=4cm]
\begin{singlespace}
{\footnotesize
Destaca-se, visto que o CFP lidera o movimento ideológico contra as comunidades terapêuticas, não considera os profissionais da psicologia que atuam nas entidades, não dando voz a estes, que nas conferências, seminários e fóruns, compartilham que não se sentem representados pelo conselho federal e pelos conselhos estaduais (regionais). Segundo pesquisa do IPEA há em média 1,8 psicólogos por CTs, em sua maioria empregados contratados, considerando a opção do voluntariado, bem como de outros profissionais, como assistentes sociais (1,3 por CT), o que denota a preocupação técnica e profissional no atendimento nas entidades, o que ajuda a afastar eventuais abusos ou condutas relacionados a violação dos direitos humanos. Vemos importante, que o CFP faça uma pesquisa com os profissionais psicólogos que atuam nas CTs, para que possam dar consistência ao levantamento de informações sobre o segmento. Situação muito similar acontece com o CFSS – Conselho Federal de Serviço Social e os estaduais/regionais, que tem uma atitude radical contra as CTs muito similar e até maior, onde a partir de um viés ideológico desrespeitam os valiosos profissionais que atuam nas CTs." \cite[p. 03-04]{confenact2018b}.}
\end{singlespace}
\end{quoting}
Ainda sobre isto acrescenta:
\begin{quoting}[rightmargin=0cm,leftmargin=4cm]
\begin{singlespace}
{\footnotesize
REPUDIAMOS A POSTURA DE DESCONSTRUÇÃO ADOTADA PELOS SIGNATÁRIOS DO RELATÓRIO DE INSPEÇÃO, selecionando entidades que prestam desserviços e também de outras, que atendem dentro dos preceitos legais, mas com desrespeito dos signatários deste infeliz documento para com as atividades terapêuticas desenvolvidas por estas, fazendo julgamentos dentro de um viés ideológico que não admite a existência das Comunidades Terapêuticas, em franco desrespeito à legislação \cite[p. 06]{confenact2018b}.}
\end{singlespace}
\end{quoting}
\hypertarget{o-uxe1pice}{%
\subsection{(2019 - 2020) O Ápice}\label{o-uxe1pice}}

Apesar das crescentes vitórias que o movimento vivenciava é em 2019 que acontece o evento que escreveria de vez o nome das CT's na história do país. É nesse ano que elas tomam o lugar de destaque e passam a atuar como carro chefe das Políticas Nacionais de Drogas (\acrshort{PNAD}).

Um dos primeiros acontecimentos que corroboraram diretamente para isso foi o início do mandato do presidente Jair Bolsonaro (PSL/Sem partido), candidato que desde o início da sua campanha se revelava enquanto político conservador e ávido apoiador de pautas liberais e religiosas (vide seu lema \emph{``Brasil acima de tudo, Deus acima de todos''}) que iam contra as ditas ``políticas ideológicas'' realizadas pelos governos anteriores. Em seu plano de governo\footnote{disponível neste \href{https://divulgacandcontas.tse.jus.br/candidaturas/oficial/2018/BR/BR/2022802018/280000614517/proposta_1534284632231.pdf}{link}} a questão das drogas é introduzida enquanto parte de uma problemática que não apenas é ligada diretamente à ascensão da esquerda como é alimentada pela mesma, citando como exemplo o chamado ``bolsa crack'', nome pejorativo do Programa ``De Braços Abertos'' criado por Fernando Haddad em 2014 enquanto prefeito de São Paulo\autocite{de2019programa}. Esse foco no elemento ``ideologia'' que é aí empregado é bastante similar ao da CONFENACT e afins ao falar das instituições que lhe contrapunham.

Ao tomar posse, Bolsonaro decide manter por perto alguns nomes que já lidavam com a questão do uso abusivo de drogas e trabalhavam em prol das CT's no ambiente federal, como por exemplo Osmar Terra que se tornou ministro da Cidadania e Quirino Cordeiro, atual chefe da SENAPRED (Secretaria Nacional de Cuidados e Prevenção às Drogas), mas não apenas isso como também insere novos personagens de respaldo como a advogada e pastora Damares Alves, atual ministra da Mulher, da Família e dos Direitos Humanos e que atua ativamente na luta contra a legalização do aborto, o feminismo e outras pautas progressistas. Esses três personagens são os responsáveis durante os anos iniciais deste governo de direcionarem parte de sua energia para a construção e aprovação de leis e orçamentos que favorecessem as Comunidades, Logo em março é assinado por eles um documento que aumentava significativamente o número de vagas financiadas pelo Governo Federal, que passou de 6.609 para 10.883. No evento Damares falou que: \emph{``Neste ato, Estado laico reconhece a importância das comunidades religiosas. Estamos construindo uma nova nação, isto é o retrato de um novo Brasil''}\footnote{Disponível em \url{https://rb.gy/8ibudi}}
\begin{figure}[H]

{\centering \includegraphics[width=0.8\linewidth]{images/imagem11} 

}

\caption{Damares Alves assinando documento que prevê aumento de vagas.}\label{fig:imagem11}
\end{figure}
\bcenter

Fonte: Sítio do Governo Federal\footnote{Disponível em \url{https://rb.gy/8ibudi}}
\ecenter
\begin{figure}[H]

{\centering \includegraphics[width=0.8\linewidth]{images/imagem12} 

}

\caption{Osmar Terra assinando documento que prevê aumento de vagas, Quirino ao fundo.}\label{fig:imagem12}
\end{figure}
\bcenter

Fonte: Sítio do Governo Federal\footnote{Disponível em \url{https://rb.gy/8ibudi}}
\ecenter

Ainda no começo deste novo governo, a \emph{Frente Parlamentar Mista em Defesa das Comunidades Terapêuticas e Apac's} é novamente ativada por Eros Biodini (PROS-MG). Assim como em 2015, o retorno da Frente, dessa vez com a ação conjunta dos ministros, é indicativo de mudanças nos rumos das Comunidades. O que de fato aconteceu em Junho a partir da alteração da Lei n.~11.343/2006, agora intitulada Lei 13.840, de 5 de junho de 2019. Das 14 mudanças operadas no documento três chamam a atenção. A primeira é não mencionar em nenhum ponto do documento o termo ``Redução de Danos'', o que aponta que este não é mais o norte das novas políticas públicas para uso abusivo de drogas. Pelo contrário, o texto salienta que os acolhimentos devem ser feitos em unidades ``que visam à abstinência'', em outras palavras, CT's.

O segundo ponto a ser destacado é a facilitação do processo de financiamento, dado que elas agora são o modelo que o país optou por empregar. Isso é perceptível, por exemplo, no documento mencionado nos parágrafos anteriores que ampliava o número de vagas financiadas, sendo destinado a isso o montante de R\$ 150 milhões\footnote{Disponível em: \url{https://rb.gy/wl747u}}. Esse não somente é o maior montante fornecido a elas até então como também é próximo ao valor que o governo investia nos próprios CAPS.

O último ponto a ser destacado é a possibilidade de internações involuntárias que podem ser feitas ``(\ldots) A pedido de familiar ou do responsável legal ou, na absoluta falta deste, de servidor público da área de saúde, da assistência social ou dos órgãos públicos integrantes do SISNAD, com exceção de servidores da área de segurança pública, que constate a existência de motivos que justifiquem a medida.'' \autocite{brasil_lei_2019-1}.

O presidente comentou a instauração da lei em suas redes sociais dando ênfase a este terceiro ponto. No \emph{post} ele alega que ``o dependente não é livre, é um escravo da droga'' o que, aparentemente, seria o suficiente para autorizar internações contra a sua vontade. Vale lembrar ainda que as próprias CT's são consideradas órgãos públicos que integram a SISNAD logo passíveis, em última instância, de autorizar internações desse tipo.
\begin{figure}[H]

{\centering \includegraphics[width=0.8\linewidth]{images/imagem13} 

}

\caption{Tweet da Conta Oficial do Presidente falando sobre a nova lei.}\label{fig:imagem13}
\end{figure}
\bcenter

Fonte: Twitter\footnote{Disponível em: \url{https://twitter.com/jairbolsonaro/status/1137125025226612736}}
\ecenter

Para os defensores da causa das CT's as alterações foram mais que bem vindas. A antiga lei, na visão deles, não era apenas ineficaz como também maléfica. Em artigo escrito por Quirino Cordeiro para o Observatório Brasileiros de Informações sobre Droga (\acrshort{OBID}) são elencadas várias ``desassistências'' que teriam feito o governo optar por este novo modelo, dentre elas se destacam o desvio e mal uso da verba pública, o que gerou grandes prejuízos, subnotificações de atendimento, ausência de equipes para atuação no \acrshort{CAPS}, o fechamento irresponsável de hospitais psiquiátricos e manutenção de ``cursos com baixo conteúdo técnico e alto teor de doutrinação político-ideológica''. Ainda segundo ele existia uma marginalização das CT's gerada pela política Antimanicomial e a Redução de Danos, como é possível ler no seguinte trecho\footnote{Disponível em: \url{https://rb.gy/ez2uud}}:
\begin{quoting}[rightmargin=0cm,leftmargin=4cm]
\begin{singlespace}
{\footnotesize
As Comunidades Terapêuticas eram negligenciadas e marginalizadas pelo Ministério da Saúde. Em decorrência dos graves problemas relatados acima, os péssimos Indicadores de Resultados da antiga Política Nacional de Saúde Mental eram patentes: crescimento das taxas de suicídio nos últimos 15 anos; aumento de indivíduos com transtornos mentais graves em situação de rua; encarceramento de pacientes com transtornos mentais graves; aumento da mortalidade de tais pacientes; superlotação de Serviços de Emergência com pacientes aguardando vaga para internação; aumento do uso de drogas e dependência química no país; crescimento e expansão das Cracolândias; aumento de pacientes afastados pela Previdência Social, principalmente por depressão e dependência ao crack.}
\end{singlespace}
\end{quoting}
A discordância de Quirino com as políticas aplicadas no governo anterior é visível ainda em trechos de palestras por ele presididas sobre a Nova lei. Em evento promovido em Novembro pela Federação de Comunidades Terapêuticas Evangélicas do Brasil, ele afirma que:
\begin{quoting}[rightmargin=0cm,leftmargin=4cm]
\begin{singlespace}
{\footnotesize
A luta antimanicomial trouxe conceitos completamente equivocados na organização de ofertas cuidados das pessoas com transtornos mentais e dependência química no Brasil. A luta vinha proclamando a necessidade de fechamento de serviços, por exemplo, de internação em hospitais psiquiátricos, comunidades terapêuticas e fechamento de ambulatórios de saúde mental no País. Isso levou a uma grande desassistência \cite{folha2019}.}
\end{singlespace}
\end{quoting}
Tal aversão também costuma aparecer nas falas de outras pessoas envolvidas com as CT's, como por exemplo Pablo Kurlander, atual gestor geral da Federação Brasileira de Comunidades Terapêuticas (FEBRACT). Em palestra sobre a atuação das Comunidades no Brasil feita no III Seminário Nordeste do Programa AMOR-EXIGENTE ele comenta:
\begin{quoting}[rightmargin=0cm,leftmargin=4cm]
\begin{singlespace}
{\footnotesize
O grupo que defende a Luta Antimanicomial diz que o problema da rede no Brasil é que o dinheiro é desviado para as Comunidades Terapêuticas. Porém, o governo começou a financiar as CTs de forma bem modesta apenas em 2013, 26 anos depois da Reforma Psiquiátrica. Em 1990, 3 anos depois, começou uma redução de leitos em hospitais psiquiátricos numa cruzada puramente política e sem ter para onde encaminhar os pacientes. Nesta época, o trabalho das CTs ainda era muito tímido. O Padre Haroldo há apenas 9 anos tinha iniciado o trabalho da APOT e alguns serviços religiosos tinham surgido há uma década em Goiás e no Paraná. Neste momento, então, começa a se desvirtuar o conceito de tratamento das Comunidades Terapêuticas no Brasil. Todas as distorções que vieram em seguida e que são criticadas duramente foram criadas justamente pela má intervenção deste grupo. Temos, então, uma divisão em dois segmentos: os grupos religiosos que começam a atender as pessoas carentes que deveriam ir para os hospitais e o movimento de internação involuntária que funciona, na maioria das vezes, de forma ilegal e irregular para aqueles que podem pagar pela internação. No meio desse caos e na tentativa de resgatar aquilo que as CTs de fato eram, em 1990 Padre Haroldo Rahm funda a FEBRACT \cite{febract2019}} 
\end{singlespace}
\end{quoting}
Para além de mencionar novamente o elemento ``ideologia'', comum no discurso institucional desde 2018, Kurlander também traz à tona outro ponto que CONFENACT também defendia em seus posicionamentos contra a \acrshort{CFP}: A \emph{``Real Comunidade Terapêutica''}. Dessa forma, ele coloca assistencialismo religioso e internação involuntária ilegal como elementos que pertencem a uma entidade outra que não as reais Comunidades, criando assim uma narrativa institucional própria e limpa, totalmente desvinculada desses inconvenientes do passado\footnote{Também conhecida como \emph{Falácia do Verdadeiro Escocês}, esse tipo de técnica argumentativa separa e deslegitima ações que prejudiquem o ponto de vista do interlocutor, sendo seu uso extremamente comum no que diz respeito a questões que envolvem pautas religiosas. É possível encontrar suas aplicações em várias situações, tais como ``Homens bomba não fazem parte do `Verdedeiro Islã'\,'' \autocite[171]{orsi_is_2003} ou ``O Nazismo não tem relação com o `Verdadeiro Cristianismo'\,'' \autocite[258]{avalos_fighting_2005}}.

A adoção dessa narrativa também marca uma reviravolta na forma como as CT's se defendem. Se elas antes assumiam quase que um papel passivo, alegando apenas o uso de má fé nas acusações, agora eles tomam um passo à frente e somam a essas alegações anteriores um suposto ``ataque ideológico'' a suas ações, que eram direcionadas em primeiro lugar em uma tentativa de cobrir um ``vácuo de desassistência'' gerado por esta mesma ideologia que lhe tenta condenar. Não existe nessa lógica provas factuais do ataque ou do viés ideológico, o mais próximo disso foi uma contestação feita pela \acrshort{FEBRACT} em uma das notas de esclarecimento\footnote{Disponível em: febract.org.br/portal/wp-content/uploads/2020/06/PRONUNCIAMENTO-FEBRACT-SOBRE-RELATORIO-NACIONAL-DE-CTS-CFP-MPF-MPCT.pdf} sobre problemas de amostragem e viés feitos pelo CFP invalidarem as afirmações por ele feitas. Uma parte do que foi ali argumentado viria a compor um capítulo da tese do próprio Kurlander \autocite[41-43]{perrone_fatores_2019} que, curiosamente, empregou erros similares aos que alega existir na pesquisa do CFP como, por exemplo, estabelecer amostra por conveniência e não oferecer testes confirmatórios das análises por ele feitas.\footnote{É problemático que uma pesquisa faça uso de Análise de Regressão e não disponha, no mínimo, o \(R^2\) ajustado, para futuras verificações de adequação dos dados a linha.}

A CFP, por outro lado, ao se posicionar sobre a lei destaca o quão problemático soa a internação involuntária e como o processo de declaração por escrito, estabelecido tanto para internações voluntária quanto pedido de desligamento, é discriminador, dado que não leva em consideração analfabetos (que em sua maioria estão nas camadas sociais mais baixas) e pessoas com distúrbios ou deficiências que dificultam o ato\footnote{Disponível em: www.encurtador.com.br/inv37}. Ainda são acrescentadas as críticas à forma como a lei foi criada sem debate com outras as entidades envolvidas no campo. Como é possível ler nos trechos abaixo:
\begin{quoting}[rightmargin=0cm,leftmargin=4cm]
\begin{singlespace}
{\footnotesize
O texto da lei deveria definir que esta declaração poderia ser por escrito ou feita pela forma e pelo meio de comunicação que seja mais acessível, expressando de maneira inequívoca o desejo da pessoa”, afirma Biancha Angelucci, pesquisadora consultada. “Tal imprecisão da lei favoreceria discriminações sobre populações já vulnerabilizadas, em situação de rua, por exemplo”. Segundo o presidente do Conselho Federal de Psicologia (CFP), Rogério Giannini, \textbf{as mudanças foram feitas sem o devido processo de discussão, sem passar pelos Conselhos de Saúde e Conferências.} “A profundidade destas mudanças devem passar pelo controle social, pois atingem diretamente as pessoas que são o interesse dessas políticas”, avaliou. Em nota pública conjunta assinada pelo CFP e outras entidades por meio da Plataforma Brasileira de Política de Drogas, em 6 de maio de 2019, os coletivos chamam a atenção para “a previsão da internação involuntária pelo prazo de até 3 meses, sem o devido cuidado para que esse dispositivo não seja utilizado para o recolhimento em massa da população em situação de rua como forma de higienização das grandes cidades. Ademais, diferentemente do previsto na Lei da Reforma Psiquiátrica, também não atribui à família ou ao responsável legal o poder de determinar o fim da internação involuntária \cite[grifo meu]{cfp2019}.}
\end{singlespace}
\end{quoting}
O ano de 2020, não muito diferente do anterior, consolidou os avanços das CT's enquanto política. Um dos primeiros acontecimentos que marcam este período é a verba de 300 milhões, praticamente o dobro do ano anterior, que foi destinada a estas instituições, tornando esse o ano com o maior montante de investimento federal. Na postagem oficial no site do Governo encontra-se a seguinte declaração de Eduardo Cabral, membro do SENAPRED: ``Hoje, o governo reconhece essa ação como um trabalho de excelência restaurando vidas. Hoje, são mais de 11 mil pessoas recuperadas, sabemos que temos entre 3 e 4 milhões de usuários de drogas. Quanto mais a gente conseguir mostrar às famílias que estão em desespero que existe solução para a dependência química, menos Cracolândia nós teremos.''\footnote{Disponível em: \url{https://rb.gy/yutiyq}}
\begin{figure}[H]

{\centering \includegraphics[width=1\linewidth]{images/g5-1} 

}

\caption{Verba Federal (em milhões) destinada as Comunidades Terapêuticas entre 2017 e 2020.}\label{fig:g5}
\end{figure}
\bcenter

Fonte: Editais de Licitação
\ecenter

A menção as Cracolândias e o seu fim nessa fala não é algo em vão. O ano de 2020 também é marcado pela aplicação de uma série de políticas que, somadas a nova lei, aumentam ainda mais o escopo de ação das CT's. em Julho a CONAD estabelece em resolução, com prazo de início para 2021, a internação de adolescentes a partir de 12 anos. Em outubro a ministra Damares Alves criou 1,4 mil vagas vagas específicas em CT's para usuários abusivos de droga em situação de rua sendo destinado para isso o total de de R\$ 10,2 mi, valor este que foi dispersado entre 287 instituições espalhadas por todo o país. No discurso de abertura do evento, a ministra enfatizou que:
\begin{quoting}[rightmargin=0cm,leftmargin=4cm]
\begin{singlespace}
{\footnotesize
Mais uma vez estamos trazendo para a pauta e dando visibilidade às comunidades terapêuticas. Isso é um sonho. A maior obra do governo Bolsonaro é essa: investir em vidas e pessoas. Vamos mudar desse jeito, juntos, pois a gente acredita na Comunidade Terapêutica no Brasil \cite{gov2020}.}
\end{singlespace}
\end{quoting}
Que foi complementado pela seguinte fala de Quirino:
\begin{quoting}[rightmargin=0cm,leftmargin=4cm]
\begin{singlespace}
{\footnotesize
Damos mais um passo para fortalecermos as Comunidades Terapêuticas, o cuidado efetivo com as pessoas com dependência química no Brasil e a proteção dos mais vulneráveis, em situação da rua \cite{gov2020}.}
\end{singlespace}
\end{quoting}
Esse evento, assim como os anteriormente citados, servem como ilustrativo da ação destes dois atores no que diz respeito ao aumento do reconhecimento e poder das CT's. Um sonho, nas palavras de Damares, que se torna cada vez mais realidade. Aos poucos todas as barreiras colocadas no processo de escalada destas instituições foram destruidas e em seu lugar foram postas enormes vantagens como as descritas nos parágrafos anteriores.

Ainda em 2020 aconteceram as eleições municipais e, cientes do poder que essas instituições ganham no atual cenário e da nova lei, alguns dos políticos que concorreram a cargos nas eleições municipais deste ano mencionaram elas como parte de sua política. É o caso de Joice Hasselman, candidata a prefeitura de São Paulo pelo PSL, que defendeu a construção de ``Cristolândias'', alegando que iria ``{[}\ldots{]} trazer as igrejas para dentro da Prefeitura de São Paulo''\footnote{Disponível em: \url{https://rb.gy/jsx9sz}}.

Apesar de todas estas conquistas que o Governo Bolsonaro trouxe, é em Dezembro de 2020 que é dado o passo para o que pode vir a ser o movimento final do Governo na disputa entre os modelos de atenção. Nesse período é cogitada a criação de projeto de revogação das portaria de saúde mental, o que representaria a modificação extrema ou o fim de programas criados entre 1991 a 2014, como ``De volta para Casa'', para reinserção de indivíduos que se submetem a longas internações psiquiátricas, e os CAPS, com foco especial nos CAPS AD \footnote{Disponível em: \url{https://bit.ly/3gd1RiR}} que são criados especificamente para pessoas que enfrentam o uso abusivo de drogas.

Apesar de ainda não ter havido pronunciamento oficial sobre o ato é possível entender, no contexto em que ele foi promulgado, suas reais intenções. O fim desses programas é a chave para que a disputa se encerre e as CT's consigam, de todas as formas, serem as protagonistas do cuidado de usuários abusivos de drogas, representando a vitória da abstinência sobre a redução de danos. Vários personagens e instituições se posicionaram contra as revogações. Senadores destacaram que tal ato representa uma parte do desmonte do SUS, e destacam que ``Estamos passando por uma das piores crises que nosso país já viveu. Tem desemprego, caos na saúde, desigualdade social. Desmontar a política de saúde mental é entregar nosso povo ao adoecimento. Não vamos permitir!''\footnote{Disponível em: \url{https://bit.ly/37TpOal}}. Erika Koyak, líder da Frente Parlamentar Mista em Defesa da Reforma Psiquiátrica e da Luta Antimanicomial elucida\footnote{Disponível em: \url{https://rb.gy/hwdaci}} que atráves de tais ações:
\begin{quoting}[rightmargin=0cm,leftmargin=4cm]
\begin{singlespace}
{\footnotesize
o governo açoita a Constituição, usurpa o papel do Congresso e desconstrói a legislação brasileira, além de trazer de volta o holocausto dos hospícios e manicômios. O Ministério da Saúde agora financia choques elétricos e leitos psiquiátricos, hospício cronifica a doença e priva o paciente do convívio com as pessoas. A liberdade é terapêutica}
\end{singlespace}
\end{quoting}
\hypertarget{conclusuxe3o-religiuxe3o-e-poluxedticas-puxfablicas-no-brasil}{%
\subsection{Conclusão: Religião e Políticas Públicas no Brasil}\label{conclusuxe3o-religiuxe3o-e-poluxedticas-puxfablicas-no-brasil}}

A análise investigativa da trajetória das CT's no Brasil desponta uma série de questionamentos: Por que um modelo que se apoia, seja parcialmente ou completamente, em religiosidade é tão valorizado e apoiado pelo Governo brasileiro? Mesmo que ela apareça com outras roupagens como a espiritualidade, o amor, ou a simples educação moral é inegável a presença da religião nesse meio. As CT's em sua maioria são criadas ou mantidas por instituições religiosas, como foi exposto na seção anterior. Dentro de uma lógica de ``Estado Laico'' como se sustenta essa clara contradição?

A resposta para isso se encontra, em parte, na própria linha do tempo aqui desenhada. Em 2011 não acontecia apenas a aproximação das CT's com o governo federal, mas também o fortalecimento da bancada evangélica. A eleição de 2010 foi única não somente por debater o problema do crack, mas também por ser uma das primeiras que fatores religiosos começaram a ter um peso bem maior do que tinham em outras edições \autocite{pierucci_eleicao_2011,machado_religiao_2012}. O número de candidatos religiosos, que apesar de crescerem exponencialmente desde a redemocratização tiveram uma queda de representatividade em 2006, volta a aumentar neste ano. Em especial pela presença dos evangélicos, que passam a conquistar mais espaço. O gráfico abaixo permite acompanhar esse crescimento.
\begin{figure}[H]

{\centering \includegraphics[width=1\linewidth]{images/g6-1} 

}

\caption{Representação Evangélica na Câmara Federal.}\label{fig:g6}
\end{figure}
\bcenter

Fonte: Agência Brasil; DIAP, 2018
\ecenter

O crescimento quase ascendente dos protestantes na política que acontece no período pós-redemocratização é reflexo tanto de um desejo de proteção, dada a perseguição que eles sofreram durante a ditadura militar, quanto uma resposta direta ao pluralismo instituído pela secularização\footnote{Por mais que haja uma discussão sobre a diferença entre laicidade e secularização, os termos serão aqui usados como sinônimos, dado que a discussão aqui feita reflete um aspecto que ambos os conceitos retratam de igual forma, a separação entre Estado/Igreja \autocite[246]{mariano_laicidade_2011-1}.} .

Em relação ao primeiro motivo é importante frisar que nem sempre o comportamento dos evangélicos foi tão político quanto é hoje. Por um longo período eles se mantiveram longe desse meio, sendo o ``apolitismo'' uma de suas marcas registradas \autocites[250]{mariano_laicidade_2011-1}[02]{campos_os_2006}. Entretanto, em 1986, junto com o movimento de redemocratização, o ``crente não se mete em política'' se torna o ``irmão {[}que{]} vota em irmão'' \autocite[251]{mariano_laicidade_2011-1}. Tal transformação foi impelida, principalmente, pelo medo do retorno da perseguição sofrida durante a Ditadura Militar pelo chamado Departamento de Defesa da Fé e pela perda de privilégios junto ao Estado. \autocite[251]{mariano_laicidade_2011-1}. Desse período em diante o número de pentecostais no poder aumentou consideravelmente, através de investimento em comunicação via rádio/TV e do trabalho de base dos próprios pastores \autocite[129]{freston_pentecostalism_1995-1}

No que diz respeito ao segundo motivo é perceptível que a religião não é apenas parte da vida privada. Ela também tem poder nos espaços públicos, em especial na política, espaço no qual a secularidade deveria se sobrepor. Um dos primeiros problemas na lógica empregada no debate dos espaços da religião é a de tratar o ``privado'' como uma categoria não problemática, gerada exclusivamente pelo contraste com as outras esferas da vida social e a ação direta da racionalidade, burocratização e racionalização \autocite[716]{mariano_expansao_2016}. Com dito anteriormente isso cai por terra quando se observa a atuação religiosa no âmbito público.

Casanova \autocite*{casanova_public_1994-1} e Smith \autocite*{smith_disruptive_1996,smith_american_1998} são alguns dos autores que demonstram isso através do estudo da atuação dos grupos religiosos na política. Analisando casos de vários países, incluindo o Brasil, Casanova \autocite*{casanova_public_1994-1} conclui que a separação das esferas, apesar de real, não influenciou em nada a construção e atuação de lobbys religiosos. Smith chega a conclusões similares tanto avaliando o próprio movimento evangélico \autocite{smith_american_1998} quanto em outros cenários como, por exemplo, o ativismo social. Ele identifica movimentos nos quais a participação religiosa foi a chave para o sucesso dos mesmos como, por exemplo, a Revolução da Nicarágua em 1979 e o Anti Apartheid \autocite[02]{smith_disruptive_1996}. O que se percebe, enfim, é que não apenas a religião não é um fenômeno fadado ao desaparecimento ou ao confinamento do espaço privado, mas que esse próprio tipo de lógica é o que as impulsiona a se sobressaírem e lutarem pela sua sobrevivência. Não apenas isso como também estimula os próprios indivíduos a buscarem nela esse espaço utópico de finalidade e salvação que a modernidade falha em entregar \autocite[16]{miranda_religiao_1995}. A existência de um filtro entre as esferas, enfim, não é sinônimo de aplicação do mesmo \autocite[25]{butler_political_2011}

Tendo tudo isto em mente fica mais fácil entender o motivo pelo qual essa política conseguiu tanta adesão e legitimidade. Em pleno espírito de guerra, os religiosos adentram no círculo político e formam seus lobbies que atuam tanto em defesa própria como de seus ideais, instaurando um combate a pautas que os contradigam tais como descriminalização do aborto, ideologia de gênero, casamento entre pessoas do mesmo sexo, eutanásia etc. \autocite{mariano_laicidade_2011-1,natividade_sexualidades_2009}.

Neste conjunto de pautas defendidas também se encaixam as Comunidades Terapêuticas. Tal modelo se mostra mais positivo aos protestantes, pois confronta uma política até então em vigor que era baseada apenas em ciência (Redução de Danos) e que não tinha espaço para Deus ou espiritualidade em seu escopo. Não apenas o fundador da Frente Parlamentar é filiado a Bancada Evangélica como grande parte dos membros desta também o são. O gráfico abaixo demonstra este quantitativo em 2015, ano da aprovação da RDC, e 2019, ano da aprovação da Lei 13.840:
\begin{figure}[H]

{\centering \includegraphics[width=1\linewidth]{images/g7-1} 

}

\caption{Distribuição dos Membros da Bancada Evangélica que faziam parte da Frente Parlamentar das Comunidades Terapêuticas e das APAC'S}\label{fig:g7}
\end{figure}
\bcenter

Fonte: Câmara dos Deputados
\ecenter

Apesar de em 2015 o número de integrantes pertencentes a Bancada Evangélica ser relativamente baixo (40\%) ele cresce bastante em 2019, ao mesmo tempo que diminui, em relação ao ano anterior, a participação de membros de outras bancas. O que se percebe com isso é que elas são, de fato, uma das políticas que eles procuram, enquanto lobby, defender.

\hypertarget{hipuxf3teses}{%
\chapter{Hipóteses}\label{hipuxf3teses}}

Baseado no que foi debatido duas hipóteses são levantadas e, com a ajuda dos dados coletados, serão testadas.

\hypertarget{h1-religiosidade-pode-corroborar-na-sauxfade-fuxedsicomental-dos-sujeitos-pesquisados-e-na-sensauxe7uxe3o-de-recuperauxe7uxe3o.}{%
\section{\texorpdfstring{\textbf{H1}: \emph{``Religiosidade pode corroborar na saúde físico/mental dos sujeitos pesquisados e na sensação de recuperação.''}}{H1: ``Religiosidade pode corroborar na saúde físico/mental dos sujeitos pesquisados e na sensação de recuperação.''}}\label{h1-religiosidade-pode-corroborar-na-sauxfade-fuxedsicomental-dos-sujeitos-pesquisados-e-na-sensauxe7uxe3o-de-recuperauxe7uxe3o.}}

É demonstrado em vários estudos que cultivar algum credo religioso tende a trazer uma série de benefícios, tais como facilidade na recuperação de doenças \autocite{cerqueira-santos_religiao_2004-1}, em especial as de cunho mental \autocite{krause_stress_2011}, criação e manutenção de hábitos ditos saudáveis \autocite{verona_explanations_2011}, melhorias na vida financeira \autocite{mariz_libertacao_1994,potter_growth_2016} etc. Para além desses, também é possível encontrar estudos que apontam a conversão ou a simples adesão à religiões cristãs enquanto uma das formas de enfrentamento da dependência, como os vícios em álcool e substâncias psicotrópicas em geral \autocite{mariz_libertacao_1994,rocha_o_2012-1,smilde_qualitative_2005,targino_os_2016-1}.

Ao se observar as pesquisas relacionadas a CT's percebe-se que a Religiosidade e a Conversão Religiosa são encaradas, tanto por egressos quanto indivíduos em tratamento, como um dos motivadores pessoais para permanência no tratamento \autocites[206]{chu_religious_2012}[367]{shields_religion_2007}, assim como auxiliam no empoderamento pessoal do sujeito e na transformação identitária e comportamental \autocite[327]{targino_comunidades_2017-1}. Especificamente em pesquisas com egressos de CT's percebe-se que ser convertido a algum credo, em especial judaico-cristão, é um dos motivos pelos quais eles mantêm o estado de abstêmia. Segundo Vaglum \autocite*[351]{vaglum_why_1985} ter tido experiências religiosas ou ser convertido foi um fator crucial para 10\% dos egressos pesquisados.

Tais evidências demonstram que a Conversão geralmente tende a ter algumas características positivas para quem dela vivencia e costuma trazer consigo uma série de benefícios que podem auxiliar o indivíduo em sua jornada. Logo, neste trabalho pretende-se encontrar nos dados uma perspectiva similar na qual a Conversão em si ou a simples adesão produziram alguns efeitos positivos no que diz respeito a sensação de recuperação.

\hypertarget{h2-a-conversuxe3o-religiosa-nuxe3o-seruxe1-uma-condiuxe7uxe3o-presente-em-todas-as-configurauxe7uxf5es-possuxedveis-que-expliquem-a-abstuxeamia-ou-a-recuperauxe7uxe3o}{%
\section{\texorpdfstring{\textbf{H2}: \emph{``A Conversão Religiosa não será uma condição presente em todas as configurações possíveis que expliquem a Abstêmia ou a Recuperação''}}{H2: ``A Conversão Religiosa não será uma condição presente em todas as configurações possíveis que expliquem a Abstêmia ou a Recuperação''}}\label{h2-a-conversuxe3o-religiosa-nuxe3o-seruxe1-uma-condiuxe7uxe3o-presente-em-todas-as-configurauxe7uxf5es-possuxedveis-que-expliquem-a-abstuxeamia-ou-a-recuperauxe7uxe3o}}

Apesar de reconhecer que a Conversão Religiosa, assim como a simples filiação, possuem alguns benefícios positivos para o indivíduo, não se acredita que ela será uma variável importante para explicar \textbf{todos} os casos no qual o egresso se manteve abstêmico. Isso se sustenta pelo simples fato de que não existe uma única forma de se alcançar um resultado. A essa característica damos o nome de \emph{multicausalidade}, um fenômeno comum nos objetos de pesquisa das Ciências Sociais \autocite{schneider_set-theoretic_2012}.

Na pesquisa com egressos de Vaglum \autocite*{vaglum_why_1985} aponta que, para 71\% dos egressos, a vontade própria era o que figurava como elemento motivador do estado de abstinência, outros fatores situacionais, como um novo relacionamento, melhorar a relação com os pais ou começar a trabalhar tiveram maior impacto que fatores religiosos \autocite[351]{vaglum_why_1985}. Shields et al \autocite*[367]{shields_religion_2007} destaca que a religiosidade individual não tem impacto sobre o nível de retenção (estadia) do tratamento, mas apenas sobre o cometimento individual com o proposto pelo programa. Scaduto, Barbieri e Santos \autocite*[167]{scaduto_comunidades_2014} apontam que todos os resultados encontrados em seu estudo sobre egressos de CT's variaram segundo o nível de dano sofrido no período de drogadição e características da própria personalidade e não por características do modelo empregado pela CT. Tais informações além de demonstrarem que o efeito da religião é difuso, também apontam o quão frágil e perigoso é confiar todos os casos de uso abusivo de drogas a modelos exclusivamente abstêmios, como prenuncia o Capítulo II da Lei 13.840 de 5 de junho de 2019.

A partir destas pressuposições assume-se que, nas configurações que geraram como resultado a abstêmia, algumas não contaram com a presença da Conversão Religiosa. Outras condições demonstraram um poder explicativo maior.

\hypertarget{h3-a-conversuxe3o-religiosa-nuxe3o-se-apresentaruxe1-como-uma-condiuxe7uxe3o-necessuxe1ria-ou-suficiente-para-a-explicauxe7uxe3o-geral-da-abstuxeamia-ou-recuperauxe7uxe3o}{%
\section{\texorpdfstring{\textbf{H3}: \emph{``A Conversão Religiosa não se apresentará como uma condição necessária ou suficiente para a explicação geral da Abstêmia ou Recuperação''}}{H3: ``A Conversão Religiosa não se apresentará como uma condição necessária ou suficiente para a explicação geral da Abstêmia ou Recuperação''}}\label{h3-a-conversuxe3o-religiosa-nuxe3o-se-apresentaruxe1-como-uma-condiuxe7uxe3o-necessuxe1ria-ou-suficiente-para-a-explicauxe7uxe3o-geral-da-abstuxeamia-ou-recuperauxe7uxe3o}}

Tendo como base a hipótese anterior, de que a Conversão Religiosa não conseguirá ser parte da explicação causal de todos as configurações que tenham como resultado a abstêmia, assume-se que ela não poderá ser um fator necessário para a existência da mesma, e muito menos suficiente em si mesmo para a explicação do resultado. A única pesquisa encontrada que vai no oposto do aqui proposto é a de Perrone \autocite*{perrone_fatores_2019} na qual se afirma haver correlação inversamente proporcional entre participação de grupos religiosos e chances de recidiva \autocite[142]{perrone_fatores_2019}. Tal efeito, no entanto, se mostra como não pertencente a Religião em si dado que as próprias variáveis que mensuram elementos religiosos (Religião e Prática Religiosa) deram resultados não significativos. As redes sociais providas é que se provaram como mais eficazes de explicação da não recidiva.

Redes sociais são também um fator essencial na própria Conversão dos indivíduos. Um exemplo disso é o estudo de Smilde \autocite*{smilde_qualitative_2005} sobre motivação para Conversão entre Venezuelanos. Apesar de, discursivamente, os principais motivos elencados para a Conversão Religiosa fossem ``querer mudar de vida'' ou ``se livrar das drogas'' análises de QCA (Qualitative Comparative Analysis) apontaram que era conviver com protestantes, e não a resolução de problemas, o fator necessário, suficiente e significativo para causar a Conversão naquele grupo \autocite[766]{smilde_qualitative_2005}. Estudos de outras áreas também apontam sobre o efeito das redes sociais no próprio processo de recuperação do uso abusivo de drogas \autocite{costa2001processo}. Estudos como esses servem de indicativo para sempre prestar um olhar crítico e parcimonioso sobre elementos que discursivamente podem parecer ter efeito, mas que, sob análise acurada, podem não o ter.

Acredita-se, por fim, que outras condições e configurações terão uma consistência maior do que o simples fato de se Converter quando o resultado for a manutenção da abstêmia, logo tal variável não será suficiente, muito menos necessária, para o outcome positivo

\newpage

\hypertarget{metodologia}{%
\chapter{Metodologia}\label{metodologia}}

Após uma explanação sobre do que esse estudo trata, dos desafios que essa temática impõe e das hipóteses criadas com base na teoria existente sobre o assunto, o próximo passo é detalhar quais meios serão utilizados para atingir os desideratos aqui propostos. Nas próximas seções irei falar sobre as soluções que foram aqui empregadas. Primeiramente discorrerei sobre o questionário, em especial sobre como ele foi arquitetado, o que cada seção pretendia capturar, como as perguntas foram projetadas e outros elementos como randomização e diminuição de vieses. Após, irei falar sobre a metodologia utilizada tanto para a construção do estudo quanto para a análise dos dados coletadas, no caso a \emph{Qualitative Comparative Analysis} (QCA).

Ainda dentro da segunda seção tratarei de explicar os mecanismos teóricos e matemáticos que dão suporte a esta metodologia, em especial Teoria dos Conjuntos e Lógica Booleana. Tal medida é essencial para que o leitor entenda como a metodologia se opera e como os resultados são obtidos e interpretados. Ressalto ainda que a QCA é totalmente diferente de outras abordagens estatísticas até então utilizadas pela Sociologia, dado que não tem precedentes como Normalidade e Aleatoriedade \autocite[04]{schneider_set-theoretic_2012} o que torna importante a leitura desta seção. O uso de notação matemática é inevitável, mas será feito da forma mais acessível possível. Para além dos mecanismos internos também serão abordadas questões como a Causalidade Conjuntural, dado que se entende debater sobre efeitos, e relações de Necessidade e Suficiência entre as condições.

\hypertarget{o-questionuxe1rio}{%
\section{O Questionário}\label{o-questionuxe1rio}}

\hypertarget{construuxe7uxe3o-do-questionuxe1rio}{%
\subsection{Construção do Questionário}\label{construuxe7uxe3o-do-questionuxe1rio}}

Como dito anteriormente são escassos os estudos quantitativos nas duas grandes temáticas deste estudo: Conversão Religiosa e Comunidades Terapêuticas. A grande maioria se baseia em Entrevistas em Profundidade e Etnografias/Observação Participante, o que acabou concedendo a essas técnicas uma espécie de ``cânone metodológico''. Logo, um dos primeiros desafios foi a criação de um instrumental que conseguisse capturar, de forma acurada, todos esses elementos que pertencem ao campo da subjetividade e digam respeito as hipóteses que aqui se levantam. Para isso o apoio nos trabalhos qualitativos já existentes foi fundamental, dado que foram a base para a construção de construtos mensuráveis do conceito.

O questionário foi dividido em 6 seções: \emph{Dados Socioeconômicos}, \emph{Religião}, \emph{Histórico de Dependência Química e Passagens por Comunidades Terapêuticas}, \emph{Pós Comunidade Terapêutica}, \emph{Escalas de Atitude} e \emph{Fatores Sociais}. Cada uma dessas seções procura extrair, em ordem, informações específicas sobre o respondente, sua relação com a igreja/fé, suas passagens por CT's, sua relação passada e atual com o uso abusivo de drogas e suas redes sociais. Nos próximos parágrafos serão explanadas cada uma dessas seções, dando foco ao que cada uma delas tenta medir e de que forma o pretendem fazer.

\hypertarget{dados-socioeconuxf4micos}{%
\subsubsection{Dados Socioeconômicos}\label{dados-socioeconuxf4micos}}

Constituída de 8 perguntas, sendo uma delas feita em duas partes \emph{(double step)}, esta seção se destina a capturar as seguintes informações básicas do respondente: Idade, Sexo, Profissão, Número de filhos e se tem contato com eles, Escolaridade, Estado Civil, Estado no qual reside, e a idade que começou a trabalhar. Como não existiu uma etapa qualitativa utilizou-se essa seção para criar um ``perfil geral'' da pessoa que respondeu o questionário.

\hypertarget{religiuxe3o}{%
\subsubsection{Religião}\label{religiuxe3o}}

A seção de religião é constituída de 9 questões. É a partir dela que o questionário começa a se especificar segundo lógica própria logo apenas 6 eram visíveis para todos os respondentes. Caso o respondente não estivesse frequentando nenhuma igreja ele não veria a questão perguntando qual seria, se não fosse protestante não veria a que pedia pela especificação da denominação que ele fazia parte e se não fosse convertido não seria perguntado aonde este processo deu início. No geral esta seção tentou avaliar frequência com que vai à missas/cultos/reuniões, nível de cometimento com atividades religiosas, nível de crença em elementos religiosos e, por fim, uma escala de religiosidade na qual o próprio respondente foi convidado a considerar, de 1 a 10, o quão religioso era.

Outro destaque desta seção é a pergunta sobre Conversão, uma das que iram configurar como parte do que aqui pretendo testar. Como ressaltei na discussão do capítulo anterior ela possui propriedades bastante específicas sendo uma delas o seu caráter processual e discursivo, logo não se pode limitar o questionamento a um simples ``sou'' ou ``não sou'', mas também aceitar um estado de potência, um ``serei''. Com base nisso optei por usar 4 opções: \emph{``Sim, sou convertido''}, \emph{``Não, não sou convertido''}, \emph{``Estou em processo de conversão''} e \emph{``Não existe algo como conversão na minha religião''}.

\hypertarget{histuxf3rico-de-dependuxeancia-quuxedmica-e-passagens-por-comunidades-terapuxeauticas}{%
\subsubsection{Histórico de Dependência Química e Passagens por Comunidades Terapêuticas}\label{histuxf3rico-de-dependuxeancia-quuxedmica-e-passagens-por-comunidades-terapuxeauticas}}

Esta parte do questionário com 14 perguntas (sendo específicas uma pergunta sobre o credo da CT, caso fosse religiosa, e outra sobre quanto tempo passou na instituição caso não tivesse concluído o tratamento), se destinava a checar questões específicas referentes ao uso abusivo de drogas e vivências em Comunidades Terapêuticas. Assim como a primeira, esta seção serviu mais como delineador do respondente do que dado para teste de hipótese. Por meio dela se pode saber mais sobre as condições nas quais o respondente foi acolhido, relação com a família, escala auto pontuada de vício, quais drogas fazia uso, características do modelo terapêutica empregado pela CT e quais funções exerceu durante a passagem.

\hypertarget{puxf3s-comunidade-terapuxeautica}{%
\subsubsection{Pós Comunidade Terapêutica}\label{puxf3s-comunidade-terapuxeautica}}

A antepenúltima seção do questionário é uma das mais importantes, dado que mensura o estado de abstinência. Composta de 6 perguntas, sendo uma delas específica para pessoas em abstinência e uma para pessoas que entraram em relapso, essa parte tenta avaliar como o respondente se encontra atualmente e suas opiniões sobre o processo terapêutico, em especial se acha que outros tratamentos seriam tão eficazes quanto, a opinião sobre tratamentos à base de redução de danos e recuperação e reinserção social.

\hypertarget{escalas-de-atitude}{%
\subsubsection{Escalas de Atitude}\label{escalas-de-atitude}}

Essa seção, como o próprio nome já diz, tenta avaliar como o respondente reage a uma série de afirmações que dizem respeito ao tratamento em si, a recaída e a importância da religião no processo de se manter limpo. Ao todo são 11 itens em Escala Likert de 5 níveis indo de ``Concordo Totalmente'' para ``Discordo Totalmente''

\hypertarget{fatores-sociais}{%
\subsubsection{Fatores Sociais}\label{fatores-sociais}}

A última seção do questionário, com 6 perguntas, tenta acessar a rede de apoio do respondente. Perguntou-se quais pessoas ou grupos de pessoas forneceram apoio, se ainda havia contato com quem conheceu durante o período de uso abusivo e pessoas que estavam na mesma Comunidade, a importância que a pessoa atribui a si mesma no processo de se manter limpo, a atual relação com a família e uma escala auto pontuada de recuperação

\hypertarget{aplicauxe7uxe3o-do-questionuxe1rio}{%
\subsection{Aplicação do Questionário}\label{aplicauxe7uxe3o-do-questionuxe1rio}}

O próximo passo após a criação do questionário foi obter a aprovação do Comitê de Ética para a aplicação, dado que alguns dos temas que se pretendiam sondar são considerados sensíveis. Mesmo que perguntas mais densas fossem evitadas, algumas sobre relação familiar e quais drogas eram consumidas podiam potencialmente servir como gatilho para memórias ruins, sentimentos de constrangimento e ansiedade etc. A pesquisa foi aprovada sem muitas modificações com o CAAE de número 42194620.9.0000.5149.

Após aprovação o questionário foi hospedado na plataforma \emph{SurveyMonkey} e pode-se iniciar a coleta que foi dos dias 10 à 24 de Maio. A divulgação do link de acesso do questionário foi feita em páginas do Facebook direcionadas a CT's e Egressos, Grupos de WhatsApp de Clínicas de Recuperação e outras Redes Sociais. Foi-se utilizada a arte abaixo junto com um texto que introduzia rapidamente a pesquisa, o que era necessário para respondê-la, possíveis riscos e agradecimentos por considerar participar.
\begin{figure}[H]

{\centering \includegraphics[width=0.8\linewidth]{images/imagem14} 

}

\caption{Imagem oficial de divulgação da pesquisa.}\label{fig:imagem14}
\end{figure}
\bcenter

Fonte: Do autor, 2021
\ecenter

Outros cuidados adicionais durante a coleta foram a proibição do site de coletar o IP do respondente, o que garantiria o anonimato, e resposta múltipla, o que limitou a um questionário por aparelho.

Em ordem de minimizar o viés de ordem as perguntas foram aleatorizadas dentro dos blocos. Ou seja, a ordem no qual as perguntas apareciam era diferente para cada respondente. Ao final da pesquisa o participante era convidado a, caso quisesse, assistir um vídeo para minimizar os efeitos de qualquer gatilho gerado pelo questionário.

\hypertarget{causalidade-nas-ciuxeancias-sociais}{%
\section{Causalidade nas Ciências Sociais}\label{causalidade-nas-ciuxeancias-sociais}}

Até o presente momento foram debatidos quase todos os tópicos que são mencionados no título desta dissertação: Uso Abusivo de Drogas, Conversão Religiosa e Políticas Públicas. Todos eles são pertinentes à discussão que aqui se pretende realizar. Resta, entretanto, falar sobre um último assunto que também terá importância nesta pesquisa: \emph{Causalidade}.

Este tema entre em foco a partir do momento que se pretende debater sobre efeitos. Falar sobre isso é, em algum nível, afirmar que as hipóteses assumem relações causais. É dizer que o elemento que se tenta explicar acontece, ou tem grande possibilidade de acontecer, em função de um outro que o antecede. Esse tipo de pesquisa, no entanto, carrega alguns problemas. Um dos primeiros é o da origem da causação. Digamos que \(X\) seja um fenômeno qualquer e \(Y\) seja um elemento por ele afetado: É seguro afirmar que \(X\) causa \(Y\)? A imagem abaixo descreve visualmente algumas possibilidades de relação entre as variáveis. Note que uma primeira possibilidade (A) é \(X\) causar \(Y\), uma segunda (B) é o oposto disso e uma terceira (C) é uma relação de dependência, na qual ambas se causam.
\begin{figure}[H]

{\centering \includegraphics[width=0.8\linewidth]{images/imagem15} 

}

\caption{Direções possíveis para relação de efeito entre X e Y}\label{fig:imagem15}
\end{figure}
\bcenter

Fonte: Do autor, 2021
\ecenter

Apesar de ser um problema, entender a origem da causação é algo relativamente simples de resolver. Basta que se separe as variáveis que se pretende analisar em um ambiente isolado, no qual elas não sofram interferência e realizar uma série de experimentos que possam comprovar a influência de uma sobre a outra. Esse tipo de resolução, muito fácil de se realizar em outras ciências, é bastante difícil nas Ciências Sociais dado que a natureza das coisas que se investigam não permite que tal façanha aconteça com muita facilidade \autocite[277]{montenegro_desenho_2016}. Os elementos da vida social quase sempre tendem a acontecer ao mesmo tempo, gerando uma rede de fenômenos que dificilmente se pode separar. Nada no mundo social acontece em vácuo \autocite[01]{gerrits_social_2020}.

Tendo isto em mente voltemos ao exemplo do parágrafo anterior. Dentro da relação de \(X\) e \(Y\) é possível que exista uma outra variável \(\mu\) que, quando considerada, também possui \acrshort{efeito} sobre \(Y\) ou até mesmo sobre \(X\), de forma que toda relação antes hipotetizada pode não existir mais. As imagens abaixo descrevem essas possíveis relações de efeito.
\begin{figure}[H]

{\centering \includegraphics[width=0.8\linewidth]{images/imagem16} 

}

\caption{Direções possíveis para relação de efeito as variáveis}\label{fig:imagem16}
\end{figure}
\bcenter

Fonte: Do autor, 2021
\ecenter

Uma das primeiras possibilidades de interação entre as variáveis é a chamada Causação Indireta (D), na qual \(X\) possui algum impacto em \(Y\) mas isso acontece por meio de \(\mu\) que é gerado por ele, não porque exista uma relação direta entre ambos. Uma outra interação é a Moderada (E) na qual se percebe o impacto de \(\mu\) sobre o efeito de \(X\) em \(Y\) e a última é a Espúria (F) aonde a ligação entre as variáveis se desfaz quando se insere uma terceira, comprovando assim que o efeito antes existente era falso.

Em ordem de averiguar a presença de Causalidade, como é possível perceber nos exemplos dados acima, alguns pressupostos devem ser observados. No geral eles se dividem em: \emph{Associação entre as variáveis}, \emph{Precedência Temporal} e \emph{Não espuriosidade} \autocite{pearl_causality_2009,asher_causal_1983}. Ou seja, é preciso que haja alguma espécie de associação (quase sempre correlacional), que a variável que se usa pra explicar tenha ocorrido antes da que será explicada e que a relação entre elas seja real e não mediada por uma outra, seja essa identificada ou não. Apesar da existência de discussões sobre outros pressupostos necessários, boa parte deles mantêm estes três como os necessários para estabelecer Causação \autocite[272]{paranhos_causalidade_2013}.

Essa noção, no entanto, não é a única que existe nas Ciências Sociais\footnote{Para uma discussão mais apurada sobre as noções de causalidade nas ciências sociais e uma visão alternativa sobre pluralismo ver Gerring \autocite*{gerring_causation_2005}}. Isso ocorre, em geral, porque algumas situações só podem ser melhor trabalhadas utilizando noções de Causalidade que não funcionem em um sistema de causação única. Ao mesmo tempo que uma pessoa se converte ao protestantismo, por exemplo, ela está conhecendo novas pessoas, recebendo apoio, fortificando laços e redes sociais e assimilando uma nova maneira de traduzir ao mundo ao seu redor. Não é possível destrinchar essas coisas e observar o efeito específico de uma por uma sobre um determinado \emph{outcome}. Isso também se reflete em outros fenômenos sociais mais complexos, como a pobreza e educação nos quais múltiplas variáveis atuam em conjunto, umas com mais força, outras com menos, no resultado final. A complexidade de alguns fenômenos é mais bem compreendida quando se parte do pressuposto que as causas podem atuar \emph{juntas} na produção de um efeito ou que se pode conseguir o mesmo efeito a partir de causas diferentes.

Para além desses problemas de ordem ontológica existem também os de ordem metodológica. Técnicas estatísticas rotineiras, ou até mesmo mais complexas, são fortemente fundamentadas, como dito anteriormente, em correlação. Logo costumam lidar com a \emph{presença} de elementos e como essa presença atua sobre um dado resultado, pouco ou nada debatem sobre situações de \emph{ausência} \autocite[100]{dusa_mathematical_2008} que, quando trabalhada, é quase sempre no intuito de entender o quanto afeta a ausência do resultado \autocite[22]{ragin_redesigning_2008}. Ambas as qualidades são trabalhadas de forma a gerar equivalência e simetria, a presença na presença, a falta na falta. Situações plenamente assimétricas, nos quais a falta acontece apenas em um dos lados da equação, costumam ser ignoradas ou punidas.

Outra questão de mesmo mote é o número de casos necessários para se obter resultados significativos. Graças a pressupostos matemáticos como Normalidade, Aleatoriedade e Homocedasticidade espera-se que uma análise estatística válida possua um \emph{n} de valor consideravelmente alto o que nem sempre é possível nas Ciências Sociais, dado que os temas/fenômenos pesquisados na maioria das vezes não permitem que sujeitos aleatórios sejam entrevistados. Quase sempre se é estudado situações ou pessoas específicas cuja aleatorização seria extremamente difícil e não desejada.

\hypertarget{qualitative-comparative-analysis-qca}{%
\section{Qualitative Comparative Analysis (QCA)}\label{qualitative-comparative-analysis-qca}}

Visando não apenas levar em conta especificidades advindas da complexidade dos fenômenos sociais que surgem ao lidar com causalidade mas também contornar frustrações de ordem metodológica que o sociólogo Charles Ragin criou, em 1987, a primeira versão do algoritmo que daria origem ao que seria chamado de \emph{Qualitative Comparative Analysis}. De forma geral, Ragin defini sua técnica enquanto ``(\ldots) data reduction that uses Bolean Algebra to simplify complex data structures in a logical and holistic manner'' \autocite[viii]{ragin_comparative_1987} o que se diferencia bastante das demais abordagens até então empregadas na Sociologia dado que une álgebra booleana à comparação qualitativa, o que concede à técnica o ar holístico mencionado.

A \acrshort{QCA} tem como pilares três conceitos principais: \emph{Causalidade Conjuntural}, \emph{Equifinalidade} e \emph{Multifinalidade}. Todos esses pressupostos, mesmo que se refiram a partes diferentes do processo, se complementam no que diz respeito ao \emph{framework} da técnica. Por causalidade conjuntural assume-se que uma combinação particular de causas, não uma causa específica, é o que gera um resultado. A Equifinalidade pressupõe que o mesmo resultado pode ser obtido a partir de causas e combinações diferentes e a Multicausalidade atesta que o impacto que uma condição pode ter sobre um determinado resultado pode ser afetado pelo seu contexto, ou seja, em condições diferentes uma condição pode não gerar o mesmo resultado \autocites[xxii]{ragin_comparative_1987}[164]{gerring_causation_2005}[02]{gerrits_social_2020}.

As diferenças desta metodologia em contraste com os métodos convencionais não param por aí. Existem ainda diferenças nas nomenclaturas e objetos utilizados. A tabela abaixo demonstra o comparativo entre os modelos:
\begin{table}[H]

\caption{\label{tab:qcamodelos}Diferenças entre Modelos Estatísticos Convencionais e QCA}
\centering
\begin{tabular}[t]{ll}
\toprule
\textbf{Modelo Estatístico Convencional} & \textbf{QCA}\\
\midrule
Variáveis & Conjuntos\\
Mensuramento & Calibração\\
Variável Dependente & Outcome Qualitativo\\
Populações Existentes & Populações Construídas\\
Correlações & Relações Teorético Conjunturais\\
\addlinespace
Matriz de Correlação & Tabela Verdade\\
Efeitos líquidos das variáveis & Caminhos Causais\\
\bottomrule
\end{tabular}
\end{table}
\bcenter

Fonte: Adaptado de Ragin \autocite*[xxiii]{ragin_comparative_1987}
\ecenter

Uma das primeiras coisas que é possível notar é que todo o arcabouço da QCA comunga diretamente com os pressupostos por ela colocados, sendo a principal via de separação o total rompimento com correlações. Como dito anteriormente, correlações são a base das técnicas estatísticas mais comuns e elas costumam ser reflexos de relações simétricas. Veja os exemplo abaixo:
\begin{figure}[H]

{\centering \includegraphics[width=1\linewidth]{images/correlacoes-1} 

}

\caption{Relações de Correlação entre variáveis.}\label{fig:correlacoes}
\end{figure}
\bcenter

Fonte: Do autor, 2021
\ecenter

Note que quanto maior o nível de correlação entre as variáveis mais lineares elas tendem a ser uma em relação a outra e é essa característica que faz com que se possa afirmar que, em algum grau, elas promovem um certo impacto, ou efeito, uma sobre a outra. Apesar de ser uma pressuposição um tanto lógica e até mesmo fácil de visualizar ela não é de todo confiável dada que o único fato que ela representa é que as variáveis têm uma relação linear entre si, não que elas se causam. Um exemplo claro disso são as chamadas Correlações Espúrias, observe a imagem abaixo:
\begin{figure}[H]

{\centering \includegraphics[width=0.8\linewidth]{images/imagem17} 

}

\caption{Correlação entre Mortes por Vapor e Idade da Miss America}\label{fig:imagem17}
\end{figure}
\bcenter

Fonte: Tylervigen.com, 2021 \footnote{Para checar mais correlações espúrias acesse \url{http://www.tylervigen.com/spurious-correlations}}
\ecenter  

Note que a correlação entre ambos os itens é alta (r= 0.87) e eles possuem um padrão linear entre si, a ausência sendo refletida na ausência e a presença refletida na presença, mas não fazem o mínimo sentido. Claro que isso não serve para negar ou diminuir essas técnicas, muito menos ofuscar o seu potencial e importância para a estatística. No entanto, é importante que se tenha em mente que correlação não é causação e é necessário que, caso se pretenda debater efeitos, se tenha em mente que esse tipo de abordagem correlacional deve ser usada com bastante cautela.

A opção pelas relações teórico conjunturais, por outro lado, rompe com a necessidade de relação linear e exige, por outro lado, que o que se estuda tenha uma conexão teórica e conjuntural. Ou seja, são analisadas a relação de coisas que sejam, em teoria, relacionadas e que possam ser agrupadas em conjuntos de pertencimento e não pertencimento não simétricos. Um exemplo prático da diferença entre ambas as abordagens é afirmar que países desenvolvidos são democráticos. Ao se fazer isso pelo modo correlacional os países que não são desenvolvidos e possuem regimes democráticos serviriam como um potencial de diminuição do coeficiente de correlação, ou seriam considerados \emph{outliers}. Por outro lado, ao abordar isso pelo modo teorético conjuntural percebe-se que é possível assumir que países de primeiro mundo são democráticos ao mesmo tempo que se pode lidar com casos contrários pois o fato do conjunto ``Países Desenvolvidos'' estar contido no conjunto ``Países Democráticos'' não impede que outros casos também façam parte desse conjunto \autocite[xxvi-xxvii]{ragin_comparative_1987}. A imagem abaixo demonstra graficamente como a lógica de conjuntos funciona nesse exemplo.
\begin{figure}[H]

{\centering \includegraphics[width=0.6\linewidth]{images/imagem18} 

}

\caption{Relações Conjunturais entre Democracia e Países de Primeiro Mundo}\label{fig:imagem18}
\end{figure}
\bcenter

Fonte: Adaptado de \autocite{ragin_redesigning_2008}
\ecenter

A opção por trabalhar com Conjuntos ao invés de Variáveis permite que se utilize notação e linguagem própria. \(A\), nesse caso, é subconjunto de \(B\) e \(C\) tem uma intercessão com \(B\). Isso pode ser descrito matematicamente da seguinte forma:

\[ A \subset B \]
\[  C \cup B \]

Ao fazer a transição entre variáveis e conjunto assume-se que eles poderão ter três níveis de pertencimento: Pertencimento \acrshort{total} (\(\subset\)), \acrshort{Semi} Pertencimento (União (\(\cap\)) \acrshort{ou} Intercessão (\(\cup\))) e Não \acrshort{pertencimento} (\(\not\subset\)). Esses níveis de pertencimento também se manifestam na forma como os dados são preparados para análise, o que leva para o próximo tópico: Calibração.

O processo de mensuração de um elemento, dentro da metodologia convencional, se baseia na comparação de um elemento a partir de suas próprias características distributivas. Quando se analisa uma variável nós dizemos que um ponto \(X\) está em um quartil específico, ou acima/abaixo da média/mediana. Todas essas medidas têm em comum o fato de dizerem respeito ao comportamento de um determinado termo em relação a sua própria distribuição. O processo de Calibração, ao contrário, é feito a partir de critérios externos. O nível de pertencimento não é dado pela própria variável, mas sim por parâmetros fornecidos \emph{a priori} pela teoria ou pelo próprio pesquisador. Esses parâmetros, por sua vez, se manifestam em duas formas: \emph{Crisp} e \emph{Fuzzy}.

Conjuntos do tipo Crisp são criados levando em conta a simples presença (1) ou ausência (0) de uma condição, não assumindo valores acima, abaixo ou entre 0 e 1. Isso pode ser expresso matematicamente da seguinte forma:

\[ A:X \Rightarrow \{0,1\} \]

Para um conjunto de tipo Crisp \(A\) os únicos valores possíveis para \(X\) são 0 e 1.

Já os do tipo Fuzzy consideram presença enquanto um espectro, sendo possível ajustar valores que representem níveis de presença entre 0 e 1. Geralmente esses valores se dividem em curvas do tipo S (0, 0.5, 1) ou Sino (0,\(X1\),\ldots,\(Xn\), 1). A introdução da lógica Fuzzy na Sociologia representa um grande salto analítico dado que permite analisar estados de ``semi pertencimento'' com maior acurâcia, ao mesmo tempo que preserva a característica qualitativa dos dados via Calibração \autocites[08]{ragin_redesigning_2008}[18]{schneider_set-theoretic_2012}. A QCA, graças a essa característica, configura-se como uma alternativa viável a binariedade cultivada pelas técnicas quantitativas e criticada por alguns qualitativistas \autocite{mahoney_tale_2006}. Em linguagem matemática podemos dizer que os conjuntos Fuzzy são representados por:

\[ A:X \Rightarrow [0,1]  \subset \Re \]

Para um conjunto de tipo Fuzzy \(A\) assume como valores possíveis para \(X\) os números entre 0 e 1 pertencente ao conjunto dos Números \acrshort{Reais}. A distribuição destes valores pode assumir várias formas, uma das mais comuns é a disposta abaixo, extraída de Duşa \autocite*{dusa_qca_2018}:

\[ a_{x} = \begin{Bmatrix}
0 & \mbox {se} & x\leq e, \\ 
\frac{1}{2}  \left (\frac{e - x}{e - c} \right)^b & \mbox {se} &  e< x\leq c\\ 
1 - \frac{1}{2} \left (\frac{i - x}{i - c} \right)^a &\mbox {se} & c < x \leq i \\
1 & \mbox {se} & x > i  
\end{Bmatrix} \]
Na qual \(e\) é o valor de exclusão total, \(c\) é o limite de convergência entre os valores máximos e mínimo, \(i\) é o valor de inclusão total, \(x\) é o valor real da variável a ser calibrada, \(b\) é o valor abaixo do limite de convergência e \(a\) o valor acima \autocite[82]{dusa_qca_2018}

Uma última diferença a ser comentada é a entre Matriz de Correlação e Tabela Verdade. Como vem sendo debatido neste capítulo, a QCA age contra a corrente e utiliza algebra booleana em ordem de superar limitações impostas pelos métodos convencionais. Como ela se guia por esse método é necessário que se utilize no processo uma ferramenta que possa conter todas as possibilidades de combinação e sua relação com o outcome. Quem cumpre esse papel é a chamada \emph{Tabela Verdade}. É nela que serão armazenadas todas as correspondências possíveis e suas relações entre si e com o resultado. Observe o exemplo abaixo:
\begin{table}[H]

\caption{\label{tab:tabverdade}Exemplo de Tabela Verdade}
\centering
\begin{tabular}[t]{rrrr}
\toprule
\textbf{A} & \textbf{B} & \textbf{C} & \textbf{Outcome}\\
\midrule
1 & 0 & 1 & 1\\
1 & 0 & 0 & 0\\
0 & 1 & 1 & 0\\
\bottomrule
\end{tabular}
\end{table}
Note que existem condições (ou conjuntos) chamados A, B e C e uma quarta coluna chamada Outcome, que atesta quando aquelas condições geram ou não o resultado. A anotação para identificação de Presença ou Ausência é por meio dos números 0 e 1, sendo 1 para o primeiro e 0 para o último. Através dela a localização de padrões lógicos e combinações causais fica mais fácil e intuitiva. Na próxima seção serão explicados os últimos conceitos necessários para entender por completo esta técnica: Necessidade, Suficiência, Cobertura, Consistência e Operações Booleanas.

\hypertarget{paruxe2metros-de-ajuste}{%
\section{Parâmetros de Ajuste}\label{paruxe2metros-de-ajuste}}

Após ter discutido sobre o que é a QCA e de quais formas ela se diferencia das técnicas convencionais resta debater sobre quais parâmetros se utilizam para saber se uma combinação é válida e consegue explicar o fenômeno pesquisado. Antes de explicar essas propriedades é necessário voltar um pouco a Teoria dos Conjuntos e entender mais sobre como ela se manifesta na Tabela Verdade.

Como dito anteriormente, existem três formas nas quais os conjuntos podem ser categorizados: Pertencimento Total, Semi pertencimento e Não pertencimento e isso é percebido na forma como os dados são calibrados. Dentro da tabela verdade isso se manifesta a partir dos chamados operadores lógicos. A combinação de condições pode acontecer através da \emph{União} ou da \emph{Intercessão}. No caso em que as condições se combinam em ordem de gerar uma única suficiente (União) elas são unidas pelo símbolo (+) e interpretadas como \emph{E}, no caso em que mais de um caminho causal é detectado, ou seja, mais que uma combinação pode gerar o outcome utiliza-se o sinal (*) e interpretadas como \emph{OU} e quando uma condição precisa estar negada para se obter o outcome utilizasse tanto o (\textasciitilde) quanto o nome dela em minúsculo. A imagem abaixo ajuda a sumarizar melhor os operadores.
\begin{figure}[H]

{\centering \includegraphics[width=0.8\linewidth]{images/imagem19} 

}

\caption{Operadores Lógicos}\label{fig:imagem19}
\end{figure}
\bcenter

Fonte: Adaptado de \autocite{betarelli_junior_introducao_2018}
\ecenter

Sabendo as formas como as condições são categorizadas podemos discutir sobre os parâmetros que são utilizados para validação. Um dos primeiros são a Necessidade e a Suficiência. De forma geral uma condição é considerada \emph{necessária} quando o resultado só acontece na presença da mesma e \emph{suficiente} quando apenas ela consegue causar o resultado, noções bastante próximas das que se costuma utilizar no dia a dia para os mesmos termos. Vejamos um exemplo simples de como esses conceitos funcionam pela lógica da QCA.

Imagine que se queira explicar o crescimento saudável de uma planta. Muitas condições podem ser elencadas: \emph{Presença de Luz Solar}, \emph{Presença constante de Água}, \emph{Presença de Nutrientes no Solo}, \emph{Ausência de Predadores} e outras mais, dependendo do vegetal. Percebe-se que, dentre os pontos elencados todos são necessários para que ela possa crescer. A ausência de luz solar provoca a morte antes do que deveria, o mesmo vale para a ausência de água e nutrientes no solo. Um ambiente com vários animais herbívoros ou onívoros também não faria com que o amadurecimento fosse pleno. Apesar dessas afirmações, nenhuma dessas condições é o suficiente para explicar o processo. Não se pode afirmar com propriedade algo como ``A única coisa que não pode faltar, em hipótese nenhuma, é \emph{x}''. Não existe algo único que pode garantir em 100\% o desenvolvimento. Em outras palavras, não existe condição que seja, em si, suficiente para explicar o processo. A união de todas estas coisas, no entanto, consegue realizar esse feito. O diagrama abaixo ilustra esse processo:
\begin{figure}[H]

{\centering \includegraphics[width=0.8\linewidth]{images/imagem20} 

}

\caption{Diagrama de Venn da relação entre as condições necessárias para o Crescimento Saudável de uma planta}\label{fig:imagem20}
\end{figure}
\bcenter

Fonte: Do Autor, 2021
\ecenter

É possível notar pelo diagrama que a intercessão entre as condições é o que gera o resultado e que de nenhuma forma ele se encontra contido em alguma delas. Nenhuma das opções propostas é suficiente para explicar o fenômeno, a união de todas, no entanto, o é. Desse exemplo podemos extrair algumas informações importantes sobre a QCA. Uma primeira é que nem sempre existiram elementos que serão suficientes em si, mas que podem ser necessários para produzir um resultado e a segunda é que um conjunto de condições necessárias podem se tornar suficientes. Para esse fenômeno se dá o nome de condições \acrshort{INUS} (Insufficient, but necessary). O oposto disso, no qual existem opções suficientes, mas desnecessárias é chamado de \acrshort{SUIN} (Sufficient, but unnecessary). É a partir dessas combinações que se pode identificar os tipos de relação existentes entre os conjuntos e qualificar de forma mais precisa o resultado final. A tabela abaixo resume essas relações
\begin{longtable}[]{@{}
  >{\raggedright\arraybackslash}p{(\columnwidth - 2\tabcolsep) * \real{0.22}}
  >{\raggedright\arraybackslash}p{(\columnwidth - 2\tabcolsep) * \real{0.78}}@{}}
\caption{\label{tab:suin-inus} Relações entre Necessidade e Suficiência}\tabularnewline
\toprule
Relação & Descrição \\
\midrule
\endfirsthead
\toprule
Relação & Descrição \\
\midrule
\endhead
Necessária e Suficiente & A condição é suficiente e necessária

B \(\Rightarrow\) Y \\
Necessária, mas não suficiente & O resultado ocorre apenas na presença da condição, mas a condição sozinha não consegue explicar o resultado

B + T * B + c = B * ( T + c ) \(\Rightarrow\) Y

\textasciitilde B \(\Rightarrow\) \textasciitilde Y \\
Suficiente, mas não necessária & O resultado pode ser produzido pela condição, mas também pode ser produzida pela combinação de outras condições

B + T*c \(\Rightarrow\) Y \\
Nem necessária, nem suficiente & A condição só produz o resultado quando acompanhada de outras

B*c + C*T + B*T \(\Rightarrow\) Y \\
\bottomrule
\end{longtable}
\bcenter

Fonte: Adaptado de \autocite{betarelli_junior_introducao_2018}
\ecenter

Uma forma de identificar essas relações é a partir de plots específicos, nos quais a área aonde os pontos se encontram revelam qual das situações é mais presente na relação entre condição e \emph{outcome}. De forma geral em gráficos de Suficiência a relação existe quando a maioria dos dados se encontra na parte superior do gráfico, para a Necessidade vale o oposto \autocite{legewie2013introduction}.

Como dito anteriormente, para além das relações de necessidade e suficiência existem outras duas que também são levadas em consideração no processo: Consistência e Cobertura. De forma geral pode-se definir essas medidas como as que descrevem \emph{``(\ldots) the strength of the empirical support for theoretical arguments describing set relations''} \autocite[292]{ragin_set_2006}, sendo a Consistência a medida que os casos possuem condições ou combinação de condições que suportem a presença do outcome e Cobertura mede a relevância empírica de uma combinação \autocite[292]{ragin_set_2006}. Diferentes dos outros conceitos aqui comentados, esses podem ser calculados sendo a Consistência a principal medida na análise, dado que é o parâmetro que define se uma combinação gera ou não o resultado.

O cálculo da Consistência varia segundo o modelo de conjunto pela qual se opta. Nos casos de Conjuntos Crisp isso pode ser calculado pela seguinte fórmula:

\[ C_{x} = \frac{N_{casos} \quad \mbox {aonde} \quad X = 1 \; e \;  Y = 1}{N_{casos} \quad \mbox{aonde} \quad X = 1} \]
Na qual a consistência de uma condição X em relação ao \emph{outcome} Y é dada pela divisão entre o número de casos aonde X e Y são 1 pelo de casos aonde apenas X é 1. Para conjuntos \acrshort{Fuzzy}, aonde os valores são variações entre 0 e 1, esse valor é obtido pela divisão do somatório dos valores mínimos de adesão entre a condição X e o resultado Y da linha pelo somatório dos valores de X da linha. A fórmula é a seguinte:

\[ C_{Xi\leq Yi} = \frac{\sum (min(X_{i}, Y_{i}))}{\sum X_{i}} \]

Já a cobertura, como é baseada no resultado, é calculada da seguinte forma:

\[ C_{x} = \frac{N_{casos} \quad \mbox {aonde} \quad X = 1 \; e \;  Y = 1}{N_{casos} \quad \mbox{aonde} \quad Y = 1} \]

Na qual a cobertura de uma condição X em relação ao outcome Y é dada pela divisão entre o número de casos aonde X e Y são 1 e o número de casos aonde apenas Y é 1. Para casos Fuzzy o cálculo é efetuado da seguinte forma:

\[ C_{Xi\leq Yi} = \frac{\sum (min(X_{i}, Y_{i}))}{\sum Y_{i}} \]

Aonde o resultado é obtido pelo valor da divisão do somatório dos valores mínimos de adesão entre a condição X e o outcome Y da linha pelo somatório dos valores de Y da linha.

\hypertarget{anuxe1lises-descritivas}{%
\chapter{Análises Descritivas}\label{anuxe1lises-descritivas}}

Após discutir sobre a teoria que embasa essa pesquisa e sobre a metodologia que irei utilizar para tratar e interpretar os dados, resta demonstrar os resultados obtidos. 32 pessoas responderam o questionário e, neste capítulo, irei fazer uma análise descritiva básica, de forma a entender melhor quem são, de onde vieram e como foi a relação deles com as drogas e o tratamento.

\hypertarget{amostra}{%
\section{Amostra}\label{amostra}}

Nesta pesquisa utilizou-se Amostragem Não-Probabilística, dado que não é possível ter acesso a toda população e não se sabe a probabilidade de uma pessoa escolhida aleatoriamente ser egressa de uma CT. A forma de encontrar respondentes foi a \emph{Intencional} \autocite[187]{babbie_practice_2015}, ou seja, ao invés de selecionar de forma aleatória dentro da população os indivíduos que se encaixavam no padrão exigido, foi-se feita uma busca ativa delas em lugares onde já se tinha conhecimento prévio de que elas estariam, tais como grupos de Facebook e Whatsapp direcionados a este público.

Esse tipo de amostra, apesar de ser conveniente ao estudo, tem como contraponto não ser capaz de generalização. O que se afirma com esse estudo não tem poder de representar todas as pessoas que se encaixam no perfil pesquisado.

\hypertarget{dados-socioeconuxf4micos-1}{%
\section{Dados Socioeconômicos}\label{dados-socioeconuxf4micos-1}}

Na primeira seção do questionário o respondente era convidado a responder uma série de perguntas que ajudassem a criar um perfil socioeconômico. Um dos primeiros dados que gostaria de salientar, antes de começar a descrição de fato, é o da localização dos respondentes. A imagem abaixo demonstra o quantitativo por região.
\begin{figure}[H]

{\centering \includegraphics[width=0.9\linewidth]{images/imagem22-1} 

}

\caption{Número de respondentes por Região}\label{fig:imagem22}
\end{figure}
\bcenter

Fonte: ECRIECT\footnote{Acrônimo para \emph{Efeitos da Conversão Religiosa entre Egressos de Comunidades Terapêuticas}}, 2021
\ecenter

A maioria dos respondentes são do Ceará (35\%), ficando em segundo lugar Minas Gerais (16\%) e em terceiro São Paulo (12\%). A única região da qual não se obtiveram respostas foi a Centro-Oeste.

Em relação a idade dos participantes grande parte tem entre 20 e 30 anos, sendo a menor idade 21 e a maior 56. As faixas de idade podem ser conferidas no gráfico abaixo:
\begin{figure}[H]

{\centering \includegraphics[width=0.8\linewidth]{images/imagem23-1} 

}

\caption{Faixas de Idade dos Respondentes}\label{fig:imagem23}
\end{figure}
\bcenter

Fonte: ECRIECT, 2021
\ecenter

Mais da metade dos respondentes são homens (58\%) e, no que diz respeito ao Estado Civil, 31\% são solteiros, nunca tendo sido casados. Ao observar este último dado por sexo, no entanto, é possível ver que esse padrão se aplica melhor aos homens. As mulheres pesquisadas são, em maioria, divorciadas.
\begin{figure}[H]

{\centering \includegraphics[width=0.8\linewidth]{images/imagem24-1} 

}

\caption{Relação entre Sexo e Estado Civil}\label{fig:imagem24}
\end{figure}
\bcenter

Fonte: ECRIECT\glsunset{ECRIECT}, 2021
\ecenter

Em relação a filhos, 31\% afirma não os ter e dentre os que tem o maior número é 2 (25\%). Apenas dois respondentes afirmaram ter mais de 4. No que diz respeito a contato apenas 2 afirmaram não ter mais nenhum.

No que diz respeito a trabalho todos os respondentes afirmaram atuar no setor de serviços, sendo Vendedor (20\%) e Autônomo (12\%) as profissões mais comuns. A tabela abaixo mostra todas as profissões fornecidas:
\begin{table}[H]

\caption{\label{tab:trab}Atual profissão dos respondentes}
\centering
\begin{tabular}[t]{ll}
\toprule
\textbf{Profissões} & \textbf{Frequências}\\
\midrule
Artesã & 1\\
Lavadeira & 1\\
Atendente & 1\\
Autonomo & 4\\
Auxiliar de Serviços & 1\\
\addlinespace
Caixa de Supermercado & 1\\
Conselheiro em Dependência Química & 1\\
Desempregado & 4\\
Secretária do lar & 2\\
Entregador & 1\\
\addlinespace
Funcionário público aposentado & 1\\
Mecânico & 1\\
Motorista de aplicativo & 2\\
Recepcionista & 1\\
Servente & 1\\
\addlinespace
Técnica de Enfermagem & 1\\
Terapeutico conselheireiro & 1\\
Vendedor & 6\\
NA/NR & 1\\
Total & 32\\
\bottomrule
\end{tabular}
\end{table}
\bcenter

Fonte: ECRIECT, 2021
\ecenter

Ao dividir todas essas profissões em subgrupos de serviços, guiados pelo modelo de divisão disposto pela \emph{Classificação Nacional de Atividades Econômicas} (CNAE) \autocite{ibge_pesquisa_2005}, é possível visualizar de forma mais acurada a variação nas atividades. Gostaria de destacar que uma parte dos entrevistados atua em CT's, logo criou-se um grupo específico para eles. O gráfico abaixo demonstra, enfim, esta divisão:
\begin{figure}[H]

{\centering \includegraphics[width=0.8\linewidth]{images/imagem25-1} 

}

\caption{Profissões divididas por Subgrupos de Serviços}\label{fig:imagem25}
\end{figure}
\bcenter

Fonte: ECRIECT, 2021
\ecenter

O último dado coletado nesta seção foi a idade com a qual o respondente começou a trabalhar. 84\% dos respondentes começaram a trabalhar antes dos 18 anos, sendo a menor idade 10 anos. Não obstante, ao observar esse dado por sexo percebe-se que, apesar dos homens serem a maioria entre os que trabalharam antes de atingir a maior idade, todas as mulheres entrevistadas passaram por esse processo.
\begin{figure}[H]

{\centering \includegraphics[width=0.8\linewidth]{images/imagem27-1} 

}

\caption{Relação entre Idade com a qual começou a trabalhar e Sexo}\label{fig:imagem27}
\end{figure}
\bcenter

Fonte: ECRIECT, 2021
\ecenter

\pagebreak

\hypertarget{dados-religiosos}{%
\section{Dados Religiosos}\label{dados-religiosos}}

Para além de entender as características socioeconômicas dos respondentes, o questionário também coletou dados da vida religiosa, em ordem de entender seus trânsitos, se são convertidos e quanto de sua vida é relacionado a religião. 59\% dos entrevistados alegaram, no momento da pesquisa, frequentar algum culto/religião. O gráfico abaixo demonstra as respostas que foram fornecidas:
\begin{figure}[H]

{\centering \includegraphics[width=0.8\linewidth]{images/imagem28-1} 

}

\caption{Religiões respondidas pelos participantes}\label{fig:imagem28}
\end{figure}
\bcenter

Fonte: ECRIECT, 2021
\ecenter

É possível perceber que a maioria é Protestante, sendo seguida pelos Sem Filiação e, por fim, Católicos. Nesta pesquisa, ao contrário do que vem sendo usado como padrão na coleta de dados quantitativos sobre religião no Brasil, utilizei uma pergunta \emph{double-step} para melhor capturar a filiação de religiões com mais de uma forma, no caso o protestantismo. O gráfico abaixo destrincha melhor as especificações coletadas.

Dentre os Protestantes a maioria frequenta a Assembleia de Deus ou alguma outra Pentecostal, poucos dizem pertencer a correntes mais históricas como a Presbiteriana.
\begin{figure}[H]

{\centering \includegraphics[width=0.8\linewidth]{images/imagem29-1} 

}

\caption{Ramificações protestantes respondidas pelos participantes}\label{fig:imagem29}
\end{figure}
\bcenter

Fonte: ECRIECT, 2021
\ecenter

Mesmo que a pergunta sobre Religião não fosse aberta a todos, a pergunta sobre frequência com que se vai a Igreja/Culto era. Dessa forma foi possível avaliar, mesmo se a pessoa não alegasse ter filiação, se ainda existia algum nível de contato com religião. O gráfico abaixo demonstra essa relação.
\begin{figure}[H]

{\centering \includegraphics[width=0.8\linewidth]{images/imagem30-1} 

}

\caption{Relações entre Frequência com a qual vai a Igreja e Religião}\label{fig:imagem30}
\end{figure}
\bcenter

Fonte: ECRIECT, 2021
\ecenter

Curiosamente, uma pequena parte dos que disseram ser Sem Filiação ainda frequentam a igreja aos fins de semana. Para além disso percebe-se que as maiores frequências se encontram entre os Protestantes.

Outra temática abordada durante o questionário foi a vivência de experiências sobrenaturais. Pelo menos 13 respondentes (43\%) afirmaram ter experienciado algo do tipo. No gráfico abaixo temos a relação deles:
\begin{figure}[H]

{\centering \includegraphics[width=0.8\linewidth]{images/imagem31-1} 

}

\caption{Experiências Religiosas/Sobrenaturais}\label{fig:imagem31}
\end{figure}
\bcenter

Fonte: ECRIECT, 2021
\ecenter

Ao ver como essas experiências se dividem por religião percebe-se que Católicos foram menos propensos a vivências desse tipo, enquanto pessoas Sem Filiação já afirmaram ter vivenciado algumas delas
\begin{figure}[H]

{\centering \includegraphics[width=0.8\linewidth]{images/imagem32-1} 

}

\caption{Experiências Religiosas/Sobrenaturais por Religião}\label{fig:imagem32}
\end{figure}
\bcenter

Fonte: ECRIECT, 2021
\ecenter

Um dos dados mais importantes desta seção foram as perguntas direcionadas a questão da Conversão Religiosa. A pergunta, como dito anteriormente, possui 4 níveis: ``Sou convertido'', ``Não sou convertido'', ``Estou em processo de Conversão'' e ``Não existe algo como conversão na minha religião'', dessa forma ela conseguia abarcar o espectro da questão de forma coesa. Seguem abaixo os resultados.
\begin{figure}[H]

{\centering \includegraphics[width=0.8\linewidth]{images/imagem33-1} 

}

\caption{Estados de Conversão}\label{fig:imagem33}
\end{figure}
\bcenter

Fonte: ECRIECT, 2021
\ecenter

Note que o número de convertidos e não convertidos é o mesmo e que número de pessoas que admitiram estar em um processo de conversão também foi relativamente alto. A opção com menos respondentes foi ``Não existe algo como Conversão na minha religião'' o que soa, num primeiro momento, estranho, dado que as duas religiões que aparecem como respostas admitem processos de Conversão. Ao observar esse dado por religião, no entanto, percebe-se que as pessoas que forneceram essa resposta são, majoritariamente, Católicas. Isso ajuda a endossar a visão de que a identidade católica é, pelo menos entre uma parte dos entrevistados, não uma opção ou parte de um processo, mas um elemento não necessariamente adesional, advindo da tradição \autocite{pierucci_religiao_2006}. Outro ponto a ser destacado é a presença dos Sem Filiação na opção ``Estou em processo de Conversão'', o que demonstraria que conseguiu-se captar a adoção de modelos de vivência religiosa mais fluidos e menos institucionalizados.
\begin{figure}[H]

{\centering \includegraphics[width=0.8\linewidth]{images/imagem34-1} 

}

\caption{Estados de Conversão por Religião}\label{fig:imagem34}
\end{figure}
\bcenter

Fonte: ECRIECT, 2021
\ecenter

Às pessoas que responderam ser convertidas ou estarem no processo foi-se perguntado aonde esse processo se iniciou. 62.5\% delas afirmaram que esse processo se iniciou dentro da Comunidade Terapêutica que deram entrada. Um dado expressivo que ajuda a apoiar a tese levantada neste estudo de que as CT's promovem esse tipo de processo. Um outro dado encontrado foi que alguns dos Sem Filiação foram, em algum momento, convertidos e esse processo também aconteceu, em parte, dentro de CT's.
\begin{figure}[H]

{\centering \includegraphics[width=0.8\linewidth]{images/imagem35-1} 

}

\caption{Local de Conversão por Religião}\label{fig:imagem35}
\end{figure}
\bcenter

Fonte: ECRIECT, 2021
\ecenter

Outro dado coletado durante a pesquisa foi relacionado as frequências com que os respondentes faziam atividades exclusivamente religiosas ou que envolvessem, em algum nível, religião. Foram dadas 4 opções: Orar, Ler a Bíblia, Conversar com pessoas ao seu redor sobre fé e Ir para outras atividades oferecidas pela igreja que você frequenta. Os resultados se encontram abaixo:
\begin{figure}[H]

{\centering \includegraphics[width=0.8\linewidth]{images/imagem36-1} 

}

\caption{Atividades Religiosas praticadas pelos respondentes}\label{fig:imagem36}
\end{figure}
\bcenter

Fonte: ECRIECT, 2021
\ecenter

Vendo por religião é possível perceber alguns padrões intrigantes. Entre os católicos as atividades mais praticadas são a Oração e a Conversa sobre a Fé. Ler a Bíblia e frequentar atividades oferecidas pela igreja são as atividades com menor frequência. Já entre os Protestantes todas as atividades propostas pela questão são praticadas em algum nível, sendo Orar e Ler a Bíblia as com maior intensidade, os poucos ``nunca'' respondidos foram para atividades como ``conversar sobre a fé'' e ``ir para atividades extras da igreja que frequenta. Entre os Sem Filiação''orar" foi a única atividade feita com maior frequência. Apesar disso outras atividades como Conversar sobre a Fé e Ler a Bíblia também foram elencadas como feitas ``às vezes''.
\begin{figure}[H]

{\centering \includegraphics[width=0.8\linewidth]{images/imagem37-1} 

}

\caption{Atividades Religiosas praticadas pelos respondentes por Religião}\label{fig:imagem37}
\end{figure}
\bcenter

Fonte: ECRIECT, 2021
\ecenter

Para além de entender filiação, frequência religiosa e nível de cometimento com elementos religiosos também se procurou atentar a crença em elementos religiosos, de forma a capturar o nível de aceitação de elementos ditos teológicos. De forma simples era mostrado aos respondentes alguns elementos referentes ao campo religioso e eles deveriam responder se acreditavam ou não. O gráfico abaixo dispõe dos resultados.
\begin{figure}[H]

{\centering \includegraphics[width=0.8\linewidth]{images/imagem38-1} 

}

\caption{Crenças em elementos teológicos}\label{fig:imagem38}
\end{figure}
\bcenter

Fonte: ECRIECT, 2021
\ecenter

O elemento do campo religioso que mais se acredita é Deus (93\%) e o que menos se crê são Milagres (41\%). Ao observar esse comportamento por religião percebe-se que os Sem Filiação são os que possuem os maiores níveis de descrença, sendo os únicos entre os entrevistados a não acreditarem em Deus. Católicos foram os únicos que manifestaram níveis totais de crença em elementos além de Deus, como Vida após a morte e demônios. Apesar disso, entre os religiosos, eles foram os menos acreditaram em milagres.
\begin{figure}[H]

{\centering \includegraphics[width=0.8\linewidth]{images/imagem39-1} 

}

\caption{Crenças em elementos teológicos por Religião}\label{fig:imagem39}
\end{figure}
\bcenter

Fonte: ECRIECT, 2021
\ecenter  

\pagebreak

\hypertarget{histuxf3rico-de-uso-abusivo}{%
\section{Histórico de Uso Abusivo}\label{histuxf3rico-de-uso-abusivo}}

Uma das últimas seções do questionário se dedicou a coletar dados sobre o antes, durante e depois do uso abusivo de drogas. Visando capturar as relações sociais dos sujeitos antes e depois desse período, características das CT's aonde fizeram passagens e como eles se encontram atualmente. Um dos primeiros pontos da seção foi o quantitativo de drogas com a qual o entrevistado teve contato. Segue abaixo a frequência com a qual elas foram citadas.
\begin{figure}[H]

{\centering \includegraphics[width=0.8\linewidth]{images/imagem40-1} 

}

\caption{Substâncias utilizadas pelos entrevistados}\label{fig:imagem40}
\end{figure}
\bcenter

Fonte: ECRIECT, 2021
\ecenter

Os respondentes usavam durante o período de uso abusivo, em média, 4 substâncias diferentes. O maior número obtido foi 13. No que diz respeito ao grau de vício optei por uma escala de auto declaração de 0 a 10. A média obtida foi 9,03 o que indica que os respondentes acreditavam ser extremamente dependentes no período de uso. Até mesmo a mediana, medida mais robusta a presença de \emph{outliers}, foi 9. Uma comparação entre o número de substâncias utilizadas e a nota dada a si mesmo nessa escala é oferecida abaixo:
\begin{figure}[H]

{\centering \includegraphics[width=0.8\linewidth]{images/imagem41-1} 

}

\caption{Comparação entre Número de Substâncias utilizadas e Nota na Escala de Adicção}\label{fig:imagem41}
\end{figure}
\bcenter

Fonte: ECRIECT, 2021
\ecenter

Note que, curiosamente, a pessoa com a maior variação de uso (13 substâncias) foi a que se deu a menor nota (5) na escala de vício. Apesar disso, é possível perceber que quase sempre a nota é alta, independentemente do número de substâncias utilizadas.

Ainda nesta seção tentou-se capturar características das CT's nas quais os respondentes deram entrada. De modo geral a idade média dos entrevistados ao dar entrada era de 25 anos e pelo menos 50\% fez apenas uma entrada, o que indica que a outra parte passou por processos de rotatividade. Foi-se perguntado também de qual forma os entrevistados ficaram sabendo da Comunidade que residiram. As respostas se encontram abaixo:
\begin{figure}[H]

{\centering \includegraphics[width=0.8\linewidth]{images/imagem42-1} 

}

\caption{Formas como os respondentes ficaram sabendo da última Comunidade Terapêutica que deram entrada}\label{fig:imagem42}
\end{figure}
\bcenter

Fonte: ECRIECT, 2021
\ecenter

Percebe-se que a forma mais comum foi através de familiares (43\%), seguido de ações da própria CT (36\%) e amigos (11\%). 89\% dos respondentes afirmaram que a Comunidade que deram entrada se auto intitulava religiosa, o que coaduna com os dados divulgados pelo IPEA previamente discutidos. O gráfico abaixo descreve à quais religiões essas CT's eram filiadas. A maioria é de Protestantes da linhagem Pentecostal (76\%), seguido de Protestantes Históricos (12\%) e Católicos (8\%). A presença de Comunidades de credos diferentes dos majoritários, no caso Espírita, também foi capturada, mesmo que em menor quantidade (3,5\%). Note que uma parte dos respondentes afirmaram vir de CT's não religiosas. Para todos os efeitos essas Comunidades serão chamadas de ``Laicas'', o que será provado, em breve, falso, dado que até mesmo essas possuem comportamentos religiosos.
\begin{figure}[H]

{\centering \includegraphics[width=0.8\linewidth]{images/imagem43-1} 

}

\caption{Religião a qual a CT religiosa era filiada}\label{fig:imagem43}
\end{figure}
\bcenter

Fonte: ECRIECT, 2021
\ecenter

Também foram sondadas a presença ou ausência de alguns dos elementos terapêuticos geralmente aplicados pelas CT's: \emph{Cultos/Missas}, \emph{Espiritualidade} e \emph{Acompanhamento Psicológico/Psiquiátrico}. Os resultados encontram-se abaixo:
\begin{figure}[H]

{\centering \includegraphics[width=0.8\linewidth]{images/imagem44-1} 

}

\caption{Tipos de elementos terapêticos empregados}\label{fig:imagem44}
\end{figure}
\bcenter

Fonte: ECRIECT, 2021
\ecenter

É perceptível que Cultos e Missas, assim como Espiritualidade, são elementos quase unânimes, enquanto acompanhamento psicológico aparenta ser menos presente. Ao ver como essas divisões se comportam quando confrontadas com a religião à qual a CT é filiada percebe-se que Católicos, Espíritas e Laicos forneceram tanto elementos religiosos quanto clínicos, enquanto as Protestantes ofereciam mais religiosos do que clínicos. As Pentecostais foram as que demonstraram maior ausência de acompanhamento psicológico/psiquiátrico.
\begin{figure}[H]

{\centering \includegraphics[width=0.8\linewidth]{images/imagem45-1} 

}

\caption{Tipos de elementos terapêticos empregados por Religião a qual a CT é filiada}\label{fig:imagem45}
\end{figure}
\bcenter

Fonte: ECRIECT, 2021
\ecenter

Uma prática comum em CT's é a atribuição de funções para os internos. Tais atividades variam de coisas comuns do dia a dia a cargos de confiança dentro da residência, tais como Liderança e Auxílio em questões administrativas. O gráfico abaixo demonstra a distribuição geral destas entre os entrevistados. Descobriu-se também que alguns dos respondentes chegavam a cumprir tarefas que divergiam do \emph{milieu} terapêutico convencional, como serviços de construção e conserto.
\begin{figure}[H]

{\centering \includegraphics[width=0.8\linewidth]{images/imagem46-1} 

}

\caption{Divisão de Atividades e Cargos entre os Internos}\label{fig:imagem46}
\end{figure}
\bcenter

Fonte: ECRIECT, 2021
\ecenter

Uma divisão por sexo consegue contrastar melhor a distribuição dessas atividades. Nenhuma mulher que participou da pesquisa ocupou cargo de Liderança ou Administrativo. As atividades mais comuns entre este grupo eram as de auxiliar e cozinheira e, apesar de haver mais homens do que mulheres nessa pesquisa, são elas que dominam as funções relacionadas a cozinha. O gráfico abaixo demonstra essa divisão.
\begin{figure}[H]

{\centering \includegraphics[width=0.8\linewidth]{images/imagem47-1} 

}

\caption{Divisão de Atividades e Cargos entre os Internos por Sexo}\label{fig:imagem47}
\end{figure}
\bcenter

Fonte: ECRIECT, 2021
\ecenter

Após todas estas especificações sobre a CT e uso abusivo de drogas, resta saber mais sobre como se encontra a vida dos respondentes atualmente. 34\% dos respondentes não apresentaram relapsos e estavam, até a data que responderam a pesquisa, em estado de abstinência, o que implica que pelo menos dois terços dos entrevistados utilizaram alguma das substâncias que apareceram nas opções em algum momento após sua saída da CT. O gráfico abaixo dispõe das respostas oferecidas e suas eventuais frequências.
\begin{figure}[H]

{\centering \includegraphics[width=0.8\linewidth]{images/imagem48-1} 

}

\caption{Drogas consumidas no período Pós CT}\label{fig:imagem48}
\end{figure}
\bcenter

Fonte: ECRIECT, 2021
\ecenter

Perceba que as drogas que disparam no rank são Álcool e Cigarro, mais fáceis de serem consumidas, dado que são lícitas. Já entre as ilícitas Maconha e Cocaína são as mais utilizadas, em último lugar ficou o Crack. Tanto para os que apresentaram recaída quanto para os que permaneceram abstêmios foi-se perguntado qual o principal motivador para tal estado. Os resultados se encontram abaixo:
\begin{figure}[H]

{\centering \includegraphics[width=0.8\linewidth]{images/imagem49-1} 

}

\caption{Motivos para Abstêmia fornecidos pelos entrevistados}\label{fig:imagem49}
\end{figure}
\bcenter

Fonte: ECRIECT, 2021
\ecenter
\begin{figure}[H]

{\centering \includegraphics[width=0.8\linewidth]{images/imagem49-1-1} 

}

\caption{Motivos para Recaída fornecidos pelos entrevistados}\label{fig:imagem49-1}
\end{figure}
\bcenter

Fonte: ECRIECT, 2021
\ecenter

Percebe-se em ambos os grupos que elementos religiosos costumam ser os menos citados. No topo dos principais motivos para abstêmia se encontra a \emph{Ajuda de amigos e familiares} e no topo dos motivos para recaída se encontra a \emph{Falta de vontade própria}. Isso pode ser um indicativo de que parece existir uma tendência entre os que sofreram recaída de utilizar elementos pessoais como explicadores e entre os abstêmios, as redes sociais. Ao ver como essa variação acontece por Sexo percebe-se que, entre mulheres abstêmias, \emph{Filhos} e \emph{Conversão Religiosa} foram os com maior impacto, enquanto homens se apoiaram mais na \emph{Ajuda de Amigos e Familiares} e na \emph{Vontade Própria}. Já entre as mulheres que recaíram a \emph{Falta de Apoio a Amigos e Familiares} e a \emph{Falta de Vontade Própria} foram os fatores mais apontados, enquanto para os homens \emph{Distanciamento dos Filhos} e \emph{Falta de Vontade Própria} desempenharam maior papel. Essas distribuições podem ser visualizadas nos gráficos abaixo:
\begin{figure}[H]

{\centering \includegraphics[width=0.8\linewidth]{images/imagem50-1} 

}

\caption{Motivos para Abstêmia por Sexo}\label{fig:imagem50}
\end{figure}
\bcenter

Fonte: ECRIECT, 2021
\ecenter
\begin{figure}[H]

{\centering \includegraphics[width=0.8\linewidth]{images/imagem51-1} 

}

\caption{Motivos para Recaída por Sexo}\label{fig:imagem51}
\end{figure}
\bcenter

Fonte: ECRIECT, 2021
\ecenter

Para além de entender motivações pessoais para o estado de abstinência procurou-se sondar as características das redes sociais do respondente. Um dos primeiros elementos a ser perguntado foi sobre a presença da família no percurso de uso abusivo. Foi-se questionado como era a relação do respondente com a sua família durante o período de uso abusivo e atualmente. Foram fornecidas três categorias: Conflituosa (C), Harmônica (H) e Inexistente (I). O gráfico abaixo retrata o quantitativo destas trajetórias sendo a primeira letra a relação \textbf{durante} o período de uso abusivo e a segunda letra \textbf{após} o tratamento, ou seja, atualmente.
\begin{figure}[H]

{\centering \includegraphics[width=0.8\linewidth]{images/imagem52-1} 

}

\caption{Trajeto da Relação Familiar do período de Uso Abusivo até o momento da pesquisa}\label{fig:imagem52}
\end{figure}
\bcenter

Fonte: ECRIECT, 2021
\ecenter

De modo geral, a maioria dos respondentes saiu do Conflito para a Harmonia (C-H) após o tratamento, com destaque para a existência de casos aonde ter feito o tratamento reatou as relações familiares (I-H). A menor frequência foram casos em que, mesmo após o tratamento, não foi possível reaver a relação familiar, logo o caráter inexistente (C-I).

A presença de outros círculos sociais na vida do respondente, como amigos do período de uso abusivo e pessoas da própria Comunidade que o respondente deu entrada, também foi sondada. Percebe-se que, no geral, as amizades feitas durante o período de uso abusivo não tendem a sobreviver o período de internação, já as feitas no âmbito da CT foram, em grande parte, preservadas. O gráfico abaixo demonstra esta relação:
\begin{figure}[H]

{\centering \includegraphics[width=0.8\linewidth]{images/imagem53-1} 

}

\caption{Laços de amizade entre o respondente e pessoas que conheceu no período de Uso Abusivo e Acolhimento na CT}\label{fig:imagem53}
\end{figure}
\bcenter

Fonte: ECRIECT, 2021
\ecenter

Também foi questionado, no quesito das relações sociais, quais pessoas ofereceram apoio ao respondente quando ele deixou a CT. Foram oferecidas como possibilidades: Família, Amigos, Pessoas da Igreja, Vizinhos, Outros egressos e Grupos de Auto Ajuda. O resultado se encontra no gráfico abaixo:
\begin{figure}[H]

{\centering \includegraphics[width=0.8\linewidth]{images/imagem54-1} 

}

\caption{Grupos que ofereceram apoio aos respondantes após a saída da CT}\label{fig:imagem54}
\end{figure}
\bcenter

Fonte: ECRIECT, 2021
\ecenter  

Percebe-se que Amigos e Grupos de Auto Ajuda são os que mais ofereceram apoio aos respondentes, Vizinhos e Pessoas da Igreja são as que ofereceram menos apoio.

A última pergunta do questionário, e também um dos grandes propósitos deste estudo, era sobre o quanto o respondente acreditava estar atualmente recuperado. Ao invés de utilizar escala psicométrica optei por usar uma escala de auto declaração que ia de 1 a 10. A média obtida foi de 7.5 e a mediana foi 8, o que indica que, no geral, os respondentes parecem estar em um nível ótimo, quase excelente, de recuperação. Essas medidas, no entanto, sofrem algumas variações ao serem contrastadas com outras variáveis. Mulheres tiveram, em média, níveis de recuperação maiores (\(\mu\) = 8.2). Quando visto por religião, os valores médios obtidos por Protestantes e Sem Filiação são bastante próximos (7.8 e 7.4, respectivamente). Por credo da CT percebe-se que as Católicas e Evangélica Histórica obtiveram os maiores valores (\(\mu\) = 9) e as Evangélicas Pentecostais obtiveram a menor colocação (\(\mu\) = 7.2). Tais relações podem ser percebidas de forma mais clara nos gráficos abaixo:
\begin{figure}[H]

{\centering \includegraphics[width=0.8\linewidth]{images/imagem55-1} 

}

\caption{Nível de recuperação e Opção Religiosa}\label{fig:imagem55}
\end{figure}
\bcenter

Fonte: ECRIECT, 2021
\ecenter  
\begin{figure}[H]

{\centering \includegraphics[width=0.8\linewidth]{images/imagem56-1} 

}

\caption{Nível de recuperação e Sexo}\label{fig:imagem56}
\end{figure}
\bcenter

Fonte: ECRIECT, 2021
\ecenter
\begin{figure}[H]

{\centering \includegraphics[width=0.8\linewidth]{images/imagem57-1} 

}

\caption{Nível de recuperação e Credo da CT}\label{fig:imagem57}
\end{figure}
\bcenter

Fonte: ECRIECT, 2021
\ecenter

Apesar de ser apenas um contraste de médias percebe-se que a Religião não parece ser um elemento que provoca sempre uma melhora na recuperação do respondente. Ao se contrastar com a frequência com a que se vai a igreja, por exemplo, percebe-se que ``Apenas no fim de semana'' e ``Não frequento ou visito a Igreja'' tem as maiores médias de recuperação. Isso também vale para o nível de religiosidade, pessoas que se deram notas 0 e 10 obtiveram maiores médias de recuperação (\(\mu\) = 9.3).
\begin{figure}[H]

{\centering \includegraphics[width=0.8\linewidth]{images/imagem90-1} 

}

\caption{Nível de recuperação e Frequência com que vai a Igreja}\label{fig:imagem90}
\end{figure}
\bcenter

Fonte: ECRIECT, 2021
\ecenter

\pagebreak

\hypertarget{estilos-de-tratamento-e-adesuxe3o-a-religiosidade}{%
\section{Estilos de Tratamento e Adesão a Religiosidade}\label{estilos-de-tratamento-e-adesuxe3o-a-religiosidade}}

O último tópico desta seção diz respeito aos dados coletados sobre o tratamento em si, em especial as atitudes que os respondentes têm em relação ao papel da religião no processo de tratamento e recuperação.

No que diz respeito ao estilo de tratamento, foi-se perguntado qual tipo era mais recomendado pelo respondente: um baseado em Religião/Espiritualidade ou um baseado em Redução de Danos. Perguntas desse tipo costumam refletir bem o apreço que se coloca em relação ao item questionado. A primeira opção foi a preferida entre os respondentes sendo a mediana 6, enquanto a segunda teve como mediana 5. Apesar da distância entre os valores não ser muito grande é perceptível uma leve preferência pelo primeiro. Essa medida não mostrou muitas disparidades ao ser contrastada por sexo, ao se contrastar por Opção Religiosa, no entanto, percebe-se uma possível tendência. O gráfico abaixo demonstra tais disposições:

Percebe-se que religiosos tendem a valorizar mais tratamentos com inclusão de elementos religiosos do que as pessoas Sem Filiação. Isso fica claro ao observar a queda drástica que acontece na nota dos Sem Filiação sobre tratamentos puramente religiosos (4.1) e na eventual subida na nota sobre Redução de Danos (4.6).
\begin{figure}[H]

{\centering \includegraphics[width=0.8\linewidth]{images/imagem58-1} 

}

\caption{Relação entre Nota para tipos de tratamento e Opção religiosa}\label{fig:imagem58}
\end{figure}
\bcenter

Fonte: ECRIECT, 2021
\ecenter

Ainda no debate sobre a opinião dos respondentes em relação ao tratamento foram feitas onze perguntas em estilo \emph{Likert} de 5 níveis (Concordo Totalmente, Concordo em parte, Não concordo nem discordo, Discordo em partes, Discordo totalmente) sobre o tratamento e os possíveis impactos de uma vida longe da igreja e de elementos religiosos no estado de abstêmia. Todas as perguntas foram processadas de forma a criarem no final um construto latente que demonstrasse atitudes referentes a religião ou laicidade no contexto do tratamento e da recuperação. Para tal desiderato foi-se feito, inicialmente, uma Análise de Confiabilidade em ordem de saber se os itens realmente mediam o que se esperava. Nessa análise é calculado o \emph{Alpha de Crombach} dos itens, a partir de tal cálculo pode-se saber se os itens variam juntos e o quanto dessa variância pode ser explicada numa possível união das variáveis em processos de redução \autocite{cronbach_coefficient_1951}. O resultado se encontra na tabela abaixo:
\begin{longtable}[]{@{}
  >{\raggedright\arraybackslash}p{(\columnwidth - 4\tabcolsep) * \real{0.76}}
  >{\raggedright\arraybackslash}p{(\columnwidth - 4\tabcolsep) * \real{0.16}}
  >{\raggedright\arraybackslash}p{(\columnwidth - 4\tabcolsep) * \real{0.08}}@{}}
\caption{\label{tab:alpha-religiao} \emph{Alpha de Crombach} dos itens utilizados na construção da variável}\tabularnewline
\toprule
Item & Natureza da Correlação & \(     
                                                                                                                                              \alpha  
                                                                                                                                              \) \\
\midrule
\endfirsthead
\toprule
Item & Natureza da Correlação & \(     
                                                                                                                                              \alpha  
                                                                                                                                              \) \\
\midrule
\endhead
É importante a presença de religião/espiritualidade durante tratamento & Positiva & \textbf{0.87} \\
É importante a presença de acompanhamento psicológico/psiquiátrico durante tratamento & Negativa & 0.29 \\
É importante que a pessoa se converta a alguma crença/religião durante o tratamento. & Positiva & \textbf{0.91} \\
Uma pessoa que não se converter a alguma crença/religião durante o tratamento tem mais chances de sofrer recaída. & Positiva & \textbf{0.93} \\
Uma pessoa que tem amigos religiosos têm menos chances de sofrer recaída & Positiva & 0.67 \\
Uma pessoa que não tenha força de vontade tem mais chances de sofrer recaída. & Negativa & 0.21 \\
Ir a igreja me ajudou a não sofrer recaída & Positiva & \textbf{0.89} \\
Ter me convertido ajudou a não sofrer recaída & Positiva & \textbf{0.75} \\
Ter amigos religiosos me ajudou a não sofrer recaída. & Positiva & \textbf{0.86} \\
Uma pessoa que fez seu tratamento numa clínica ou comunidade não-religiosa tem mais chances de sofrer recaída & Positiva & \textbf{0.85} \\
Uma pessoa que não frequenta a igreja após o tratamento tem mais chances de sofrer recaída. & Positiva & \textbf{0.93} \\
\bottomrule
\end{longtable}
\bcenter

Fonte: ECRIECT, 2021
\ecenter

Pecebe-se pela tabela que os dois itens com carga negativa são os que versam sobre elementos laicos (atendimento psicológico e psiquiátrico e força de vontade) enquanto todos os itens que envolvem religião possuem carga positiva. Todos os itens com \acrshort{carga} inferior a 0.7 (não significativos) foram eliminados em ordem de fazer o construto mais potente, com exceção do 5\textordmasculine que, após a eliminação dos negativos, conseguiu atingir carga de 0.7. Ao final a análise obteve um \(\alpha\) de 0.96 que é mais que significativo e indica que as variáveis podem representar um construto. Após a análise os itens foram somados e o resultado desse somatório foi utilizado como \emph{Índice de Apoio a Religiosidade} ou IAR.

Nesse índice usarei como guia os próprios quartis, de forma que todas as pessoas com pontuações entre o 3\textordmasculine e o 4\textordmasculine quantis serão consideradas maiores apoiadoras da presença de religiosidade no contexto do tratamento/recuperação enquanto os 1\textordmasculine e 2\textordmasculine quartis serão os dos que apoiam de forma menos intensa ou não dão apoio. O gráfico abaixo demonstra essa divisão final:
\begin{figure}[H]

{\centering \includegraphics[width=0.8\linewidth]{images/imagem59-1} 

}

\caption{Distribuição do IAR}\label{fig:imagem59}
\end{figure}
\bcenter

Fonte: ECRIECT, 2021
\ecenter

É possível perceber que grande parte dos respondentes (30\%) se localizam no primeiro quartil e uma minoria se encontra no último (22\%), o que indica que grande parte destes possuem posicionamentos fracos ou moderados no que diz respeito a Atitudes Religiosas no contexto do tratamento/recuperação. Com o índice em mãos fica mais fácil ver esse comportamento contrastado com outras variáveis do questionário.

No que diz respeito a religião o IAR é extremamente alto entre Protestantes e baixo nas outras categorias. Ao se comparar por Estado de Conversão percebe-se que ele é menor entre não convertidos e pessoas que afirmaram não existir processos de conversão em sua religião atual. Ao checar por religião da Comunidade percebe-se que pessoas advindas de CT's Católicas são os que menos apoiam a ideia de religião como suporte, sendo todos os outros (inclusive os advindos de CT ``Laica'') propensos a essa noção. Os gráficos abaixo demonstram essas relações. A linha azul no gráfico representa a média geral do IAR.
\begin{figure}[H]

{\centering \includegraphics[width=0.8\linewidth]{images/imagem60-1} 

}

\caption{Média do IAR por Estado de Conversão}\label{fig:imagem60}
\end{figure}
\bcenter

Fonte: ECRIECT, 2021
\ecenter
\begin{figure}[H]

{\centering \includegraphics[width=0.8\linewidth]{images/imagem61-1} 

}

\caption{Média do IAR por Opção Religiosa}\label{fig:imagem61}
\end{figure}
\bcenter

Fonte: ECRIECT, 2021
\ecenter
\begin{figure}[H]

{\centering \includegraphics[width=0.8\linewidth]{images/imagem63-1} 

}

\caption{Média do IAR por Credo da CT}\label{fig:imagem63}
\end{figure}
\bcenter

Fonte: ECRIECT, 2021
\ecenter

Um último dado, antes de prosseguir para a seção de Resultados propriamente dita, é a relação entre o \emph{IAR} e noção de auto importância. Percebe-se que, dentre os vários motivos citados para manutenção da abstêmia, ou da falta dela, a questão da Vontade Própria parece ter um grande papel. Pensando nisso, construiu-se uma pergunta com a intenção de avaliar qual a importância que o respondente dava a si mesmo no processo de recuperação. 85\% dos respondentes avaliou a sua própria importância como ``muito importante''. Ao contrastar as respostas desse item com o \emph{IAR} percebe-se um padrão, no mínimo, curioso. Todas as pessoas que avaliaram a sua importância com outros itens além de ``muito'' possuem altos valores no IAR, o que pode ser um indicativo de que, muito provavelmente, indivíduos com altas pontuações no IAR tendem a confiar mais na religião e nos feitos divinos do que em si mesmo. Observe o padrão no gráfico abaixo, a linha azul representa o valor médio do IAR.
\begin{figure}[H]

{\centering \includegraphics[width=0.8\linewidth]{images/imagem64-1} 

}

\caption{Relação entre IAR e Auto Importância}\label{fig:imagem64}
\end{figure}
\bcenter

Fonte: ECRIECT, 2021
\ecenter  

\hypertarget{resultados-e-discussuxe3o}{%
\chapter{Resultados e Discussão}\label{resultados-e-discussuxe3o}}

A análise descritiva dos dados ajudou a entender mais sobre as pessoas que responderam o questionário e a formular um perfil mais aprofundado sobre elas. Isso, no entanto, não ajuda a responder quais impactos a Conversão Religiosa teve sobre Abstêmia e Recuperação e, caso tenha, como eles acontecem. Para que isso seja possível uma análise mais aprofundada se faz necessária, no caso a \emph{QCA}. Essa seção do texto tratará de explicar como a análise foi feita, os resultados e uma discussão de como esses resultados se manifestam nos casos analisados e na literatura. Uma primeira parte irá explicar os critérios de Calibração, após isso será feita a análise e os resultados serão mostrados e, por fim, uma discussão será tecida. Apenas 2 condições\footnote{Apesar da argumentação das hipóteses apontarem outros processos, eles não poderiam ser utilizados em análises \emph{fuzzy}, dado que não podem ser labelados em níveis. Contato com os filhos, por exemplo, não foi uma variável explorada o suficiente para construir um arcabouço possível de fuzzificação. Apenas variáveis que suportariam esse processo foram utilizadas.}, como disposto nas hipóteses, foram utilizadas nas análises: A \emph{Conversão} e as \emph{Redes Sociais}.

\hypertarget{calibrauxe7uxe3o}{%
\section{Calibração}\label{calibrauxe7uxe3o}}

\hypertarget{conversuxe3o}{%
\subsection{Conversão}\label{conversuxe3o}}

Um dos primeiros obstáculos na construção da análise foi estabelecer os limites de pertencimento dos dados, dado que não existem pesquisas que delimitem de forma específica os níveis de pertencimento dos objetos estudados. Não existem pesquisas que informem, por exemplo, o quanto alguem pode ser considerada mais ou menos convertido, ou que informem o quanto de apoio social uma pessoa precisa para ser considerada bem assegurada. Sabendo disso dois métodos foram utilizados para estabelecer limites confiáveis. O primeiro foi o desenvolvimento de um rankeamento para os dados utilizados, o segundo foi uma técnica de \emph{clustering} para ver como os dados podem ser separados de forma ótima.

A questão sobre Conversão foi dividida em 4 níveis. ``Sim, sou convertido'', ``Não, não sou convertido'', ``Não existe algo como conversão na minha religião'' e ``Ainda estou em processo de Conversão''. Apesar desses estados não serem numéricos eles possuem em si um ordenamento lógico. Uma pessoa que está em processo de conversão está em um nível mais avançado do que alguém que não é convertido ou que está em uma religião que não necessita de conversão. Este último está um pouco acima do primeiro, mesmo que ambos compartilhem do espectro da não conversão. Dentro dessa lógica os dados foram rankeados pelos limites de pertencimento e o rankeamento foi processado por dois algoritmos de calibração. O primeiro foi o \emph{TFR} (Totally Fuzzy and Relative), criado especialmente para variáveis categóricas e ordinais \autocite[97]{dusa_qca_2018} e o segundo foi a Função Logística com \emph{thresholders} gerados a partir de Análise de Cluster. Ambos os algoritmos estão disponíveis no pacote \textbf{QCA} \autocite{QCA}. Os resultados se encontram na tabela abaixo:
\begin{longtable}[]{@{}
  >{\raggedright\arraybackslash}p{(\columnwidth - 6\tabcolsep) * \real{0.50}}
  >{\raggedright\arraybackslash}p{(\columnwidth - 6\tabcolsep) * \real{0.16}}
  >{\raggedright\arraybackslash}p{(\columnwidth - 6\tabcolsep) * \real{0.17}}
  >{\raggedright\arraybackslash}p{(\columnwidth - 6\tabcolsep) * \real{0.17}}@{}}
\caption{\label{tab:calibracao-conversão} Calibração dos itens da variável de Conversão}\tabularnewline
\toprule
Item & Valor ordinal & Valor Calibrado

(TFR) & Valor Calibrado

(Cluster) \\
\midrule
\endfirsthead
\toprule
Item & Valor ordinal & Valor Calibrado

(TFR) & Valor Calibrado

(Cluster) \\
\midrule
\endhead
Sim, eu me considero convertido & 1 & 1.00 & 0.98 \\
Ainda estou em processo de conversão & 2 & 0.46 & 0.81 \\
Não existe algo como conversão na minha religião & 3 & 0.13 & 0.18 \\
Não, eu não me considero convertido & 4 & 0.00 & 0.01 \\
\bottomrule
\end{longtable}
\bcenter

Fonte: Do Autor, 2021
\ecenter

Apesar dos valores de ambos serem diferentes no segundo item eles são bem similares no resto. Apesar disso optou-se pelos valores gerados pela TFR, como é indicado pela literatura especializada.

\hypertarget{abstuxeamia}{%
\subsection{Abstêmia}\label{abstuxeamia}}

As outras variáveis eram numéricas, duas escalares (abstêmia e recuperação) e uma discreta (número de pessoas ou grupos que oferecem ajuda). Dada a natureza quantitativa delas não foi necessário muito esforço para calibrar. Todas elas foram calibradas inicialmente vendo a distribuição dos dados e após pela função logística com limites fornecidos por clusterização.

A variável utilizada para apontar abstêmia foi o número de substâncias que a pessoas utilizou desde o dia da sua saída até a data da entrevista. Caso a pessoa não tivesse utilizado nada ela recebia o valor de total inclusão (1.00) e a medida que o número de substâncias utilizadas aumentava também crescia o limite de exclusão. O gráfico abaixo demonstra a variação dos dados e ajuda a visualizar as melhores áreas de corte.
\begin{figure}[H]

{\centering \includegraphics[width=0.6\linewidth]{images/imagem65-1} 

}

\caption{Distribuição da variável de Abstêmia}\label{fig:imagem65}
\end{figure}
\bcenter

Fonte: Do Autor, 2021
\ecenter  

É possível ver que não existem áreas com sobreposição ou superrepresentação. Logo, visualmente, os dados parecem ser melhor divididos em 2,3 e 4. O algoritmo de clusterização apontou valores similares: 2.5, 3.5 e 4.5. A tabela abaixo mostra o número de substâncias utilizadas e suas devidas calibrações
\begin{table}[H]

\caption{\label{tab:tab-abstemia}Calibração dos itens da variável de Abstêmia}
\centering
\begin{tabular}[t]{rr}
\toprule
\textbf{Frequência Original} & \textbf{Condição Calibrada}\\
\midrule
0 & 0.99\\
1 & 0.81\\
2 & 0.19\\
3 & 0.01\\
4 & 0.00\\
\bottomrule
\end{tabular}
\end{table}
\bcenter

Fonte: Do Autor, 2021
\ecenter

\hypertarget{recuperauxe7uxe3o}{%
\subsection{Recuperação}\label{recuperauxe7uxe3o}}

A questão de recuperação, como dito anteriormente, era uma escala de 0 a 10. De forma similar ao item anterior foi-se plotada a dispersão dos dados, de forma a identificar pontos de divisão em grupos. A imagem abaixo é o resultado desse processo.
\begin{figure}[H]

{\centering \includegraphics[width=0.6\linewidth]{images/imagem66-1} 

}

\caption{Distribuição da variável de Recuperação}\label{fig:imagem66}
\end{figure}
\bcenter

Fonte: Do Autor, 2021
\ecenter  

É possível perceber na imagem que grande parte dos dados se concentra na parte mais à direita da escala, produzindo uma vala entre o 4 e os valores à esquerda, uma pequena vala também se forma após o 8. O processo de clusterização indicou como valores de corte 3.5, 6.5 e 8.5. Os valores finais calibrados são mostrados na tabela abaixo:
\begin{table}[H]

\caption{\label{tab:tab-recover}Calibração dos itens da variável de Recuperação}
\centering
\begin{tabular}[t]{rr}
\toprule
\textbf{Frequência Original} & \textbf{Condição Calibrada}\\
\midrule
1 & 0.00\\
2 & 0.01\\
5 & 0.19\\
6 & 0.38\\
7 & 0.68\\
\addlinespace
8 & 0.90\\
9 & 0.98\\
10 & 0.99\\
\bottomrule
\end{tabular}
\end{table}
\bcenter

Fonte: Do Autor, 2021
\ecenter

\hypertarget{redes-sociais}{%
\subsection{Redes Sociais}\label{redes-sociais}}

A última variável que passou pelo processo de calibração foi a de redes sociais. Para que ela fosse construída utilizou-se a questão que mensurava quais das pessoas ou grupos de pessoas presente na questão forneceu apoio após a saída da CT. O valor total de grupos foi somado e o resultado dessa soma foi utilizado no processo de fuzzificação. O número máximo a ser obtido era 6 (suporte total) e o mínimo era 0 (nenhum suporte). A imagem abaixo demonstra a distribuição dos itens:
\begin{figure}[H]

{\centering \includegraphics[width=0.6\linewidth]{images/imagem67-1} 

}

\caption{Distribuição da variável de Redes Sociais}\label{fig:imagem67}
\end{figure}
É perceptível pela vala entre o 0 e o 2 que esse seria o primeiro dos limitadores, todos os outros pontos são preenchidos sem sobreposição. O algoritmo indicou como valores 1.0, 3.5 e 5.5. A tabela abaixo mostra a forma calibrada do dado.
\begin{table}[H]

\caption{\label{tab:tab-redes}Calibração dos itens da variável de Redes Sociais}
\centering
\begin{tabular}[t]{rr}
\toprule
\textbf{Frequência Original} & \textbf{Condição Calibrada}\\
\midrule
0 & 0.02\\
2 & 0.15\\
3 & 0.36\\
4 & 0.68\\
5 & 0.90\\
\addlinespace
6 & 0.98\\
\bottomrule
\end{tabular}
\end{table}
\bcenter

Fonte: Do Autor, 2021
\ecenter

Encerrados os processos de Calibração resta realizar a análise e discutir os resultados encontrados.

\hypertarget{qca}{%
\section{QCA}\label{qca}}

\hypertarget{truth-table-e-minimizauxe7uxe3o}{%
\subsection{Truth Table e Minimização}\label{truth-table-e-minimizauxe7uxe3o}}

Calculados os valores calibrados restava criar a tabela que conteria as combinações e seus devidos valores de suficiência. A tabela abaixo mostra os resultados da primeira análise, tendo como resultado a \emph{Abstêmia}. Foi-se utiizado como valor de inclusão 0.8, o que implica que apenas as condições iguais ou superiores a este valor serão consideradas como presença do \emph{Outcome} (OUT = 1).
\begin{longtable}[]{@{}
  >{\raggedright\arraybackslash}p{(\columnwidth - 12\tabcolsep) * \real{0.11}}
  >{\raggedright\arraybackslash}p{(\columnwidth - 12\tabcolsep) * \real{0.14}}
  >{\raggedright\arraybackslash}p{(\columnwidth - 12\tabcolsep) * \real{0.11}}
  >{\raggedright\arraybackslash}p{(\columnwidth - 12\tabcolsep) * \real{0.24}}
  >{\raggedright\arraybackslash}p{(\columnwidth - 12\tabcolsep) * \real{0.11}}
  >{\raggedright\arraybackslash}p{(\columnwidth - 12\tabcolsep) * \real{0.14}}
  >{\raggedright\arraybackslash}p{(\columnwidth - 12\tabcolsep) * \real{0.14}}@{}}
\caption{\label{tab:tt-abstemia} Tabela Verdade das Condições}\tabularnewline
\toprule
& conversão & redes & OUTCOME (abstêmia) & n & incl.cut & PRI \\
\midrule
\endfirsthead
\toprule
& conversão & redes & OUTCOME (abstêmia) & n & incl.cut & PRI \\
\midrule
\endhead
1 & 0 & 0 & 0 & 5 & 0.787 & 0.758 \\
2 & 0 & 1 & 1 & 6 & \textbf{0.830} & 0.806 \\
3 & 1 & 0 & 0 & 5 & 0.580 & 0.463 \\
4 & 1 & 1 & 0 & 8 & 0.379 & 0.258 \\
\bottomrule
\end{longtable}
\bcenter

Fonte: Do Autor, 2021
\ecenter

Nota-se, logo de início, que a única combinação que foi suficiente para explicar o processo de abstêmia foi a Presença de Redes e a Ausência de Conversão. O que indica, inicialmente, que a Conversão, dentro dos casos explorados, \emph{não tem efeito} sobre a Abstêmia. Ao contrário, não ser convertido e ter redes sociais foi o que gerou o resultado. Ainda é necessário, em ordem de confirmar os resultados, realizar a minimização da tabela, de forma de extrair a fórmula causal. Este processo foi feito utilizando o algoritmo de Quine-McCluskey e gerou o seguinte resultado:
\begin{longtable}[]{@{}
  >{\raggedright\arraybackslash}p{(\columnwidth - 8\tabcolsep) * \real{0.24}}
  >{\raggedright\arraybackslash}p{(\columnwidth - 8\tabcolsep) * \real{0.17}}
  >{\raggedright\arraybackslash}p{(\columnwidth - 8\tabcolsep) * \real{0.14}}
  >{\raggedright\arraybackslash}p{(\columnwidth - 8\tabcolsep) * \real{0.19}}
  >{\raggedright\arraybackslash}p{(\columnwidth - 8\tabcolsep) * \real{0.14}}@{}}
\caption{\label{tab:min-abstemia} Minimização da Tabela Verdade}\tabularnewline
\toprule
& inclS & PRI & CovS & CovU \\
\midrule
\endfirsthead
\toprule
& inclS & PRI & CovS & CovU \\
\midrule
\endhead
\textasciitilde conv*redes & 0.830 & 0.806 & 0.379 & - \\
M1 & 0.830 & 0.806 & 0.379 & \\
\bottomrule
\end{longtable}
\bcenter

Fonte: Do Autor, 2021
\ecenter  

A fórmula causal pode ser interpretada como:

\[ conversao * REDES \rightarrow Abstinencia \]

Como esperado, a fórmula confirma que a Conversão não tem efeito positivo sobre o outcome, dado que o mesmo só acontece em sua ausência. Apesar de constatar que, em relação a Abstêmia, ser Convertido não tem peso, resta saber de que forma ela afeta o processo de Recuperação. A tabela abaixo mostra os resultados.
\begin{longtable}[]{@{}
  >{\raggedright\arraybackslash}p{(\columnwidth - 12\tabcolsep) * \real{0.10}}
  >{\raggedright\arraybackslash}p{(\columnwidth - 12\tabcolsep) * \real{0.15}}
  >{\raggedright\arraybackslash}p{(\columnwidth - 12\tabcolsep) * \real{0.10}}
  >{\raggedright\arraybackslash}p{(\columnwidth - 12\tabcolsep) * \real{0.30}}
  >{\raggedright\arraybackslash}p{(\columnwidth - 12\tabcolsep) * \real{0.10}}
  >{\raggedright\arraybackslash}p{(\columnwidth - 12\tabcolsep) * \real{0.15}}
  >{\raggedright\arraybackslash}p{(\columnwidth - 12\tabcolsep) * \real{0.11}}@{}}
\caption{\label{tab:tt-recover} Tabela Verdade das Condições}\tabularnewline
\toprule
& conversão & redes & OUTCOME (recuperação) & n & incl.cut & PRI \\
\midrule
\endfirsthead
\toprule
& conversão & redes & OUTCOME (recuperação) & n & incl.cut & PRI \\
\midrule
\endhead
1 & 0 & 0 & 1 & 5 & \textbf{0.824} & 0.758 \\
2 & 0 & 1 & 1 & 6 & \textbf{0.910} & 0.875 \\
3 & 1 & 0 & 0 & 5 & 0.797 & 0.728 \\
4 & 1 & 1 & 0 & 8 & 0.771 & 0.710 \\
\bottomrule
\end{longtable}
\bcenter

Fonte: Do Autor, 2021
\ecenter

Percebe-se que duas combinações obtiveram scores de suficiência válidos: Uma na qual nenhuma das duas condições ocorre e uma segunda na qual apenas as redes sociais são presentes. Apesar de ser claro que, no que diz respeito a Recuperação, as redes sociais tiveram um grande papel, dado o alto valor de suficiência, a minimização da tabela demonstra que, o fator que corrobora com o processo é a ausência de conversão.
\begin{longtable}[]{@{}
  >{\raggedright\arraybackslash}p{(\columnwidth - 8\tabcolsep) * \real{0.24}}
  >{\raggedright\arraybackslash}p{(\columnwidth - 8\tabcolsep) * \real{0.17}}
  >{\raggedright\arraybackslash}p{(\columnwidth - 8\tabcolsep) * \real{0.14}}
  >{\raggedright\arraybackslash}p{(\columnwidth - 8\tabcolsep) * \real{0.19}}
  >{\raggedright\arraybackslash}p{(\columnwidth - 8\tabcolsep) * \real{0.14}}@{}}
\caption{\label{tab:min-recover} Minimização da Tabela Verdade}\tabularnewline
\toprule
& inclS & PRI & CovS & CovU \\
\midrule
\endfirsthead
\toprule
& inclS & PRI & CovS & CovU \\
\midrule
\endhead
\textasciitilde conv & 0.796 & 0.755 & 0.532 & - \\
M1 & 0.796 & 0.755 & 0.532 & \\
\bottomrule
\end{longtable}
\bcenter

Fonte: Do Autor, 2021
\ecenter  

A fórmula causal pode ser escrita como:

\[ conversao \rightarrow Recuperacao \]

Percebe-se que a Conversão, em ambos os resultados estudados, não gera efeitos positivos. Ser convertido não é algo que contribui com a abstinência, muito menos com a recuperação. A presença de redes de apoio, no entanto, se demonstrou fundamental para a explicação da Abstêmia e também tem um grande peso no processo de Recuperação, como é possível ver no gráfico de Necessidade abaixo.
\begin{figure}[H]

{\centering \includegraphics[width=0.8\linewidth]{images/imagem68-1} 

}

\caption{Relação de Necessidade}\label{fig:imagem68}
\end{figure}
\bcenter

Fonte: Do Autor, 2021
\ecenter 

Note que grande parte dos casos se concentram na área de aceitação de necessidade. O que revela que a condição é necessária para que o outcome aconteça. O alto valor de inclusão também é um indicativo disso.

\hypertarget{discussuxe3o}{%
\section{Discussão}\label{discussuxe3o}}

A análise confirmou uma parte das suspeitas propostas pelas hipóteses: A Conversão não apareceu em todas as configurações possíveis de abstêmia/recuperação (H1) e não foi necessária ou suficiente para explicar as condições nas quais esses eventos ocorreram (H2). Outras condições tiveram um peso maior do que ela e a sua ausência foi mais significativa para que o \emph{outcome} acontecesse. A QCA também conseguiu demonstrar que, mesmo em adição a outras condições, a Conversão Religiosa não produz, entre os casos estudados, efeitos significativos. Esse padrão pode ser percebido também ao se cruzar as variáveis utilizadas na análise em suas formas originais. Veja os gráficos abaixo:
\begin{figure}[H]

{\centering \includegraphics[width=0.8\linewidth]{images/imagem69-1} 

}

\caption{Conversão e Nota média na Escala de Recuperação}\label{fig:imagem69}
\end{figure}
\bcenter

Fonte: Do Autor, 2021
\ecenter 
\begin{figure}[H]

{\centering \includegraphics[width=0.8\linewidth]{images/imagem70-1} 

}

\caption{Conversão e Mediana do número de drogas utilizadas após a saída da CT}\label{fig:imagem70}
\end{figure}
\bcenter

Fonte: Do Autor, 2021
\ecenter 

Note que as notas entre Não convertidos e Convertidos na Escala de Recuperação são bem próximas. Apesar disso, o que parece ser o maior determinante do argumento aqui empregado é que os Não Convertidos foram os únicos que pontuaram 0 na mediana do número de drogas consumidas no período pós CT. Tudo isto serve como forte indicativo de que o papel que essa variável tem nos casos estudados é quase nulo. Percebe-se também, ao mesmo tempo que se constata que a Conversão não obteve peso significativo no processo, que a Religiosidade não parece ser um indicativo de recuperação, diferente do que a primeira hipótese desse estudo sugeria, dado que a análise descritiva confirmou que pessoas religiosas e não religiosas parecem experiementar os mesmos níveis de melhora.

Tais achados, mesmo que um tanto controversos, estão em consonância com o que foi aqui discutido. Vaglum \autocite*{vaglum_why_1985}, Shields \emph{et al.} \autocite*{shields_religion_2007} e até mesmo Perrone \autocite*{perrone_fatores_2019} dão alguns indícios de que a Conversão Religiosa não tem tanto peso em si e que os efeitos da Religião pertencem a outras searas como a das Relações Sociais e Pessoais. Percebe-se também que a aquisição da Gramática através da Conversão, que é o mecanismo até então utilizado para explicar a relação entre Conversão e Abstêmia, não demonstrou efeito algum entre os sujeitos pesquisados. Em outras palavras, os processos de descontinuidade não foram suficientes, ou muito menos necessários, para promover um estado de ``estar limpo''. De fato, como demonstrado, foram os não convertidos que permaneceram abstêmicos após o tratamento.

Mesmo desmistificando o efeito da Conversão, esse achado evidencia um outro processo que é relegado ao segundo plano e que se demonstrou poderoso para explicar ambos os processos analisados neste estudo: As Relações Sociais. Pouco se discute, no âmbito das categorias de cuidado, sobre uma rede de apoio para o Egresso. O foco da discussão, como visto até agora, é sobre trazer mais gente pra dentro das CT's, criar mais vagas e dominar as políticas de drogadição. O destino que essas pessoas têm após a passagem não é muito discutido. As poucas pesquisas nacionais sobre egressos que existem apontam para a importância do contato com círculos sociais, em especial a família \autocite{da2016reinserccao,ramos2018reinserccao} no processo de recuperação. Até mesmo os contatos criados através da espiritualidade/religião se tornam benéficos pela criação de novos círculos sociais, sendo a falta de apoio um dos principais motivos para o relapso \autocite{costa2001processo}. Percebe-se, por fim, que trabalhar na criação de ``Comunidades'' após a saída da ``Comunidade'' é o melhor caminho para não se precisar voltar para ela.

É importante ter em mente, no entanto, que este estudo é um dos primeiros no campo nacional a discutir efeitos de uma parte específica da Religião sobre processos referentes a Abstêmia/Recuperação no caso das CT's nacionais. É importante que se tome as contribuições aqui fornecidas com certa cautela, dado que mais estudos com metodologias variadas devam ser feitos na temática em ordem de consolidar os resultados.

Dito isto, alguns destaques ainda podem ser feitos. O primeiro é que o fato de que a Conversão Religiosa não demonstrou efeito é um indicativo de que a prática de incentivá-la deve ser revista. Não apenas por representar, em algum nível, um tabu ético, mas porque se demonstrou em estudo científico ineficaz. O segundo é que as suspeitas levantadas na segunda e terceira hipóteses se comprovaram, em parte, reais. O Governo Federal realmente está investindo em técnicas que coadunam com condições que não produzem o efeito esperado. Investir em Abstêmia e instituições que promovem Conversão/Filiação Religiosa, ao invés de investir na produção de Redes de Apoio e Segurança que possam proteger e auxiliar o egresso em sua jornada de volta a sociedade, se provou no estudo aqui feito um feito perigoso e com baixas possibilidades de retorno positivo.

Para além disso, percebe-se que as suspeitas levantadas anteriormente sobre o processo de ascenção das CT's ser apenas uma parte de um processo maior de dominação conservadora e não uma pauta de cuidado altruísta foram, em parte, confirmadas. A religiosidade empregada durante o processo não colaborou, nos casos observados, com nenhum dos processos que as próprias instituições dizem realizar.

Cabe, enfim, as autoridades e orgãos públicos a criação e investimento de estudos com maior calibre sobre a temática, em ordem de avaliar se vale realmente a pena financiar instituições que promovem comportamentos sem eficácia terapêutica. Estudos Longitudinais são mais que bem vindos nesse sentido, dado que grande parte das pesquisas focam nos sujeitos que dão entrada em CT's, seja de forma espontânea ou não. Pouco se sabe ou se procura saber sobre quem já saiu e quais efeitos (no sentido causal de efeito) as práticas as quais foram submetidas tiveram em suas vidas e no seu processo de recuperação.

\hypertarget{conclusuxe3o}{%
\chapter{Conclusão}\label{conclusuxe3o}}

Em guisa de conclusão, percebe-se que, na medida do possível, os objetivos colocados por esta pesquisa foram alcançados. Apesar dos encalços e das problemáticas com a natureza da pesquisa e da população estudada conseguiu-se com êxito entender suas jornadas e avaliar de forma efetiva os efeitos das práticas de Conversão Religiosa, em especial da Conversão Religiosa

Constatou-se que em nenhuma das fórmulas causais encontradas pela QCA a Conversão teve papel ativo ou positivo, ao contrário, foi a sua ausência que proporcionou com que ambos os outcomes estudados, abstêmia e percepção de recuperação, acontecessem. Para além disso os dados coletados também comprovam que a Conversão, diferente da narrativa vendida pelas instituições que defendem as CT's, é uma prática comum e incentivada no \emph{milieu} terapêutico.

A presente pesquisa também conseguiu traçar uma gênese dessas instituições no Brasil, tendo como ponto de partida seus encontros e desencontros com instâncias público governamentais e privadas. Conseguiu-se ainda traçar sua conexão direta com outras instituições religiosas e entender, em parte, como o projeto das CT's e o dos religiosos no poder são duas faces da mesma moeda.

Recomenda-se, enfim, a elaboração de mais pesquisas, seja no campo das metodologias configuracionais, seja no campo quantitativo convencional, que levem em consideração efeitos a longo prazo das práticas terapêuticas utilizadas por essas instituições que crescem em poder e quantidade no solo nacional. Se faz necessário ultrapassar a barreira do apoio exclusivo do discurso que as metologias qualitativas construíram e começar a coletar outros tipos de evidência que possam comprovar ou contestar a narrativa que as CT's vem construindo de si e do seu trabalho.

\postextual

\begingroup

\printbibliography[title=REFERÊNCIAS]

\endgroup

\markboth{Referências}{REFERÊNCIAS}

\hypertarget{refs}{}
\begin{CSLReferences}{0}{0}
\end{CSLReferences}
\banexos

\hypertarget{script-da-anuxe1lise}{%
\chapter{\texorpdfstring{\textbf{SCRIPT DA ANÁLISE}}{SCRIPT DA ANÁLISE}}\label{script-da-anuxe1lise}}
\begin{Shaded}
\begin{Highlighting}[]
\FunctionTok{library}\NormalTok{ (QCA)}
\FunctionTok{library}\NormalTok{ (readxl)}
\FunctionTok{library}\NormalTok{ (tidyverse)}

\CommentTok{\#banco de dados}
\NormalTok{bd }\OtherTok{\textless{}{-}} \FunctionTok{read\_excel}\NormalTok{(}\StringTok{"\textasciitilde{}/tr\_02alpha/Untitled/data/dados\_mestrado.xlsx"}\NormalTok{)}

\CommentTok{\#seleção dos dados}
\NormalTok{dt }\OtherTok{\textless{}{-}}\NormalTok{ bd }\SpecialCharTok{\%\textgreater{}\%} 
  \FunctionTok{select}\NormalTok{(PCT\_n, FS06, FS05\_n ,R06\_n) }\SpecialCharTok{\%\textgreater{}\%} 
  \FunctionTok{drop\_na}\NormalTok{()}

\CommentTok{\# calibração}
\NormalTok{dt}\SpecialCharTok{$}\NormalTok{conv }\OtherTok{\textless{}{-}} \FunctionTok{calibrate}\NormalTok{(dt}\SpecialCharTok{$}\NormalTok{R06\_n, }\AttributeTok{type =} \StringTok{"fuzzy"}\NormalTok{, }
                     \AttributeTok{thresholds =} \FunctionTok{c}\NormalTok{( }\FloatTok{1.5}\NormalTok{, }\FloatTok{2.5}\NormalTok{, }\FloatTok{3.5}\NormalTok{), }
                     \AttributeTok{logistic =}\NormalTok{ T)}
\NormalTok{dt}\SpecialCharTok{$}\NormalTok{recov }\OtherTok{\textless{}{-}} \FunctionTok{calibrate}\NormalTok{(dt}\SpecialCharTok{$}\NormalTok{FS06, }\AttributeTok{type =} \StringTok{"fuzzy"}\NormalTok{, }
                      \AttributeTok{thresholds =} \FunctionTok{c}\NormalTok{(}\FloatTok{3.5}\NormalTok{,}\FloatTok{6.5}\NormalTok{,}\FloatTok{8.5}\NormalTok{), }
                      \AttributeTok{logistic =}\NormalTok{ T)}
\NormalTok{dt}\SpecialCharTok{$}\NormalTok{abst }\OtherTok{\textless{}{-}} \FunctionTok{calibrate}\NormalTok{(dt}\SpecialCharTok{$}\NormalTok{PCT\_n, }\AttributeTok{type =} \StringTok{"fuzzy"}\NormalTok{, }
                     \AttributeTok{thresholds =} \FunctionTok{c}\NormalTok{(}\FloatTok{2.5}\NormalTok{,}\FloatTok{3.5}\NormalTok{,}\FloatTok{4.5}\NormalTok{), }
                     \AttributeTok{logistic =}\NormalTok{ T)}
\NormalTok{dt}\SpecialCharTok{$}\NormalTok{redes }\OtherTok{\textless{}{-}} \FunctionTok{calibrate}\NormalTok{(dt}\SpecialCharTok{$}\NormalTok{FS05\_n, }\AttributeTok{type =} \StringTok{"fuzzy"}\NormalTok{, }
                      \AttributeTok{thresholds =} \FunctionTok{c}\NormalTok{(}\FloatTok{1.0}\NormalTok{,}\FloatTok{3.5}\NormalTok{,}\FloatTok{5.5}\NormalTok{), }
                      \AttributeTok{logistic =}\NormalTok{ T)}

\CommentTok{\#banco de dados fuzzy}
\NormalTok{fuz }\OtherTok{\textless{}{-}} \FunctionTok{data.frame}\NormalTok{(dt[,}\DecValTok{5}\SpecialCharTok{:}\DecValTok{8}\NormalTok{])}


\CommentTok{\#truthtable}
\NormalTok{t1 }\OtherTok{\textless{}{-}} \FunctionTok{truthTable}\NormalTok{(fuz, }\AttributeTok{outcome =} \StringTok{"abst"}\NormalTok{,}
                 \AttributeTok{conditions =} \FunctionTok{c}\NormalTok{(}\StringTok{"conv"}\NormalTok{, }\StringTok{"redes"}\NormalTok{),}
                 \AttributeTok{incl.cut =}\NormalTok{ .}\DecValTok{8}\NormalTok{,}\AttributeTok{complete =}\NormalTok{ T, }\AttributeTok{show.cases =}\NormalTok{ T, }
                 \AttributeTok{sort.by =} \FunctionTok{c}\NormalTok{(}\StringTok{"incl"}\NormalTok{, }\StringTok{"n"}\NormalTok{))}

\NormalTok{t2 }\OtherTok{\textless{}{-}} \FunctionTok{truthTable}\NormalTok{(fuz, }\AttributeTok{outcome =} \StringTok{"recov"}\NormalTok{,}
                 \AttributeTok{conditions =} \FunctionTok{c}\NormalTok{(}\StringTok{"conv"}\NormalTok{, }\StringTok{"redes"}\NormalTok{),}
                 \AttributeTok{incl.cut =}\NormalTok{ .}\DecValTok{8}\NormalTok{,}\AttributeTok{complete =}\NormalTok{ T, }\AttributeTok{show.cases =}\NormalTok{ T, }
                 \AttributeTok{sort.by =} \FunctionTok{c}\NormalTok{(}\StringTok{"incl"}\NormalTok{, }\StringTok{"n"}\NormalTok{))}

\CommentTok{\#minimizações}
\FunctionTok{minimize}\NormalTok{(t1, }\AttributeTok{details =}\NormalTok{ T, }\AttributeTok{show.cases =}\NormalTok{ T, }\AttributeTok{use.tilde =}\NormalTok{ F)}
\FunctionTok{minimize}\NormalTok{(t2, }\AttributeTok{details =}\NormalTok{ T, }\AttributeTok{show.cases =}\NormalTok{ T, }\AttributeTok{use.tilde =}\NormalTok{ F)}
\end{Highlighting}
\end{Shaded}
\hypertarget{questionuxe1rio}{%
\chapter{\texorpdfstring{\textbf{QUESTIONÁRIO}}{QUESTIONÁRIO}}\label{questionuxe1rio}}
\begin{center}\includegraphics[width=1.03\linewidth]{images/quest_final-1} \end{center}
\begin{center}\includegraphics[width=1.03\linewidth]{images/quest_final-2} \end{center}
\begin{center}\includegraphics[width=1.03\linewidth]{images/quest_final-3} \end{center}
\begin{center}\includegraphics[width=1.03\linewidth]{images/quest_final-4} \end{center}
\begin{center}\includegraphics[width=1.03\linewidth]{images/quest_final-5} \end{center}
\begin{center}\includegraphics[width=1.03\linewidth]{images/quest_final-6} \end{center}
\begin{center}\includegraphics[width=1.03\linewidth]{images/quest_final-7} \end{center}
\begin{center}\includegraphics[width=1.03\linewidth]{images/quest_final-8} \end{center}

\hypertarget{termo-de-livre-esclarecimento-tcle}{%
\chapter{\texorpdfstring{\textbf{TERMO DE LIVRE ESCLARECIMENTO (TCLE)}}{TERMO DE LIVRE ESCLARECIMENTO (TCLE)}}\label{termo-de-livre-esclarecimento-tcle}}
\begin{center}\includegraphics[width=1.03\linewidth]{images/TCLE_ATUALIZADO-1} \end{center}
\begin{center}\includegraphics[width=1.03\linewidth]{images/TCLE_ATUALIZADO-2} \end{center}

\hypertarget{parecer-do-comituxea-de-uxe9tica}{%
\chapter{\texorpdfstring{\textbf{PARECER DO COMITÊ DE ÉTICA}}{PARECER DO COMITÊ DE ÉTICA}}\label{parecer-do-comituxea-de-uxe9tica}}
\begin{center}\includegraphics[width=1.03\linewidth]{images/PB_PARECER_CONSUBSTANCIADO_CEP_4607384-1} \end{center}
\begin{center}\includegraphics[width=1.03\linewidth]{images/PB_PARECER_CONSUBSTANCIADO_CEP_4607384-2} \end{center}
\begin{center}\includegraphics[width=1.03\linewidth]{images/PB_PARECER_CONSUBSTANCIADO_CEP_4607384-3} \end{center}
\begin{center}\includegraphics[width=1.03\linewidth]{images/PB_PARECER_CONSUBSTANCIADO_CEP_4607384-4} \end{center}

\eanexos

% ----------------------------------------------------------
% Glossário
% ----------------------------------------------------------
%
% Consulte o manual da classe abntex2 para orientações sobre o glossário.
%
%\glossary

%---------------------------------------------------------------------
% INDICE REMISSIVO
%---------------------------------------------------------------------
%\phantompart
%\printindex
%---------------------------------------------------------------------

\end{document}
