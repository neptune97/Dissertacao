% ------------------------------------------------------------------------
% ------------------------------------------------------------------------
% Modelo UFSC para Trabalhos Academicos (tese de doutorado, dissertação de
% mestrado) utilizando a classe abntex2
%
% Autor: Alisson Lopes Furlani
% 	Modificações:
%	- 27/08/2019: Alisson L. Furlani, add pacote 'glossaries' para listas
% - 30/10/2019: Alisson L. Furlani, adjusted some spacing errors and changed math fonts
% - 17/01/2019: Alisson L. Furlani, updated certification page
% - 03/03/2020: Luiz F. P. Droubi, change file to be used as a template with R.
% ------------------------------------------------------------------------
% ------------------------------------------------------------------------

\documentclass[
	% -- opções da classe memoir --
	12pt,				% tamanho da fonte
	%openright,			% capítulos começam em pág ímpar (insere página vazia caso preciso)
	oneside,			% para impressão no anverso. Oposto a twoside
	a4paper,			% tamanho do papel.
	sumario=tradicional,
	% -- opções da classe abntex2 --
	%chapter=TITLE,		% títulos de capítulos convertidos em letras maiúsculas
	%section=TITLE,		% títulos de seções convertidos em letras maiúsculas
	%subsection=TITLE,	% títulos de subseções convertidos em letras maiúsculas
	%subsubsection=TITLE,% títulos de subsubseções convertidos em letras maiúsculas
	% -- opções do pacote babel --
	english,			% idioma adicional para hifenização
	%french,				% idioma adicional para hifenização
	%spanish,			% idioma adicional para hifenização
	brazil				% o último idioma é o principal do documento
	]{abntex2}

\usepackage{setup/ufscthesisA4-alf}

\DisemulatePackage{setspace}
\usepackage{setspace}

\usepackage{quoting}

\usepackage[bottom]{footmisc}

$for(bibliography)$
\addbibresource{$bibliography$}
$endfor$

\usepackage[table]{xcolor}
\let\newfloat\undefined
\usepackage{floatrow}
\floatsetup[table]{capposition=top}
\floatsetup[figure]{capposition=top}

\newcommand{\pkg}[1]{{\normalfont\fontseries{b}\selectfont #1}}
\let\proglang=\textsf
\let\code=\texttt

$if(csl-refs)$
\newlength{\cslhangindent}
\setlength{\cslhangindent}{1.5em}
\newenvironment{CSLReferences}%
  {$if(csl-hanging-indent)$\setlength{\parindent}{0pt}%
  \everypar{\setlength{\hangindent}{\cslhangindent}}\ignorespaces$endif$}%
  {\par}
$endif$ 

$if(highlighting-macros)$
$highlighting-macros$
$endif$

\newcommand{\bcenter}{\begin{center}}
\newcommand{\ecenter}{\end{center}}

\newcommand{\bapendices}{\begin{apendicesenv}}
\newcommand{\eapendices}{\end{apendicesenv}}

\newcommand{\banexos}{\begin{anexosenv}}
\newcommand{\eanexos}{\end{anexosenv}}

% ---
% Filtering and Mapping Bibliographies
% ---
\DeclareSourcemap{
	\maps[datatype=bibtex]{
		% remove fields that are always useless
		\map{
			\step[fieldset=abstract, null]
			\step[fieldset=pagetotal, null]
			\step[fieldset= doi, null]
		}
%		 remove URLs for types that are primarily printed
		\map{
			\pernottype{software}
			\pernottype{online}
			\pernottype{report}
			\pernottype{techreport}
			\pernottype{standard}
			\pernottype{manual}
			\pernottype{misc}
			\step[fieldset=url, null]
			\step[fieldset=urldate, null]
		}
		\map{
			\pertype{inproceedings}
			% remove mostly redundant conference information
			\step[fieldset=venue, null]
			\step[fieldset=eventdate, null]
			\step[fieldset=eventtitle, null]
			% do not show ISBN for proceedings
			\step[fieldset=isbn, null]
			% Citavi bug
			\step[fieldset=volume, null]
		}
	}
}
% ---

% ---
% Informações de dados para CAPA e FOLHA DE ROSTO
% ---
% FIXME Substituir 'Nome completo do autor' pelo seu nome.
\autor{$author$}
% FIXME Substituir 'Título do trabalho' pelo título da trabalho.
\titulo{$title$}
% FIXME Substituir 'Subtítulo (se houver)' pelo subtítulo da trabalho.
% Caso não tenha substítulo, comente a linha a seguir.
$if(subtitle)$
  \subtitulo{$subtitle$}
$endif$
% FIXME Substituir 'XXXXXX' pelo nome do seu
% orientador.
\orientador{$advisor$}
% FIXME Se for orientado por uma mulher, comente a linha acima e descomente a linha a seguir.
% \orientador[Orientadora]{Nome da orientadora, Dra.}
% FIXME Substituir 'XXXXXX' pelo nome do seu
% coorientador. Caso não tenha coorientador, comente a linha a seguir.
$if(altadvisor)$
\coorientador{$altadvisor$}
$endif$
% FIXME Se for coorientado por uma mulher, comente a linha acima e descomente a linha a seguir.
% \coorientador[Coorientadora]{XXXXXX, Dra.}
% FIXME Substituir '[ano]' pelo ano (ano) em que seu trabalho foi defendido.
\ano{$date_year$}
% FIXME Substituir '[dia] de [mês] de [ano]' pela data em que ocorreu sua defesa.
\data{$date_day$ de $date_month$ de $date_year$}
% FIXME Substituir 'Local' pela cidade em que ocorreu sua defesa.
\local{$local$}
\instituicaosigla{$inst_short$}
\instituicao{$institution$}
% FIXME Substituir 'Dissertação/Tese' pelo tipo de trabalho (Tese, Dissertação).
\tipotrabalho{$doc_type$}
% FIXME Substituir '[mestre/doutor] em XXXXXX' pela grau adequado.
\formacao{$degree$}
% FIXME Substituir '[mestrado/doutorado]' pelo nivel adequado.
\nivel{$level$}
% FIXME Substituir 'Programa de Pós-Graduação em XXXXXX' pela curso adequado.
\programa{$department$}
% FIXME Substituir 'Campus XXXXXX ou Centro de XXXXXX' pelo campus ou centro adequado.
\centro{$division$}
\preambulo
{%
\imprimirtipotrabalho~submetida~ao~\imprimirprograma~da~\imprimirinstituicao~para~a~obtenção~do~título~de~\imprimirformacao.
}
% ---

% ---
% Configurações de aparência do PDF final
% ---
% alterando o aspecto da cor azul
\definecolor{blue}{RGB}{41,5,195}
% informações do PDF
\makeatletter
\hypersetup{
     	%pagebackref=true,
		pdftitle={\@title},
		pdfauthor={\@author},
    	pdfsubject={\imprimirpreambulo},
	    pdfcreator={LaTeX with abnTeX2},
		pdfkeywords={ufsc, latex, abntex2},
		colorlinks=true,       		% false: boxed links; true: colored links
    	linkcolor=black,%blue,          	% color of internal links
    	citecolor=black,%blue,        		% color of links to bibliography
    	filecolor=black,%magenta,      		% color of file links
		urlcolor=black,%blue,
		bookmarksdepth=4
}
\makeatother
% ---

% ---
% compila a lista de abreviaturas e siglas e a lista de símbolos
% ---

% Declaração das siglas
%lista de siglas
\siglalista{CT}{Comunidades Terap{\^e}uticas}
\siglalista{CNAE}{Classifica{\c c}{\~a}o Nacional de Atividades Eco{\^o}micas}
\siglalista{CFESS}{Conselho Federal de Serviço Social}
\siglalista{CFP}{Conselho Federal de Psicologia}
\siglalista{RAPS}{Rede de Assist{\^e}ncia Psicossocial}
\siglalista{CAPS}{Centro de Assist{\^e}ncia Psicossocial}
\siglalista{CAPS AD}{Centro de Assist{\^e}ncia Psicossocial {\'A}lcool e Drogas}
\siglalista{SENAD}{Secretaria Nacional de Pol{\'i}tica sobre Drogas}
\siglalista{PRD}{Programa de Reduç{\~a}o de Danos}
\siglalista{SUPERA}{Sistema para Detec{\c c}{\~a}o do Uso Abusivo e Dependência de Subst{\^a}ncias Psicoativas: Encaminhamento, interven{\c c}{\~a}o Breve, Reinser{\c c}{\~a}o Social e Acompanhamento}
\siglalista{PEAD}{Plano Emergencial de Amplia{\c c}{\~a}o do Acesso ao Tratamento e {\`a} Preven{\c c}{\~a}o em {\'A}lcool e outras Drogas}
\siglalista{FEBRACT}{Federa{\c c}{\~a}o Brasileira de Comunidades Terap{\^e}uticas}
\siglalista{HIV}{Human Immunodeficiency Virus}
\siglalista{CONFENACT}{Confedera{\c c}{\~a}o Nacional de Comunidades Terap{\^e}uticas}
\siglalista{FETEB}{Federa{\c c}{\~a}o de Comunidades Terap{\^e}uticas Evang{\'e}licas do Brasil}
\siglalista{ANVISA}{Ag{\^e}ncia Nacional de Vigil{\^a}ncia Sanit{\'a}ria}
\siglalista{QCA}{Qualitative Comparative Analysis}
\siglalista{SUS}{Sistema {\'U}nico de Sa{\'u}de}
\siglalista{INUS}{Insufficiente, but Necessary}
\siglalista{SUIN}{Sufficiente, but Unnecessary}
\siglalista{MJ}{Minist{\'e}rio da Justi{\c c}a}
\siglalista{MDS}{Minist{\'e}rio do Desenvolvimento Social}
\siglalista{MNPCT}{Mecanismo Nacional de Preven{\c c}{\~a}o e Combate {\`a} Tortura}
\siglalista{PFDC}{Procuradoria Federal dos Direitos do Cidad{\~a}o}
\siglalista{PNAD}{Pol{\'i}ticas Nacionais de Drogas}
\siglalista{SENAPRED}{Secretaria Nacional de Cuidados e Preven{\c c}{\~a}o {\`a}s Drogas}
\siglalista{ECRIECT}{Efeitos da Convers{\~a}o Religiosa entre Egressos de CT}
\siglalista{A.A}{Alco{\'o}licos An{\^o}nimos}
\siglalista{CNPJ}{Cadastro Nacional de Pessoa Jur{\'i}dica}
\siglalista{IPEA}{Instituto de Pesquisa Econ{\^o}mica Aplicada}
\siglalista{RDC}{Regime Diferenciado de Contrata{\c c}{\~a}o}
\siglalista{OBID}{Observat{\'o}rio Brasileiro de Informa{\c c}{\~o}es sobre Droga}
% Declaração dos simbolos
%lista de simbolos
\simbololista{Reais}{$ \Re $}{N{\'u}meros reais}
\simbololista{Fuzzy}{$ \Sigma $}{Somat{\'o}rio}
\simbololista{efeito}{$ \mu $}{Mi}
\simbololista{total}{$ \subset $}{Subconjunto}
\simbololista{Semi}{$ \cap $}{Uni{\~a}o}
\simbololista{ou}{$ \cup $}{Intercess{\~a}o}
\simbololista{pertencimento}{$ \not\subset $}{N{\~a}o Pertencimento}
\simbololista{carga}{$ \alpha $}{Alpha de Crombach}

% compila a lista de abreviaturas e siglas e a lista de símbolos
\makenoidxglossaries

% ---

% ---
% compila o indice
% ---
\makeindex
% ---

% ----
% Início do documento
% ----
\begin{document}

% Seleciona o idioma do documento (conforme pacotes do babel)
%\selectlanguage{english}
\selectlanguage{brazil}

% Retira espaço extra obsoleto entre as frases.
\frenchspacing

% Espaçamento 1.5 entre linhas
\OnehalfSpacing

% Corrige justificação
%\sloppy

% ----------------------------------------------------------
% ELEMENTOS PRÉ-TEXTUAIS
% ----------------------------------------------------------
% \pretextual %a macro \pretextual é acionado automaticamente no início de \begin{document}
% ---
% Capa, folha de rosto, ficha bibliografica, errata, folha de apróvação
% Dedicatória, agradecimentos, epígrafe, resumos, listas
% ---
% ---
% Capa
% ---
\imprimircapa
% ---

% ---
% Folha de rosto
% (o * indica que haverá a ficha bibliográfica)
% ---
\imprimirfolhaderosto*
% ---

% ---
% Inserir a ficha bibliografica
% ---
% http://ficha.bu.ufsc.br/
\begin{fichacatalografica}
	\includepdf{Ficha_Catalografica.pdf}
\end{fichacatalografica}
% ---

% ---
% Inserir folha de aprovação
% ---
\begin{folhadeaprovacao}
	\OnehalfSpacing
	\centering
	\imprimirautor\\%
	\vspace*{10pt}		
	\textbf{\imprimirtitulo}%
	\ifnotempty{\imprimirsubtitulo}{:~\imprimirsubtitulo}\\%
	%		\vspace*{31.5pt}%3\baselineskip
	\vspace*{\baselineskip}
	%\begin{minipage}{\textwidth}
	O presente trabalho em nível de \imprimirnivel~foi avaliado e aprovado por banca examinadora composta pelos seguintes membros:\\
	%\end{minipage}%
	\vspace*{\baselineskip}
  $for(examiner)$
  $examiner.title$ $examiner.name$, $examiner.degree$\\
  $examiner.institution$ - $examiner.inst_short$\\
  \vspace*{\baselineskip}
  $endfor$
  
	\vspace*{2\baselineskip}
	\begin{minipage}{\textwidth}
		Certificamos que esta é a \textbf{versão original e final} do trabalho de conclusão que foi julgado adequado para obtenção do título de \imprimirformacao.\\
	\end{minipage}
	%    \vspace{-0.7cm}
	\vspace*{\fill}
	\assinatura{\OnehalfSpacing $coordinator$ \\ Coordenação do Programa de Pós-Graduação}
	\vspace*{\fill}
	\assinatura{\OnehalfSpacing\imprimirorientador \\ \imprimirorientadorRotulo}
	%	\ifnotempty{\imprimircoorientador}{
	%	\assinatura{\imprimircoorientador \\ \imprimircoorientadorRotulo \\
	%		\imprimirinstituicao~--~\imprimirinstituicaosigla}
	%	}
	% \newpage
	\vspace*{\fill}
	\imprimirlocal, \imprimirano.
\end{folhadeaprovacao}
% ---

% ---
% Dedicatória
% ---
\begin{dedicatoria}
	\vspace*{\fill}
	\noindent
	\begin{adjustwidth*}{}{5.5cm} 
		\raggedleft       
		$dedication$
	\end{adjustwidth*}
\end{dedicatoria}
% ---

% ---
% Agradecimentos
% ---
\begin{agradecimentos}
	$thanks$
\end{agradecimentos}
% ---

% ---
% Epígrafe
% ---
\begin{epigrafe}
	\vspace*{\fill}
	\begin{flushright}
		\textit{$epigrafe$}
	\end{flushright}
\end{epigrafe}
% ---

% ---
% RESUMOS
% ---

% resumo em português
\setlength{\absparsep}{18pt} % ajusta o espaçamento dos parágrafos do resumo
\begin{resumo}
	\SingleSpacing
  $abstract$ 
  
  \textbf{Palavras-chave}: 
  $for(palavras-chave)$
  $palavras-chave$.
  $endfor$
\end{resumo}

% resumo em inglês
\begin{resumo}[Abstract]
	\SingleSpacing
	\begin{otherlanguage*}{english}
		$foreignabstract$
		
		\textbf{Keywords}:
	  $for(keyword)$
    $keyword$.
    $endfor$
	\end{otherlanguage*}
\end{resumo}

%% resumo em francês 
%\begin{resumo}[Résumé]
% \begin{otherlanguage*}{french}
%    Il s'agit d'un résumé en français.
% 
%   \textbf{Mots-clés}: latex. abntex. publication de textes.
% \end{otherlanguage*}
%\end{resumo}
%
%% resumo em espanhol
%\begin{resumo}[Resumen]
% \begin{otherlanguage*}{spanish}
%   Este es el resumen en español.
%  
%   \textbf{Palabras clave}: latex. abntex. publicación de textos.
% \end{otherlanguage*}
%\end{resumo}
%% ---

{%hidelinks
	\hypersetup{hidelinks}
	% ---
	% inserir lista de ilustrações
	% ---
	\pdfbookmark[0]{\listfigurename}{lof}
	\listoffigures
	\cleardoublepage
	% ---
	
	
	% ---
	% inserir lista de tabelas
	% ---
	\pdfbookmark[0]{\listtablename}{lot}
	\listoftables
	\cleardoublepage
	% ---
	
	% ---
	% inserir lista de abreviaturas e siglas (devem ser declarados no preambulo)
	% ---
	\imprimirlistadesiglas
	% ---
	
	% ---
	% inserir lista de símbolos (devem ser declarados no preambulo)
	% ---
	\imprimirlistadesimbolos
	% ---
	
	% ---
	% inserir o sumario
	% ---
	\pdfbookmark[0]{\contentsname}{toc}
	\tableofcontents*
	\cleardoublepage
	
}%hidelinks
% ---

% ---

% ----------------------------------------------------------
% ELEMENTOS TEXTUAIS
% ----------------------------------------------------------
\textual

$body$

% ----------------------------------------------------------
% Glossário
% ----------------------------------------------------------
%
% Consulte o manual da classe abntex2 para orientações sobre o glossário.
%
%\glossary

%---------------------------------------------------------------------
% INDICE REMISSIVO
%---------------------------------------------------------------------
%\phantompart
%\printindex
%---------------------------------------------------------------------

\end{document}
